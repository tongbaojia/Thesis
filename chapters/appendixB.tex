%!TEX root = ../dissertation.tex
\chapter{Data Analysis Workflow}
\label{AppendixB}

\paragraph{}
Data analysis is done everywhere, not just in physics. Usually, the data analysis workflow is slipt into eight steps:
\begin{enumerate}
	\item Goal. Define the goal of the study. What ultimate feature is to be searched/measured? Why is it important?
	\item Theory. What's the current status of this field? What is the theory behind this feature? What's the best method to reach the goal? 
	\item Data. What is the necessary dataset? How to produce the data with limited resourced? How much data is needed?
	\item Selection. How to clean the dataset? How to make the variables meaningful? How to select the ultimate feature?
	\item Background. How to estimate the post-selection composition of the dataset? How to validate the estimations? 
	\item Uncertainty. What's the uncerainties on the theory, dataset collection, dataset selection and estimation?
	\item Result. Inside the data, is the ultimate feature present? What's the significance of the feature? How confident is the result?
	\item Interpretation. Is the goal of the study reached? If not, how to improve the analysis?
\end{enumerate}
