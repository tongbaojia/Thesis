%!TEX root = ../dissertation.tex
\begin{savequote}[75mm]
Pick the berry.
\qauthor{Bear}
\end{savequote}

\chapter{Event Selection}


\section{Optimization Strategy}
\paragraph{}
To improve the analysis, a quantity which defines the sensitivity of analysis is maximized. Techicially, the optimal sensitivity is described as $\sqrt{2((S+B)\ln{1 + \frac{S}{B}} - S}$, see \href{https://www.pp.rhul.ac.uk/~cowan/stat/notes/SigCalcNote.pdf}{note}. This is usually considered at the $S << B$ limit and simplified as $\frac{S}{\sqrt{B}}$, when no knowledge of the model cross section is available, or $\frac{S}{\sqrt{S + B}}$, if the signal cross section is known. These parametrizations have limitations, particularly when the signal yield and the number of estiamted background are both small. A better parametrization for low signal strengh is $\frac{S}{\sqrt{1 + B}}$, where the extra $\sqrt{1 + B}$ accounts for poisson fluctuations. For a discussion of p-values, please see this \href{https://arxiv.org/pdf/hep-ex/0208005.pdf}{note}.

\paragraph{}
For this analysis, two methods are used: one is calculate the number of signal and backgrounds within $68\%$ of the signal \mhh mass window, the other is to implement the full signal and background predicions after smoothing and compare the asymptotic expected exclusion limits. Both methods yield comparable results.

\paragraph{}
To avoid bias during the optimzation process, data's signal regions are blinded.


\section{$b$ tagging}
\paragraph{}
\b-tagging, which is the identification of the \b hadron,  ~\cite{Reco-btag-2016} is the core and main limiting factor of this analysis. Because of the relatively long lifetime, it is possible to tag the \b hadron using the inner detector informations. A higher \b-tagging efficiency will increase the signal selection efficiency, while a lower \b-tagging fake rate will reduce the background like $gg \to c\bar{c}$ in the signal regions.


\section{Signal Efficiency}

\paragraph{}
Acceptance referes to purely geometric fiducial volume of the detector. Efficiency refers to purely detector effectivenss in finding objects. The final value in the study is Acceptance $\times$ Efficiency, where both effects are considered.

















