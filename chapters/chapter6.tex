%!TEX root = ../dissertation.tex
\begin{savequote}[75mm]
Even in the darkest night, stars and angels still shine bright.   
%\qauthor{Bear}
\end{savequote}


\chapter{Background Estimation}

\section{Overview}
\paragraph{}
The invariant mass of the two-Higgs-boson-candidate system, \mtwoJ, is used as the final discriminant between Higgs boson pair production and SM backgrounds.
Multi-jet (QCD) is the dominant background.
About $40\%$ of the total QCD events are $gg \to b\bar{b}b\bar{b}$, and the rest $60\%$ are $gg \to c\bar{c}c\bar{c}$ or $gg \to q\bar{q}q\bar{q}$ where the quark-jets fake multiple $b$-jets.
There are neither accurate nor high-statistics MC samples with three or four $b$-tagged track jets contained in two high-\pt~ large-\R jets.
A data-driven method is used for estimating both the yield and kinematic distribution of the QCD events.
For \ttbar~ backgrounds, MC samples are available, which provide \ttbar~ kinematic distributions.
The \ttbar~ yield is also estimated using the same data-driven method to avoid mis-modeling in the MC.
The $Z$+jets background is small, and it is estimated by the $Z$+heavy flavor jets MC.
The SM $ZZ\to b\bar{b}b\bar{b}$ has been estimated to be completely negligible using a MC based analysis.
For the three signal channels $4b/3b/2bs$, the fraction of expected backgrounds are:
\begin{itemize}
	\item $4b$: QCD $\sim 95\%$, $t\bar{t}$ $\sim 5\%$, $Z$+jets$< 1\%$. 
	\item $3b$: QCD $\sim 90\%$, $t\bar{t}$ $\sim 10\%$, $Z$+jets $< 1\%$.  
	\item $2bs$: QCD $\sim 80\%$, $t\bar{t}$ $\sim 20\%$, $Z$+jets $< 1\%$.
\end{itemize}

All of the kinematic distributions of the QCD background are estimated from low signal yield data $b$-tag channels.
These channels use the signal region selection, with the exception of having a different number of $b$-tagged track jets.
The differences in kinematic distributions between the fewer $b$-tagged and $n$-tagged signal channels are corrected by reweighting the fewer $b$-tagged samples.

\paragraph{}
The fewer $b$-tagged channels estimate the shapes of the expected background, but do not estimate the total yield.
The \textit{sideband region} (SB), with the same number of $b$-tags as the signal regions but a cut on \mleadJ-\msublJ different from the $X_{hh}$ cut, is used to estimate the yield.
A third region, called the \textit{control region} (CR), is used to validate the background estimations.
The sideband and control regions are optimized to accurately estimate the rate of QCD and \ttbar~ backgrounds. 

\paragraph{}
The definitions of sideband region and control region are introduced in section~\ref{sec:boosted-SBCR}.
The QCD and \ttbar~ background shape templates are introduced in section~\ref{sec:boosted-qcd} and section~\ref{sec:boosted-ttbar}.
The normalization estimation for QCD and \ttbar~ is introduced in section~\ref{sec:ttbarnorm}.
The reweighting procedure is explained in section~\ref{sec:boosted-reweight}.
The sideband region and control region distributions are shown in section~\ref{sec:boosted-sb} and section~\ref{sec:boosted-cr}.
The signal region prediction rescaling and smoothing are discussed in section~\ref{sec:boosted-SR-smoothing}.
The yields are listed section~\ref{sec:yields}.

%%%%%%%%%%%%%%%%%%%%%%%%%%%%%%%%%%%%%%%%%%%%%%%%%%%%%%%%%%%%%%%%%%%%%%%
%%%%%%%%%%%%%%%%%%%%%%%%%%%%%%%%%%%%%%%%%%%%%%%%%%%%%%%%%%%%%%%%%%%%%%%
%%%  SB. CR
%%%%%%%%%%%%%%%%%%%%%%%%%%%%%%%%%%%%%%%%%%%%%%%%%%%%%%%%%%%%%%%%%%%%%%%
%%%%%%%%%%%%%%%%%%%%%%%%%%%%%%%%%%%%%%%%%%%%%%%%%%%%%%%%%%%%%%%%%%%%%%%

\section{Definition of the sideband and control regions}
\label{sec:boosted-SBCR}

\paragraph{}
A circular variable $R_{hh}$ is defined in the 2D \mleadJ-\msublJ mass plane. 
It has the same central values as $X_{hh}$, but without resolution terms in the denominators:
\begin{equation}
\label{eq:boosted_RhhDef}
R_{hh} = \sqrt{\left(m^{\rm lead}_{\rm J} - \text{124 GeV}\right)^2 + \left(m^{\rm subl}_{\rm J} - \text{115 GeV}\right)^2}
\end{equation}

\paragraph{}
Similarly, $R_{hh}^{\text{high}}$, the circular region with the previous central values shifted up by $10$ \GeV, is defined as:
\begin{equation}
\label{eq:boosted_RhhhighDef}
R_{hh}^{\text{high}} = \sqrt{\left(m^{\rm lead}_{\rm J} - \text{134 GeV}\right)^2 + \left(m^{\rm subl}_{\rm J} - \text{125 GeV}\right)^2}
\end{equation}

\paragraph{}
The definitions of the sideband region, control region, and signal region in the 2D \mleadJ-\msublJ plane are listed in Table~\ref{tab:boosted-sbcr-constraints}.
These regions are shown in Figure~\ref{fig:boosted-region-def}.
This Figure also shows that the QCD background rate is decreasing from the low \mleadJ-\msublJ mass region to high \mleadJ-\msublJ mass region in the 2D plane.
The \ttbar~ events are also distinguishable as the vertical cyan band where \mleadJ $\sim 173$ \GeV.

\begin{table}[htb!]
\begin{center}
\caption{Definitions of the signal region, the control region, and the sideband region.}
\begin{tabular}{c|c}
\hline
  Region                                      & Definition \\
  \hline
  signal region (SR) & $X_{hh}$ < 1.6\\
  control region (CR) & $R_{hh}$ < 33~\GeV\ and $X_{hh}$ > 1.6 \\
  sideband region (SB) & 33~\GeV < $R_{hh}$ and $R_{hh}^{\text{high}}$ < 58~\GeV
  \end{tabular}
\label{tab:boosted-sbcr-constraints}
\end{center}
\end{table}

\paragraph{}
The control region is as close to the signal region as possible.
The ring shape around the SR allows a good test for the background predictions with different \mleadJ-\msublJ combinations, and avoids too few or too many \ttbar~ events.
The current control region gives $N_{CR} \sim 2 N_{SR}$, which contains good statistics and does not affect the sideband region design.

\paragraph{}
The sideband region must be a reasonable approximation for the QCD events contained in the control region and signal region.
It is therefore chosen as the ring outside the control region.
The shift upwards in the center of $R_{hh}^{\text{high}}$ helps to contain more \ttbar~ events in the normalization estimates.
%Different Large-\R jet mass reflects very different underlying kinematics.
A larger SB will potentially increase the kinematic differences between different QCD processes in CR/SR and SB.
Different \mleadJ-\msublJ~ reflects of different fractions of $g \to b\bar{b}$ and $g \to c\bar{c}/q\bar{q}$ events.
This kinematic bias should be reduced as much as possible.
However, a smaller SB will introduce larger statistical uncertainties in the normalization estimation.
This is a bias-variance trade off.
The current design gives roughly $N_{SB} \sim 4 N_{SR}$ and $N_{SB} \sim 2 N_{CR}$, which provides a good balance between the bias and variance in the background estimation.

\begin{figure*}[htb!]
\begin{center}
  \includegraphics[width=0.6\textwidth,angle=-90]{figures/boosted/Other/TwoTag_split_Incl_data_mH0H1.pdf}
  \caption{The \mleadJ vs \msublJ distribution of data in $2bs$ region. The signal region is the area surrounded by the inner (red) dashed contour line, centered on (\mleadJ=$124$~\GeV, \msublJ=$115$~\GeV). The control region is the area between the signal region and the intermediate (orange) contour line. The sideband region is the area between the control region and the outer (yellow) contour line.}
  \label{fig:boosted-region-def}
\end{center}
\end{figure*}

\paragraph{}
The fraction between the expected events in different \mleadJ-\msublJ~ regions and the total number of events as a function of \Grav~ resonance mass is shown in Figure~\ref{fig:boosted-selection-region-efficiency}. 
For the $4b$, $3b$, and $2bs$ channels, there is no significant signal contamination in the control and sideband regions.


\begin{figure}[htb!]
  \centering
  \captionsetup{justification=centering}
    \begin{subfigure}[b]{0.4\textwidth}
        \includegraphics[width=\textwidth,angle=-90]{figures/boosted/SigEff/region_2b_lst_Moriond_Efficiency_PreSel.pdf}
        \caption{$2bs$ regions}
        \label{fig:2bs-selection-region-efficiency}
    \end{subfigure}
    \quad \quad 
    \begin{subfigure}[b]{0.4\textwidth}
        \includegraphics[width=\textwidth,angle=-90]{figures/boosted/SigEff/region_3b_lst_Moriond_Efficiency_PreSel.pdf}
        \caption{$3b$ regions}
        \label{fig:3b-selection-region-efficiency}
    \end{subfigure} \\ 
    \begin{subfigure}[b]{0.4\textwidth}
        \includegraphics[width=\textwidth,angle=-90]{figures/boosted/SigEff/region_4b_lst_Moriond_Efficiency_PreSel.pdf}
        \caption{$4b$ regions}
        \label{fig:4b-selection-region-efficiency}
    \end{subfigure}
    \quad \quad 
    \begin{subfigure}[b]{0.4\textwidth}
        \includegraphics[width=\textwidth,angle=-90]{figures/boosted/SigEff/region_alltag_lst_Moriond_Efficiency_PreSel.pdf}
        \caption{Inclusive regions}
        \label{fig:alltag-selection-region-efficiency}
    \end{subfigure} \\ 
   \caption{
   Detailed signal efficiency in different signal/control/sideband regions in $n$-tagged channels and the inclusive $b$-tagged channel, which includes $2b$, $1b$, and 0$b$ as well, (bottom right) as a function of \Grav~ resonance mass for selection cuts. The efficiencies are relative to the total number of events in the preselection.}
  \label{fig:boosted-selection-region-efficiency}
\end{figure}


%%%%%%%%%%%%%%%%%%%%%%%%%%%%%%%%%%%%%%%%%%%%%%%%%%%%%%%%%%%%%%%%%%%%%%%
%%%%%%%%%%%%%%%%%%%%%%%%%%%%%%%%%%%%%%%%%%%%%%%%%%%%%%%%%%%%%%%%%%%%%%%
\section{QCD multi-jet}
\label{sec:boosted-qcd}

\begin{table}[htb!]
\begin{center}
\caption{Definitions of the QCD background estimation sample for $4b/3b/2bs$. $n^{lead/sublJ}_{b}$ stands for the number of $b$-tagged track jets in the leading/subleading large-\R jet.}
\begin{tabular}{c|c}
\hline
  Region ($n^{leadJ}_{b} - n^{sublJ}_{b}$)  & QCD background sample ($n^{leadJ}_{b} - n^{sublJ}_{b}$) \\
  \hline
  $4b$ (2-2) & $2b$ data (2-0 or 0-2) \\
  $3b$ (2-1 or 1-2)& $2b$ data (2-0 or 0-2) \\
  $2bs$ (1-1) & $1b$ data (1-0 or 0-1) \\
  \end{tabular}
\label{tab:boosted-qcd-bkgsample}
\end{center}
\end{table}

\paragraph{}
The QCD multi-jet background prediction relies on finding $b$-tagged channels that have events similar to signal channel event properties.
These channels are defined to be identical to the signal channels except with a lower required number of $b$-tagged track jets associated with the large-\R jets.
They are orthogonal to the $n$-tagged channels with no overlapping events, as listed in Table~\ref{tab:boosted-qcd-bkgsample}.
They are assumed to be all background, and the signal yield in these channels are negligible, as shown in Figure~\ref{fig:boosted-nbjet-signal-efficiency-region}.
Detailed selection requirements and details are listed:
\begin{itemize}
  \item For the $2bs$ category, $1b$ data events, where one large-\R jet has one and only one $b$-tagged track jets, and the other large-\R jet has no $b$-tagged track jet, is used for modeling.
  \item For the $3b$ and $4b$ categories, $2b$ data events, where one large-\R jet has two $b$-tagged track jets, and the other large-\R jet has no $b$-tagged track jet, is used for modeling. The $2b$ sample is further split into $80\%-20\%$ parts, where each is used separately for $3b$ and $4b$ background estimations. This ensures the shape estimations of $3b$ and $4b$ QCD estimates are uncorrelated. 
\end{itemize}
The $2b$ and $2bs$ channels are orthogonal channels without any overlapping events.
The $1b$-tagged channel also requires that both large-\R jets are associated with at least one track jet. 
Similarly, the $2b$-tagged channel requires that one large-\R jet has at least one track jet and the other one has at least two track jets.  
Specifically for the $4b$ background modeling, each large-\R jet must have at least two track jets. 
This prevents biases in the \mtwoJ~ distribution from the number of associated track jets.

\paragraph{}
The MC predicted \ttbar~ and $Z$+jets events in the $1b/2b$ channels are subtracted from the data distributions to produce the QCD estimation. 
Therefore, the number of QCD events, $N_{\text{qcd}}$, is estimated in Equation~\ref{eq:nqcd}.
\begin{eqnarray}
\label{eq:nqcd}
N_{\text{qcd}} = N_{\text{data}} - N_{t\bar{t}} - N_{Z+jets} 
\end{eqnarray}

\paragraph{} 
The resolved veto imposed impacts the $4b$ background estimation. 
To account for this effect, the $20\%$ $2b$ data events used for $4b$ background estimation are excluded.
The excluded events must have at least two small-\R jets that are $b$-tagged (passing the resolved $70\%$ $b$-tagging working point), and in the case where the two other non $b$-tagged resolved jets can make a Higgs candidate, events must also pass the $X_{hh-resolved} < 1.6$ cut.
This ensures that a similar sculpting effect from the resolved veto is reflected in the background estimation in the $4b$ signal region.
This reduces the estimated number of events in the signal region by $10\%$.

\paragraph{}
Given the $1b/2b$ QCD background samples, the normalization estimation of the $4b/3b/2bs$ QCD background is determined in the sideband by fitting the \mleadJ~ distribution simultaneously with QCD and \ttbar~ background templates, as described in section~\ref{sec:ttbarnorm}.
There are kinematic differences between the $1b/2b$ channels and the $4b/3b/2bs$ channels.  
A kinematic reweighting is applied to correct for such differences, as described in Section~\ref{sec:boosted-reweight}.


%%%%%%%%%%%%%%%%%%%%%%%%%%%%%%%%%%%%%%%%%%%%%%%%%%%%%%%%%%%%%%%%%%%%%%%
%%%%%%%%%%%%%%%%%%%%%%%%%%%%%%%%%%%%%%%%%%%%%%%%%%%%%%%%%%%%%%%%%%%%%%%
\section{\ttbar~}
\label{sec:boosted-ttbar}

\paragraph{}
The \ttbar~ events in the signal region decay to $bW^{+}/\bar{b}W^{-}$ and the $W$ further decays into two quarks.
The leptonically decaying $W$s are rare due to the large-\R jet \pt~ cut.
The \ttbar~ MC is first scaled by the total luminosity times the cross-section.
The boosted event selections are then applied.
The estimated \ttbar~ kinematic distributions come from MC simulations.
To account for possible mis-modeling in the MC, a normalization scaling factor is derived from a fit to data in the sideband region, as described in section~\ref{sec:ttbarnorm}.

\paragraph{}
For $4b$ and $3b$ signal regions, there are not sufficient \ttbar~ MC statistics for high \mtwoJ.
Instead, the $2bs$ \ttbar~ MC shapes are used.
It is rescaled to the $4b$ and $3b$ \ttbar~ MC yields.
This reduces the statistical uncertainties for $4b$ and $3b$ \ttbar~ at high \mtwoJ.
A comparison between the $4b/3b/2bs$ shapes for the \mtwoJ distributions in the signal regions is shown in Figure~\ref{fig:ttshapeComp}.
The shapes are compatible, with the $4b$ having much larger statistical uncertainties.  
Differences between these distributions will be used as a systematic, as described in chapter~\ref{sec:systematics}.
Since the same \ttbar~ distribution is used for the $4b/3b/2bs$ SR predictions, the \ttbar~ shape systematics are considered correlated in the final results. 


\begin{figure}[htb!]
  \centering
  \captionsetup{justification=centering}
  \hspace{-0.5cm}
    \begin{subfigure}[b]{0.45\textwidth}
        \includegraphics[width=\textwidth,angle=-90]{figures/boosted/Other/ttbar_compare_mHH_l.pdf}
        \caption{Linear scale}
        \label{fig:ttshapeComp_lin}
    \end{subfigure}
    \quad
    \begin{subfigure}[b]{0.45\textwidth}
        \includegraphics[width=\textwidth,angle=-90]{figures/boosted/Other/ttbar_compare_mHH_l_1.pdf}
        \caption{Log scale}
        \label{fig:ttshapeComp_log}
    \end{subfigure}
   \caption{Normalized $2bs$, $3b$, and $4b$ \ttbar~ MC \mtwoJ~ (mHH) distribution in the signal region. The uncertainties are statistical.}
  \label{fig:ttshapeComp}
\end{figure}


%%%%%%%%%%%%%%%%%%%%%%%%%%%%%%%%%%%%%%%%%%%%%%%%%%%%%%%%%%%%%%%%%%%%%%%
%%%%%%%%%%%%%%%%%%%%%%%%%%%%%%%%%%%%%%%%%%%%%%%%%%%%%%%%%%%%%%%%%%%%%%%
%%%  Fitting
%%%%%%%%%%%%%%%%%%%%%%%%%%%%%%%%%%%%%%%%%%%%%%%%%%%%%%%%%%%%%%%%%%%%%%%
%%%%%%%%%%%%%%%%%%%%%%%%%%%%%%%%%%%%%%%%%%%%%%%%%%%%%%%%%%%%%%%%%%%%%%%
\section{Fitting procedure for QCD and \ttbar~ normalization}
\label{sec:ttbarnorm}

\paragraph{}
The number of $4b/3b/2bs$ background events in a given region R (SB / CR / SR) is shown in Equations~\ref{eq:fitparams-4b}, \ref{eq:fitparams-3b} and \ref{eq:fitparams-2bs}:
\begin{eqnarray}
\label{eq:fitparams-4b}
N^{4b\text{-R}}_{\text{bkg}} = \mu_{\text{qcd}}^{4b} N^{2b\text{-R}}_{\text{qcd}} + \alpha_{t\bar{t}}^{4b} N^{4b\text{-R}}_{t\bar{t}} + N^{4b\text{-R}}_{Z+jets};
\quad \mu_{\text{qcd}}^{4b} \sim \frac{N^{4b\text{-SB}}_{\text{qcd}}}{N^{2b\text{-SB}}_{\text{qcd}}}
\\
\label{eq:fitparams-3b}
N^{3b\text{-R}}_{\text{bkg}} = \mu_{\text{qcd}}^{3b\text{-R}} N^{2b\text{-R}}_{\text{qcd}} + \alpha_{t\bar{t}}^{3b} N^{3b\text{-R}}_{t\bar{t}} + N^{3b\text{-R}}_{Z+jets} ;
\quad \mu_{\text{qcd}}^{3b} \sim \frac{N^{3b\text{-SB}}_{\text{qcd}}}{N^{2b\text{-SB}}_{\text{qcd}}}\\
\label{eq:fitparams-2bs}
N^{2bs\text{-R}}_{\text{bkg}} = \mu_{\text{qcd}}^{2bs\text{-R}} N^{1b\text{-R}}_{\text{qcd}} + \alpha_{t\bar{t}}^{2bs} N^{2bs\text{-R}}_{t\bar{t}} + N^{2bs\text{-R}}_{Z+jets};
\quad \mu_{\text{qcd}}^{2bs} \sim \frac{N^{2bs\text{-SB}}_{\text{qcd}}}{N^{1b\text{-SB}}_{\text{qcd}}}
\end{eqnarray} 
\muqcd~ is essentially an extrapolation factor to account of the fewer number of events passing the higher number of $b$-tagged track jet selection.
\muqcd~ is roughly a constant over SB/CR/SR \mleadJ-\msublJ regions.
Validation of this assumption can be found in Appendix~\ref{AppendixMuqcd}.
\alphatt~ is a scale factor, correcting the \ttbar~ MC yield.
\muqcd~ and \alphatt~ are derived in the sideband region.
They are used as multiplicative constants in other regions of the mass plane (i.e. the control region or the signal region).

\paragraph{}
A binned maximum likelihood fit is used to find the values of \muqcd~ and \alphatt~, as well as the correlation between them.
These scaling parameters are determined independently using the same procedure in the $4b/3b/2bs$ sideband regions.
Due to the \pt~ $>450$ \GeV~ cut imposed on the leading large-\R jet, the top quark decay products are fully contained inside the leading large-\R jet in \ttbar~ events. 
The \mleadJ~ from \ttbar~ events has a clean peak around $m_{top} \sim 175$ \GeV~ in the sideband region.
It has the best separation between QCD and \ttbar~ shapes.
Therefore, the fit is performed on the \mleadJ spectrum in the sideband region.

\paragraph{}
Figure~\ref{fig:ttbar-fit} shows the post-fit spectrum of \mleadJ~ in the $n$-tagged sideband regions.
The \mleadJ~ shapes in data are well modeled by the predicted backgrounds.
The sharp turn-on at \mleadJ~ $\sim 75$ \GeV~ and sharp fall around $180$ \GeV~ are due to the sideband region mass cut on \mleadJ.
The dip in the \mleadJ~ distribution at $\sim 125$ \GeV~ is because the sideband region doesn't contain events from the control and signal regions.
The fitting errors on \muqcd~ and \alphatt~ are applied as systematic uncertainties taking into account their correlation.

\begin{figure}[htb!]
  \centering
  \captionsetup{justification=centering}
    \hspace{-4cm}
    \begin{subfigure}[b]{0.3\textwidth}
        \includegraphics[width=\textwidth,angle=-90]{figures/boosted/Fit/fitNorm_i4.pdf}
        \caption{$4b$ \mleadJ~ fit}
        \label{fig:ttbar-fit-4b}
    \end{subfigure}
    \quad \quad \quad \quad \quad
    \begin{subfigure}[b]{0.3\textwidth}
        \includegraphics[width=\textwidth,angle=-90]{figures/boosted/Fit/fitNorm_i3.pdf}
        \caption{$3b$ \mleadJ~ fit}
        \label{fig:ttbar-fit-3b}
    \end{subfigure}
    \\
    \hspace{-4cm}
    \begin{subfigure}[b]{0.3\textwidth}
        \includegraphics[width=\textwidth,angle=-90]{figures/boosted/Fit/fitNorm_i2s.pdf}
        \caption{$2bs$ \mleadJ~ fit}
        \label{fig:ttbar-fit-2bs}
    \end{subfigure}
   \caption{Simultaneous fit of \muqcd~ and \alphatt~ in $4b$, $3b$ and$2bs$ sideband regions using the leading large-\R jet mass.}
  \label{fig:ttbar-fit}
\end{figure}

\paragraph{}
The \ttbar~ MC sample normalization is further corrected from the fitted \alphatt.
At first, \alphatt~ is assumed to be $1$.
Once the first fit is done, the QCD background is re-estimated using the updated $2bs$ \alphatt~value, which has the smallest uncertainty.
Then, the fit is repeated until the change in \alphatt~ is less than $0.01$.
This iterative procedure helps correct the \alphatt~ bias in the data-driven QCD template.

\paragraph{}
The values of \muqcd~ and \alphatt~ estimated by the fits in the $4b/3b/2bs$ sideband regions can be found in Table~\ref{tab:bkgfit_prereweight}, along with the correlation $\rho(\mu_{qcd},\alpha_{t\bar{t}}) = \frac{Cov(\rm \mu_{qcd},\rm \alpha_{tt})}{\rm \sigma_{\mu_{qcd}} \rm \sigma_{\alpha_{tt}} }$. 
\muqcd~ and \alphatt~ are approximately $70\%$ negatively correlated, due to the nature of the two components fit and the fixed normalization.
This negative correlation leads to a smaller total normalization uncertainty.

\begin{table}[htb!]
\begin{center}
\caption{Background scaling parameters (\muqcd and \alphatt) estimated from fits to the \mleadJ distributions in $4b/3b/2bs$ sideband regions pre-reweighting. $\rho(\mu_{qcd},\alpha_{t\bar{t}}) = \frac{Cov(\rm \mu_{qcd},\rm \alpha_{\rm tt})}{\rm \sigma_{\mu_{qcd}} \rm \sigma_{\rm \alpha_{ tt}} }$.}
\begin{footnotesize} 
\begin{tabular}{c|c|c|c} 
Sample & $\mu_{qcd}$ & $\alpha_{t\bar{t}}$ & $\rho(\mu_{qcd}, \alpha_{t\bar{t}})$ \\ 
\hline\hline 
FourTag & 0.033987 $\pm$ 0.0043057 & 1.01697 $\pm$ 0.58642 & -0.76397\\
ThreeTag & 0.16247 $\pm$ 0.0041713 & 0.86508 $\pm$ 0.069019 & -0.6778\\
TwoTag split & 0.066713 $\pm$ 0.00091137 & 1.03747 $\pm$ 0.026199 & -0.74785\\
\hline\hline 
\end{tabular} 
\end{footnotesize} 
\newline 

\label{tab:bkgfit_prereweight}
\end{center}
\end{table}


%%%%%%%%%%%%%%%%%%%%%%%%%%%%%%%%%%%%%%%%%%%%%%%%%%%%%%%%%%%%%%%%%%%%%%%
%%%%%%%%%%%%%%%%%%%%%%%%%%%%%%%%%%%%%%%%%%%%%%%%%%%%%%%%%%%%%%%%%%%%%%%
%%%  Reweighting
%%%%%%%%%%%%%%%%%%%%%%%%%%%%%%%%%%%%%%%%%%%%%%%%%%%%%%%%%%%%%%%%%%%%%%%
%%%%%%%%%%%%%%%%%%%%%%%%%%%%%%%%%%%%%%%%%%%%%%%%%%%%%%%%%%%%%%%%%%%%%%%

\section{QCD reweighting}
\label{sec:boosted-reweight}

\paragraph{}
Reweighting gives the QCD background sample events weights different from one.
This results in a different distribution in all QCD kinematic distributions.
The weights are derived from kinematic differences between a Higgs candidate that has $b$-tags and a Higgs candidate that doesn't have $b$-tags.
The reweighting variables are the track jet $p_\text{T}$s and large-\R jet \pt.
More supporting plots are shown in Appendix~\ref{AppendixRW}.
%This is the most time consuming part of this analysis.

\paragraph{}
It is important to model the QCD background as well as possible in all regions of the phase space.
Using the $1/2b$ channels to model the $2bs/3b/4b$ channels can introduce discrepancies in the modeling of the estimated QCD background versus the data. 
These discrepancies arise from the non-trivial effect that $b$-tagging requirements can have in sculpting the track jet \pt~ and $\eta$ distributions.

\begin{figure}[htb!]
  \centering
  \captionsetup{justification=centering}
    \begin{subfigure}[b]{0.4\textwidth}
        \includegraphics[width=\textwidth,angle=-90]{figures/boosted/Prereweight/2bs_directcompare_leadHCand_trk0_Pt_1.pdf}
        \caption{leading large-\R jet's leading track jet \pt}
        \label{fig:rw-2bs-comp-lead0}
    \end{subfigure}
    \quad \quad 
    \begin{subfigure}[b]{0.4\textwidth}
        \includegraphics[width=\textwidth,angle=-90]{figures/boosted/Prereweight/2bs_directcompare_leadHCand_trk1_Pt_1.pdf}
        \caption{leading large-\R jet's subleading track jet \pt}
        \label{fig:rw-2bs-comp-lead1}
    \end{subfigure} \\ 
    \begin{subfigure}[b]{0.4\textwidth}
        \includegraphics[width=\textwidth,angle=-90]{figures/boosted/Prereweight/2bs_directcompare_sublHCand_trk0_Pt_1.pdf}
        \caption{subleading large-\R jet's leading track jet \pt}
        \label{fig:rw-2bs-comp-subl0}
    \end{subfigure}
    \quad \quad 
    \begin{subfigure}[b]{0.4\textwidth}
        \includegraphics[width=\textwidth,angle=-90]{figures/boosted/Prereweight/2bs_directcompare_sublHCand_trk1_Pt_1.pdf}
        \caption{subleading large-\R jet's subleading track jet \pt}
        \label{fig:rw-2bs-comp-subl1}
    \end{subfigure} \\ 
   \caption{
   Comparison of track jet \pt distributions in different $b$-tag channels. Data are shown in the plots. They contain inclusive SB/CR/SR regions for 0$b$ and $1b$, and only the SB region for $2bs$. At the bottom of the plots, all the ratios are taken with respect to the 0$b$ tagged distribution.}
  \label{fig:rw-2bs-comp}
\end{figure}


\begin{table}[htb!]
\begin{center}
\caption{Definitions of the $1b$ subsamples.}
\begin{tabular}{c|c|c|c|c}
\hline
  Channel & N$b^{leadJ}_{leadtrkj}$ & N$b^{leadJ}_{subltrkj}$ & N$b^{sublJ}_{leadtrkj}$ & N$b^{sublJ}_{subltrkj}$ \\
  \hline
  ``OneTag lead on lead'' & 1 & 0 & 0 & 0 \\
  ``OneTag lead on subl'' & 0 & 1 & 0 & 0 \\
  ``OneTag subl on lead'' & 0 & 0 & 1 & 0 \\
  ``OneTag subl on subl'' & 0 & 0 & 0 & 1 \\
  \end{tabular}
\label{tab:boosted-qcd-1bsample}
\end{center}
\end{table}

\paragraph{}
The reweighting method is motivated by Figure~\ref{fig:rw-2bs-comp}.
In the plots, 0$b$, $1b$ and $2bs$ kinematic distributions in data are compared.
Except $2bs$ data only contains the sideband region, 0$b$ and $1b$ are inclusive of sideband, control, and signal regions.
The $1b$ sample is further split into four subcategories listed in Table~\ref{tab:boosted-qcd-1bsample}, depending on which track jet gets $b$ tagged.
% \begin{itemize}
% \item ``OneTag lead on lead'' means the only $b$-tagged track jet is the leading track jet of the leading Higgs candidate;
% \item ``OneTag lead on subl'' means the only $b$-tagged track jet is the subleading track jet of the leading Higgs candidate;
% \item ``OneTag subl on lead'' means the only $b$-tagged track jet is the leading track jet of the subleading Higgs candidate; 
% \item ``OneTag subl on subl'' means the only $b$-tagged track jet is the subleading track jet of the subleading Higgs candidate;
% \end{itemize}
The figure shows that $2bs$ channel has very similar track jet \pt~ distributions to the $1b$ channel with a $b$-tagged track jet.
It also shows that in the $1b$ channel, the track jet \pt~ distribution in the non $b$-tagged Higgs candidate behaves like the 0$b$ channel track jet \pt~ distribution.

\paragraph{}
This reweighting technique is applied to the $1/2b$ data only. 
The reweighting procedure is the same for the $4b$, $3b$, and $2bs$ channels. 
The $2b$ QCD sample is already split into orthogonal parts for $3b$ and $4b$ estimations, as described in section \ref{sec:boosted-qcd}.

\paragraph{}
For $2bs$, the $1b$ non-$b$-tagged Higgs candidate is reweighted to be like a $1b$-tagged Higgs candidate; for $3b$, the $2b$ non-$b$-tagged Higgs candidate is reweighted to be like a $1b$-tagged Higgs candidate; for $4b$, the $2b$ non-$b$-tagged Higgs candidate is reweighted to be like a $2b$-tagged Higgs candidate.
For each category, the events are split into two orthogonal subgroups, based on whether leading/subleading Higgs candidate is $b$-tagged.
Each event gets a weight, such that the reweighted untagged Higgs candidate's distributions agree with the corresponding $b$-tagged Higgs candidate's.

\paragraph{}
One natural choice of a reweighting variable is the \pt~ of the track jets in the event, since the $b$-tagging efficiency and the $b$-tagging fake rate have a strong track jet \pt~ dependence. 
Also, large-\R jet \pt~ is reweighted to account for the difference from light/$c$/$b$ quarks composition in QCD events at different energy scales.
The three reweighted variables are the leading large-\R jet \pt, leading large-\R jet leading track jet \pt, and subleading large-\R jet leading track jet \pt~.

\paragraph{}
The each reweighting iteration is described below:
\begin{itemize}
  \item Subtract $1/2b$ \ttbar~ and $Z+$jets samples in the SB + CR + SR regions from the $1/2b$ tag data in the SB + CR + SR regions to get the $1/2b$ QCD inclusive estimate. Weights from all previous iterations, if applicable, are all applied.
  \item Separate the $1/2b$ sample into two parts: (A) has the $b$-tagged Higgs as the leading \pt~ Higgs candidate, and (B) has the $b$-tagged Higgs as the subleading \pt~ Higgs candidate.
  \item For each variable, i.e. the large-\R jet $p_{T}$: normalize sample A to sample B's total number of events, take the ratio of sample A's distribution over sample B's distribution, and fit the ratio with a smooth spline function (TSpline3). The binning of the distribution needs to be carefully chosen such that each bin has enough statistics to calculate a meaningful ratio.
  \item Use this spline function to extract reweighting values $w_{e}$ for each variable that is considered. Then, the difference from one is scaled by $0.75$ to get a new weight $w$. So $w = (w_{e} - 1) * 0.75$. This accounts for over correlation by the spline and accelerates convergence.
  \item For each event, all the three weights from three variables are multiplied together to get a data event weight. Another constraint is applied, such that each event's total reweighting value is constrained to be within a $0.05$ to $10$ range. This avoids over corrections.
\end{itemize}
Examples for the reweighting procedure are shown in Appendix~\ref{AppendixRW}.
The weights are almost converging to one after three iterations.
To ensure all kinematics regions are covered, a total of ten iterations are used to stabilize the reweighting. 
The reweighting value for each variable is also constrained to be within a $-30\%$ to $+40\%$ range compared to one, to avoid over corrections.

\begin{table}[htb!]
\begin{center}
\caption{Background scaling parameters (\muqcd~ and \alphatt~) estimated from fits to the \mleadJ~ distributions in $4b/3b/2bs$ sideband regions post reweighting. $\rho(\mu_{qcd},\alpha_{t\bar{t}}) = \frac{Cov(\rm \mu_{qcd},\rm \alpha_{\rm tt})}{\rm \sigma_{\mu_{qcd}} \rm \sigma_{\alpha_{\rm tt}} }$.}
\begin{footnotesize} 
\begin{tabular}{c|c|c|c} 
Sample & $\mu_{qcd}$ & $\alpha_{t\bar{t}}$ & $\rho(\mu_{qcd}, \alpha_{t\bar{t}})$ \\ 
\hline\hline 
FourTag & 0.033167 $\pm$ 0.0042799 & 0.89136 $\pm$ 0.59866 & -0.7846\\
ThreeTag & 0.16256 $\pm$ 0.0043405 & 0.79989 $\pm$ 0.073276 & -0.72029\\
TwoTag split & 0.062726 $\pm$ 0.00057307 & 0.98637 $\pm$ 0.018582 & -0.4698\\
\hline\hline 
\end{tabular} 
\end{footnotesize} 
\newline 

\label{tab:bkgfit}
\end{center}
\end{table}

\paragraph{}
At the end of reweighting, the \muqcd~ and \alphatt~ is re-evaluated. The estimated \muqcd~ and \alphatt~ values before reweighting can be found in Table~\ref{tab:bkgfit}. 
The values are statistically consistent with the values in Table~\ref{tab:bkgfit_prereweight}.


%For reweighting method comparisons and validations in data and Dijet MC, see Appendix~\ref{app:reweightstudy}.
%For the distribution of weights and the weight as a function of different kinematic ranges, see Appendix~\ref{app:reweight-dist}.

%%%%%%%%%%%%%%%%%%%%%%%%%%%%%%%%%%%%%%%%%%%%%%%%%%%%%%%%%%%%%%%%%%%%%%%
%%%%%%%%%%%%%%%%%%%%%%%%%%%%%%%%%%%%%%%%%%%%%%%%%%%%%%%%%%%%%%%%%%%%%%%
%%%  SB plots
%%%%%%%%%%%%%%%%%%%%%%%%%%%%%%%%%%%%%%%%%%%%%%%%%%%%%%%%%%%%%%%%%%%%%%%
%%%%%%%%%%%%%%%%%%%%%%%%%%%%%%%%%%%%%%%%%%%%%%%%%%%%%%%%%%%%%%%%%%%%%%%
\clearpage
\section{\mtwoJ~ in the sideband region}
\label{sec:boosted-sb}

\paragraph{}
Figure~\ref{fig:boosted-sb-mjj} shows comparisons of the predicted \mtwoJ~background distributions to those observed in data in the sideband regions.
The predicted background and observed distributions are in agreement, with no significant excess.
Other sideband region distributions are shown in Appendix~\ref{AppendixSB}.

\begin{figure}[htb!]
  \centering
  \captionsetup{justification=centering}
    \hspace{-2cm}
    \begin{subfigure}[b]{0.35\textwidth}
        \includegraphics[width=\textwidth,angle=-90]{figures/boosted/Paperplot/Moriond_bkg_9_FourTag_Sideband_mHH_l_1.pdf}
        \caption{$4b$}
        \label{fig:boosted-sb-mjj-4b}
    \end{subfigure}
    \quad \quad \quad \quad \quad
    \begin{subfigure}[b]{0.35\textwidth}
        \includegraphics[width=\textwidth,angle=-90]{figures/boosted/Paperplot/Moriond_bkg_9_ThreeTag_Sideband_mHH_l_1.pdf}
        \caption{$3b$}
        \label{fig:boosted-sb-mjj-3b}
    \end{subfigure}
    \\
    \begin{subfigure}[b]{0.35\textwidth}
        \includegraphics[width=\textwidth,angle=-90]{figures/boosted/Paperplot/Moriond_bkg_9_TwoTag_split_Sideband_mHH_l_1.pdf}
        \caption{$2bs$}
        \label{fig:boosted-sb-mjj-2bs}
    \end{subfigure}
   \caption{The \mtwoJ~ distributions in the sideband region of the boosted analysis for the data and the predicted background for $4b$, $3b$, and $2bs$ channels. The data-to-background ratio (bottom panels) shows the statistical uncertainties as the gray hatched band.}
  \label{fig:boosted-sb-mjj}
\end{figure}


%%%%%%%%%%%%%%%%%%%%%%%%%%%%%%%%%%%%%%%%%%%%%%%%%%%%%%%%%%%%%%%%%%%%%%%
%%%%%%%%%%%%%%%%%%%%%%%%%%%%%%%%%%%%%%%%%%%%%%%%%%%%%%%%%%%%%%%%%%%%%%%
%%%  CR plots
%%%%%%%%%%%%%%%%%%%%%%%%%%%%%%%%%%%%%%%%%%%%%%%%%%%%%%%%%%%%%%%%%%%%%%%
%%%%%%%%%%%%%%%%%%%%%%%%%%%%%%%%%%%%%%%%%%%%%%%%%%%%%%%%%%%%%%%%%%%%%%%
\clearpage
\section{\mtwoJ~ in the control region}
\label{sec:boosted-cr}

\paragraph{}
Figure~\ref{fig:boosted-cr-mjj} shows comparisons of the predicted \mtwoJ~background distributions to those observed in data in the control regions.
The predicted background and observed distributions are in good agreement.
Other control region distributions are shown in Appendix~\ref{AppendixCR}.

\begin{figure}[htb!]
  \centering
  \captionsetup{justification=centering}
    \hspace{-2cm}
    \begin{subfigure}[b]{0.35\textwidth}
        \includegraphics[width=\textwidth,angle=-90]{figures/boosted/Paperplot/Moriond_bkg_9_FourTag_Control_mHH_l_1.pdf}
        \caption{$4b$}
        \label{fig:boosted-cr-mjj-4b}
    \end{subfigure}
    \quad \quad \quad \quad \quad
    \begin{subfigure}[b]{0.35\textwidth}
        \includegraphics[width=\textwidth,angle=-90]{figures/boosted/Paperplot/Moriond_bkg_9_ThreeTag_Control_mHH_l_1.pdf}
        \caption{$3b$}
        \label{fig:boosted-cr-mjj-3b}
    \end{subfigure}
    \\
    \begin{subfigure}[b]{0.35\textwidth}
        \includegraphics[width=\textwidth,angle=-90]{figures/boosted/Paperplot/Moriond_bkg_9_TwoTag_split_Control_mHH_l_1.pdf}
        \caption{$2bs$}
        \label{fig:boosted-cr-mjj-2bs}
    \end{subfigure}
   \caption{The \mtwoJ~ distributions in the control region of the boosted analysis for the data and the predicted background for $4b$, $3b$, and $2bs$ channels. The data-to-background ratio (bottom panels) shows the statistical uncertainties as the gray hatched band.}
  \label{fig:boosted-cr-mjj}
\end{figure}


%%%%%%%%%%%%%%%%%%%%%%%%%%%%%%%%%%%%%%%%%%%%%%%%%%%%%%%%%%%%%%%%%%%%%%%
%%%%%%%%%%%%%%%%%%%%%%%%%%%%%%%%%%%%%%%%%%%%%%%%%%%%%%%%%%%%%%%%%%%%%%%
%%%  SR smoothing
%%%%%%%%%%%%%%%%%%%%%%%%%%%%%%%%%%%%%%%%%%%%%%%%%%%%%%%%%%%%%%%%%%%%%%%
%%%%%%%%%%%%%%%%%%%%%%%%%%%%%%%%%%%%%%%%%%%%%%%%%%%%%%%%%%%%%%%%%%%%%%%
\section{\mtwoJ~ rescale and smoothing in the signal region}
\label{sec:boosted-SR-smoothing}

\paragraph{}
The \mtwoJ~ resolution is further improved by rescaling.
In the signal region, the four-momentum of each large-\R jet is multiplied by a factor $m_{H}/m_{\mathrm{J}}$.
From this, a scaled \mtwoJ~ is calculated.
This correction improves the jet mass and momentum resolution of each large-\R jet for signal events.
There is little impact on the background distribution.
Using the scaled \mtwoJ~ distribution, the expected exclusion limits are slightly better at low mass and slightly worse at high mass, with differences on the order of $10\%$ from the nominal \mtwoJ~ limits.
The scaled \mtwoJ~ distribution can be found in Figure~\ref{fig:signal-region-bkg-scaled}.

\begin{figure}[htb!]
\begin{center}
\includegraphics[width=0.3\textwidth,angle=-90]{figures/boosted/Other/FourTag_Signal_compare_scale_mHH_1.pdf}
\includegraphics[width=0.3\textwidth,angle=-90]{figures/boosted/Other/ThreeTag_Signal_compare_scale_mHH_1.pdf}
\includegraphics[width=0.3\textwidth,angle=-90]{figures/boosted/Other/TwoTag_split_Signal_compare_scale_mHH_1.pdf}
\caption{Normalized scaled \mtwoJ~ distributions for the $4b$ (left), $3b$ (middle), and $2bs$ (right) signal regions. For comparison, the unscaled  \mtwoJ~ distributions are shown on the same plot. }
\label{fig:signal-region-bkg-scaled}
\end{center}
\end{figure}

\paragraph{} 
Due to limited $1/2b$ statistics at high \mtwoJ~ above $2500$ \GeV~ and the limited \ttbar~statistics above $1100$ \GeV, different fits are performed to smooth the \mtwoJ~ mass distributions in the signal regions. 
The $4/3/2bs$ QCD and \ttbar~ signal region scaled \mtwoJ~ distributions are fitted with the following functional form:
\begin{equation}
\label{eq:boosted_dijet}
y = \frac{a}{(\frac{m_{2J}}{\sqrt{s}})^2} (1-\frac{m_{2J}}{\sqrt{s}})^{(b - c\ \log(\frac{m_{2J}}{\sqrt{s}}))}
\end{equation}
where $\sqrt{s} = 13000$ \GeV~, the fit range is $1200 <$ \mtwoJ~ $< 3000$ \GeV~, and the three free parameters are $a$, $b$ and $c$.
The values of the estimated fit parameters in the $4b$, $3b$, and $2bs$ signal regions can be found in Table~\ref{tab:smoothparams_pole}.
Figure~\ref{fig:signal-region-mjjscaled-smoothing-4b}, ~\ref{fig:signal-region-mjjscaled-smoothing-3b}, and ~\ref{fig:signal-region-mjjscaled-smoothing-2b} show the smoothing fits for the QCD background and the \ttbar\ background in the $4b$, $3b$, and $2bs$ signal regions.
The smoothing fit statistical uncertainties are also shown on these two plots. 
Given that $1/2b$ channels are used for deriving the QCD shape for $4/3/2bs$ signal regions, the slope parameters ($a$) are similar for $4/3/2bs$ QCD backgrounds.


\begin{figure}[htb!]
  \centering
  \captionsetup{justification=centering}
    \hspace{-2.5cm}
    \begin{subfigure}[b]{0.35\textwidth}
        \includegraphics[width=\textwidth,angle=-90]{figures/boosted/Smooth/qcd_est_FourTag_Signal_mHH_pole_l.pdf}
        \caption{QCD}
        \label{fig:signal-region-mjjscaled-smoothing-4b-qcd}
    \end{subfigure}
    \quad \quad \quad \quad \quad
    \begin{subfigure}[b]{0.35\textwidth}
        \includegraphics[width=\textwidth,angle=-90]{figures/boosted/Smooth/ttbar_est_FourTag_Signal_mHH_pole_l.pdf}
        \caption{\ttbar}
        \label{fig:signal-region-mjjscaled-smoothing-4b-ttbar}
    \end{subfigure}
    \caption{Fits for scaled background smoothing are shown for the $4b$ signal region. The figures show the distributions with the fit central value and the fit variations determined by varying the fit parameters within uncertainties and taking into account parameter correlations.}
  \label{fig:signal-region-mjjscaled-smoothing-4b}
\end{figure}

\begin{figure}[htb!]
  \centering
  \captionsetup{justification=centering}
    \hspace{-2.5cm}
    \begin{subfigure}[b]{0.35\textwidth}
        \includegraphics[width=\textwidth,angle=-90]{figures/boosted/Smooth/qcd_est_ThreeTag_Signal_mHH_pole_l.pdf}
        \caption{QCD}
        \label{fig:signal-region-mjjscaled-smoothing-3b-qcd}
    \end{subfigure}
    \quad \quad \quad \quad \quad
    \begin{subfigure}[b]{0.35\textwidth}
        \includegraphics[width=\textwidth,angle=-90]{figures/boosted/Smooth/ttbar_est_ThreeTag_Signal_mHH_pole_l.pdf}
        \caption{\ttbar}
        \label{fig:signal-region-mjjscaled-smoothing-3b-ttbar}
    \end{subfigure}
    \caption{Fits for scaled background smoothing are shown for the $3b$ signal region. The figures show the distributions with the fit central value and the fit variations determined by varying the fit parameters within uncertainties and taking into account parameter correlations.}
  \label{fig:signal-region-mjjscaled-smoothing-3b}
\end{figure}

\begin{figure}[htb!]
  \centering
  \captionsetup{justification=centering}
    \hspace{-2.5cm}
    \begin{subfigure}[b]{0.35\textwidth}
        \includegraphics[width=\textwidth,angle=-90]{figures/boosted/Smooth/qcd_est_TwoTag_split_Signal_mHH_pole_l.pdf}
        \caption{QCD}
        \label{fig:signal-region-mjjscaled-smoothing-2b-qcd}
    \end{subfigure}
    \quad \quad \quad \quad \quad
    \begin{subfigure}[b]{0.35\textwidth}
        \includegraphics[width=\textwidth,angle=-90]{figures/boosted/Smooth/ttbar_est_TwoTag_split_Signal_mHH_pole_l.pdf}
        \caption{\ttbar}
        \label{fig:signal-region-mjjscaled-smoothing-2b-ttbar}
    \end{subfigure}
    \caption{Fits for scaled background smoothing are shown for the $2bs$ signal region. The figures show the distributions with the fit central value and the fit variations determined by varying the fit parameters within uncertainties and taking into account parameter correlations.}
  \label{fig:signal-region-mjjscaled-smoothing-2b}
\end{figure}

\paragraph{}
The final signal region predictions using scaled \mtwoJ distributions, with only statistical uncertainties, are shown in Figures~\ref{fig:signal-region-mjjscaled-smooth-bkg-nosys-4b}, ~\ref{fig:signal-region-mjjscaled-smooth-bkg-nosys-3b}, and ~\ref{fig:signal-region-mjjscaled-smooth-bkg-nosys-2b}. 
This includes smoothing statistical uncertainties only. 
Uncertainties on the fit parameters are propagated as systematic uncertainties, though they are essentially replacing the bin-by-bin statistical uncertainties of the background estimates (which are not used once smoothing is applied).
More details on other systematics, including smoothing systematics, shape uncertainties, and other sources of uncertainties are discussed in section~\ref{unc-smooth-qcd-in-sr}.

\begin{table}[htb!]
\begin{center}
\caption{Smoothing parameters in $4b$, $3b$, and $2bs$ signal regions for scaled mass distributions, the correlation between parameters is almost always $0.99$.}
\begin{footnotesize} 
\begin{tabular}{c|c|c|c|c|c|c} 
Region & $ a_{t\bar{t}}$ & $ b_{t\bar{t}}$ & $ c_{t\bar{t}}$ & $ a_{qcd}$ & $ b_{qcd}$ & $c_{qcd}$ \\ 
\hline\hline 
& & & & & &\\ 
FourTag & -2.02 $\pm$ 1.17 & 42.46 $\pm$ 9.87 & 1.31 $\pm$ 8.98 & -0.49 $\pm$ 1.59 & 53.06 $\pm$ 15.2 & -11.1 $\pm$ 12.8\\ 
ThreeTag & 1.84 $\pm$ 1.17 & 42.45 $\pm$ 9.88 & 1.32 $\pm$ 8.98 & 8.51 $\pm$ 0.98 & -13.8 $\pm$ 9.14 & 42.58 $\pm$ 7.87\\ 
TwoTag split & 4.22 $\pm$ 1.17 & 42.45 $\pm$ 9.88 & 1.32 $\pm$ 8.98 & 7.06 $\pm$ 0.32 & 11.54 $\pm$ 2.77 & 19.05 $\pm$ 2.48\\ 
& & & & & &\\ 
\hline\hline 
\end{tabular} 
\end{footnotesize} 
\newline 

\label{tab:smoothparams_pole}
\end{center}
\end{table}

\begin{figure}[htb!]
  \centering
  \captionsetup{justification=centering}
    \begin{subfigure}[b]{0.41\textwidth}
        \includegraphics[width=\textwidth,angle=-90]{figures/boosted/Signal/b77_FourTag_Signal_mHH_pole_1_blind.pdf}
        \caption{before smoothing}
        \label{fig:signal-region-mjjscaled-smoothing-4b-qcd}
    \end{subfigure}
    \quad \quad \quad 
    \begin{subfigure}[b]{0.41\textwidth}
        \includegraphics[width=\textwidth,angle=-90]{figures/boosted/Smooth/Moriond_bkg_9_FourTag_pole_Signal_mHH_pole_1_blind.pdf}
        \caption{after smoothing}
        \label{fig:signal-region-mjjscaled-smoothing-4b-ttbar}
    \end{subfigure}
    \caption{Background predictions for $4b$ signal region using scaled \mtwoJ~ before and after smoothing. The uncertainty band includes only fit statistical uncertainties.}
  \label{fig:signal-region-mjjscaled-smooth-bkg-nosys-4b}
\end{figure}


\begin{figure}[htb!]
  \centering
  \captionsetup{justification=centering}
    \begin{subfigure}[b]{0.41\textwidth}
        \includegraphics[width=\textwidth,angle=-90]{figures/boosted/Signal/b77_ThreeTag_Signal_mHH_pole_1_blind.pdf}
        \caption{before smoothing}
        \label{fig:signal-region-mjjscaled-smoothing-3b-qcd}
    \end{subfigure}
    \quad \quad \quad 
    \begin{subfigure}[b]{0.41\textwidth}
        \includegraphics[width=\textwidth,angle=-90]{figures/boosted/Smooth/Moriond_bkg_9_ThreeTag_pole_Signal_mHH_pole_1_blind.pdf}
        \caption{after smoothing}
        \label{fig:signal-region-mjjscaled-smoothing-3b-ttbar}
    \end{subfigure}
    \caption{Background predictions for $3b$ signal region using scaled \mtwoJ~ before and after smoothing. The uncertainty band includes only fit statistical uncertainties.}
  \label{fig:signal-region-mjjscaled-smooth-bkg-nosys-3b}
\end{figure}

\begin{figure}[htb!]
  \centering
  \captionsetup{justification=centering}
    \begin{subfigure}[b]{0.41\textwidth}
        \includegraphics[width=\textwidth,angle=-90]{figures/boosted/Signal/b77_TwoTag_split_Signal_mHH_pole_1_blind.pdf}
        \caption{before smoothing}
        \label{fig:signal-region-mjjscaled-smoothing-2b-qcd}
    \end{subfigure}
    \quad \quad \quad 
    \begin{subfigure}[b]{0.41\textwidth}
        \includegraphics[width=\textwidth,angle=-90]{figures/boosted/Smooth/Moriond_bkg_9_TwoTag_split_pole_Signal_mHH_pole_1_blind.pdf}
        \caption{after smoothing}
        \label{fig:signal-region-mjjscaled-smoothing-2b-ttbar}
    \end{subfigure}
    \caption{Background predictions for $2bs$ signal region using scaled \mtwoJ~ before and after smoothing. The uncertainty band includes only fit statistical uncertainties.}
  \label{fig:signal-region-mjjscaled-smooth-bkg-nosys-2b}
\end{figure}

%%%%%%%%%%%%%%%%%%%%%%%%%%%%%%%%%%%%%%%%%%%%%%%%%%%%%%%%%%%%%%%%%%%%%%%
%%%%%%%%%%%%%%%%%%%%%%%%%%%%%%%%%%%%%%%%%%%%%%%%%%%%%%%%%%%%%%%%%%%%%%%
%%%  Yields
%%%%%%%%%%%%%%%%%%%%%%%%%%%%%%%%%%%%%%%%%%%%%%%%%%%%%%%%%%%%%%%%%%%%%%%
%%%%%%%%%%%%%%%%%%%%%%%%%%%%%%%%%%%%%%%%%%%%%%%%%%%%%%%%%%%%%%%%%%%%%%%
\section{Yields}
\label{sec:yields}
\paragraph{}
The event yield results showing the estimated backgrounds, the signal predictions, and the observed data in the $4b$, $3b$, and $2bs$ signal regions, control regions, and sideband regions can be found in Table~\ref{tab:yields4b}, ~\ref{tab:yields3b}, and ~\ref{tab:yields2b}, respectively. 
The total background statistical uncertainty is less than the quadratic sum of \ttbar~ and QCD because of their anti-correlation.
Z+jets has a negligible yield in the signal regions; therefore, in the later chapters, it is not included in the figures and tables.

\begin{table}[htb!]
\footnotesize
\begin{center}
\caption{Expected yields for backgrounds in the $4b$ signal region, control region, and sideband region, along with the observed number of data events.  The signal predictions for \Grav $m=1.0, 1.5, 2.0$~\TeV\ with $c=1.0$. The uncertainty listed is statistical, without fit uncertainty.}
\begin{footnotesize} 
\begin{tabular}{c|c|c|c} 
FourTag & Sideband & Control & Signal \\ 
\hline\hline 
& & & \\ 
QCD Est & 176.26 $\pm$ 2.96 & 64.21 $\pm$ 1.79 & 32.91 $\pm$ 1.25\\ 
$t\bar{t}$ Est.  & 27.86 $\pm$ 0.25 & 6.38 $\pm$ 0.13 & 1.68 $\pm$ 0.044\\ 
$Z+jets$ & 0 $\pm$ 0 & 6.18 $\pm$ 5.12 & 0 $\pm$ 0\\ 
Total Bkg Est & 204.12 $\pm$ 2.97 & 76.77 $\pm$ 5.43 & 34.59 $\pm$ 1.25\\ 
Data & 204.0 $\pm$ 14.28 & 81.0 $\pm$ 9.0 & 0 $\pm$ 0\\ 
$c=1.0$,$m=1.0TeV$ & 2.52 $\pm$ 0.1 & 5.4 $\pm$ 0.15 & 10.07 $\pm$ 0.2\\ 
$c=1.0$,$m=2.0TeV$ & 0.034 $\pm$ 0.0015 & 0.1 $\pm$ 0.0026 & 0.25 $\pm$ 0.0041\\ 
$c=1.0$,$m=3.0TeV$ & 0.00032 $\pm$ 3.7e-05 & 0.0008 $\pm$ 5.6e-05 & 0.0016 $\pm$ 8e-05\\ 
& & & \\ 
\hline\hline 
\end{tabular} 
\end{footnotesize} 
\newline 

\label{tab:yields4b}
\end{center}
\end{table}


\begin{table}[htb!]
\footnotesize
\begin{center}
\caption{Expected yields for backgrounds in the $3b$ signal region, control region, and sideband region, along with the observed number of data events.  The signal predictions for \Grav $m=1.0, 1.5, 2.0$~\TeV\ with $c=1.0$. The uncertainty listed is statistical, without fit uncertainty.}
\begin{footnotesize} 
\begin{tabular}{c|c|c|c} 
ThreeTag & Sideband & Control & Signal \\ 
\hline\hline 
& & & \\ 
QCD Est & 3518.01 $\pm$ 27.48 & 1413.52 $\pm$ 17.36 & 701.6 $\pm$ 11.95\\ 
$t\bar{t}$ Est.  & 852.88 $\pm$ 25.72 & 162.31 $\pm$ 11.15 & 79.34 $\pm$ 2.05\\ 
$Z+jets$ & 32.8 $\pm$ 11.34 & 11.21 $\pm$ 5.65 & 0.49 $\pm$ 0.49\\ 
Total Bkg Est & 4403.69 $\pm$ 39.31 & 1587.04 $\pm$ 21.4 & 781.42 $\pm$ 12.14\\ 
Data & 4403.0 $\pm$ 66.36 & 1553.0 $\pm$ 39.41 & 0 $\pm$ 0\\ 
$c=1.0$,$m=1.0TeV$ & 7.86 $\pm$ 0.18 & 12.58 $\pm$ 0.23 & 26.0 $\pm$ 0.33\\ 
$c=1.0$,$m=2.0TeV$ & 0.16 $\pm$ 0.0035 & 0.38 $\pm$ 0.0054 & 0.76 $\pm$ 0.0076\\ 
$c=1.0$,$m=3.0TeV$ & 0.0036 $\pm$ 0.00013 & 0.0075 $\pm$ 0.00018 & 0.013 $\pm$ 0.00023\\ 
& & & \\ 
\hline\hline 
\end{tabular} 
\end{footnotesize} 
\newline 

\label{tab:yields3b}
\end{center}
\end{table}

\begin{table}[htb!]
\footnotesize
\begin{center}
\caption{Expected yields for backgrounds in the $2bs$ signal region, control region, and sideband region, along with the observed number of data events.  The signal predictions for \Grav $m=1.0, 1.5, 2.0$~\TeV\ with $c=1.0$. The uncertainty listed is statistical, without fit uncertainty.}
\begin{footnotesize} 
\begin{tabular}{c|c|c|c} 
TwoTag split & Sideband & Control & Signal \\ 
\hline\hline 
& & & \\ 
QCD Est & 17216.91 $\pm$ 38.33 & 6821.96 $\pm$ 23.49 & 3393.56 $\pm$ 16.64\\ 
$t\bar{t}$ Est.  & 7852.35 $\pm$ 70.3 & 1484.57 $\pm$ 29.24 & 858.27 $\pm$ 22.23\\ 
$Z+jets$ & 67.74 $\pm$ 16.82 & 26.44 $\pm$ 10.08 & 0.13 $\pm$ 0.091\\ 
Total Bkg Est & 25137.01 $\pm$ 81.82 & 8332.97 $\pm$ 38.84 & 4251.96 $\pm$ 27.77\\ 
Data & 25137.0 $\pm$ 158.55 & 8486.0 $\pm$ 92.12 & 4376.0 $\pm$ 66.15\\ 
$c=1.0$,$m=1.0TeV$ & 4.79 $\pm$ 0.14 & 6.33 $\pm$ 0.16 & 10.87 $\pm$ 0.22\\ 
$c=1.0$,$m=2.0TeV$ & 0.18 $\pm$ 0.0039 & 0.36 $\pm$ 0.0056 & 0.6 $\pm$ 0.0072\\ 
$c=1.0$,$m=3.0TeV$ & 0.013 $\pm$ 0.00025 & 0.027 $\pm$ 0.00034 & 0.039 $\pm$ 0.00041\\ 
& & & \\ 
\hline\hline 
\end{tabular} 
\end{footnotesize} 
\newline 

\label{tab:yields2b}
\end{center}
\end{table}
