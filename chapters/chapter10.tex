%!TEX root = ../dissertation.tex
\begin{savequote}[75mm]
Tell the world.
\qauthor{Marathon}
\end{savequote}

\chapter{Intepretation}

\paragraph{}
In order to avoid the \href{https://en.wikipedia.org/wiki/Oops-Leon}{Oops-Leon} cases, commonly accepted standard for announcing the discovery of a particle is that the number of observed events is 5 standard deviations ($\sigma$) above the expected level of the background.

\paragraph{}
With no excess observed, a limit needs to be set on the cross section of the signal ~\cite{Stat-asym}. 

\paragraph{}
One method to parametries signal is called \href{https://arxiv.org/abs/1410.7388}{singal morphing}. Specifically, in momentum morphing, not only the signal strength is scaled, but also the width is modified by a non-linear transformation. 

\paragraph{}
An estimator should satisfy three criterias:
\begin{itemize}
	\item Consistency: the value of the estimator should converge to the truth value if the sample size goes to infinity
	\item Efficiency: theory limits the variance of the true value, given a sample size N (Minimum Variance bound, or MVB). If the variance of the estimator is equal to the MVB the estimator is called efficient.
	\item Unbiased: the estiator should have no difference from the true value, otherwise it is biased.
\end{itemize}
Maximum likelihood estimator is a good estimator, based on these criterias.

\paragraph{}
For measurements, most of the time \href{unfolding}{https://arxiv.org/abs/1611.01927} is done to compare with generator level distributions. This accounts for detector effects, statistical fluctuations and background mis-identification. Since the analysis is a search, this \hh$\to$\fourb is a search, unfolding is less applicable in this case.