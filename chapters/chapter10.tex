%!TEX root = ../dissertation.tex
\begin{savequote}[75mm]
Nulla facilisi. In vel sem. Morbi id urna in diam dignissim feugiat. Proin molestie tortor eu velit. Aliquam erat volutpat. Nullam ultrices, diam tempus vulputate egestas, eros pede varius leo.
\qauthor{Quoteauthor Lastname}
\end{savequote}

\chapter{Intepretation}

\paragraph{}
In order to avoid the \href{https://en.wikipedia.org/wiki/Oops-Leon}{Oops-Leon} cases, commonly accepted standard for announcing the discovery of a particle is that the number of observed events is 5 standard deviations ($\sigma$) above the expected level of the background.

\paragraph{}
With no excess observed, a limit needs to be set on the cross section of the signal ~\cite{Stat-asym}. 

\paragraph{}
An estimator should satisfy three criterias:
\begin{itemize}
	\item Consistency: the value of the estimator should converge to the truth value if the sample size goes to infinity
	\item Efficiency: theory limits the variance of the true value, given a sample size N (Minimum Variance bound, or MVB). If the variance of the estimator is equal to the MVB the estimator is called efficient.
	\item Unbiased: the estiator should have no difference from the true value, otherwise it is biased.
\end{itemize}
Maximum likelihood estimator is a good estimator, based on these criterias.