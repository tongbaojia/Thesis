%!TEX root = ../dissertation.tex
\begin{savequote}[75mm]
It was the best of times, it was the worst of times, it was the age of wisdom, it was the age of foolishness, it was the epoch of belief, it was the epoch of incredulity, it was the season of Light, it was the season of Darkness, it was the spring of hope, it was the winter of despair, we had everything before us, we had nothing before us, we were all going direct to Heaven, we were all going direct the other way. In short, the period was so far like the present period, that some of its noisiest authorities insisted on its being received, for good or for evil, in the superlative degree of comparison only.
\qauthor{Charles Dickens}
\end{savequote}

\chapter{Introduction}
\label{introduction}

\paragraph{}
In 2012, the Higgs boson was discovered by the ATLAS and CMS experiments at the LHC. 
The particle physics community faced a period just like at the beginning of \textit{A Tale of Two Cities}.
The Higgs discovery completes the Standard Model and leaves very few experimental clues for physics beyond the Standard Model at the LHC.
But with the LHC's increase in collision energy, it is a great time to search for beyond the Standard Model physics.
The newly discovered Higgs boson can be used as a tool in these searches.
Higgs boson pair production is particularly interesting.
While having a low cross-section in the Standard Model, new physics can modify its production in multiple ways. 
In particular, two Higgs bosons can be produced through heavy particle resonances
which give clear signatures at collider experiments.

\paragraph{}
This search focuses on the dominant \hbb~ decay mode to search for two Higgs bosons production. 
The ``resolved'' analysis is used for \hh~ systems in which the Higgs bosons have Lorentz boosts low enough that four $b$-jets can be reconstructed. 
The ``boosted'' analysis is used for those \hh~ systems in which the Higgs bosons have higher Lorentz boosts, which prevents the Higgs boson decay products from being resolved in the detector as separate \bjets. 
Instead, each Higgs boson candidate consists of a single large-radius jet, and $b$-decays are identified using smaller-radius jets built from charged-particle tracks.
The two analyses are complementary in their acceptance, each employing a unique technique to reconstruct the Higgs boson.
This thesis focuses on the boosted analysis due to its greater sensitivity to heavy resonance signals.

\paragraph{}
The dataset for the boosted analysis corresponds to $36.1$ $\mathrm{fb}^{-1}$ data collected in 2015 and 2016. 
The results are obtained using the resolved analysis for a resonance mass between $260$~\GeV\ and $1400$~\GeV, and the boosted analysis between $800$~\GeV\ and $3000$~\GeV. 
The main background is multijet production, which is estimated from data; the sub-leading background is \ttbar, which is estimated using both data and simulation. 
The two analyses employ orthogonal selections, and a statistical combination is performed in the mass range where they overlap.
The final discriminants are the four-jet and two-jet invariant mass distributions in the resolved and boosted analyses, respectively. 
Limits are set for the following benchmark signals: a spin-2 graviton decaying into Higgs bosons, a scalar resonance decaying into two Higgs bosons, and SM non-resonant Higgs boson pair production.
In the boosted analysis, the improvements with respect to the preceding ATLAS analysis comes from an additional signal region and novel background estimation techniques. 
%~\cite{EXOT-2015-11}. 

\paragraph{}
This thesis begins by discussing the status of di-Higgs searches. 
Chapter 1 gives an overview of double Higgs production in the Standard Model and beyond. 
Chapter 2 presents details regarding the Large Hadron Collider and the ATLAS experiment. 
Chapter 3 discusses reconstruction of physics objects. 
Chapter 4 lists the dataset and simulation samples. 
Chapter 5 shows the event selection.
Chapter 6 discusses the background estimation in details.
Chapter 7 presents the systematic uncertainties.
Finally, the results are shown in Chapter 8.
A brief summary and outlook is presented in the conclusion.
Many detailed plots and supporting material are shown in the appendices.


% There are two types of analysis in particle physics. The first one is measurement, which yields a observable with an uncertainty. 
% This could either improve our current knowledge, or show some inconsistency. 
% The other type is search, which generally assumes some new physics model and try to justify in data wether the new model is justified in some observables. 
% A successful search turns the subject into a measurement, yet a null search result will set a new limit for a given physics model.
