%!TEX root = ../dissertation.tex
\begin{savequote}[75mm]
It was the best of times, it was the worst of times, it was the age of wisdom, it was the age of foolishness, it was the epoch of belief, it was the epoch of incredulity, it was the season of Light, it was the season of Darkness, it was the spring of hope, it was the winter of despair, we had everything before us, we had nothing before us, we were all going direct to Heaven, we were all going direct the other way—in short, the period was so far like the present period, that some of its noisiest authorities insisted on its being received, for good or for evil, in the superlative degree of comparison only.
\qauthor{Charles Dickens}
\end{savequote}

\chapter{Introduction}
\label{introduction}

\paragraph{}
In 2012, the Higgs boson was discovered by the ATLAS and CMS experiment at the LHC. The particle physics community faces a period just like at the beginning of \textit{A Tale of Two Cities}. 


After Run 1 of the LHC, with the existence of the Higgs now firmly established, the focus shifted to searches for physics beyond the Standard Model.


In particular, searches for high mass resonances benefit from the LHC's increase to $\sqrt{s} = 13 \TeV$ in Run 2. 
The cross section for a generic gluon-initiated resonance with a mass of $2 \TeV$ increases tenfold in Run 2, making searches for high mass resonances a high priority. 
The newly discovered Higgs can be used as a tool in these searches. 
After the discovery, the Higgs boson provides a large swath of unmeasured phase space where new physics could be discovered. 
Higgs pair production in the Standard Model has a low cross section that requires large datasets (on the order of the LHC's lifetime) for full measurement. 
However, new physics can modify this cross section, especially through new resonances which decay to two Higgs bosons. 
Such high mass resonances also produce difficult to recognize final state topologies due to the merging of decay products from high momentum Higgs bosons. 
A search for Higgs pair production in the $HH\to b\bar{b}b\bar{b}$ final state was performed with $3.2 \ifb$ collected with ATLAS at $\sqrt{s} = 13 \TeV$ in 2015. 
The results are presented in this dissertation with a focus on a dedicated signal region for boosted final states. 
This signal region uses new techniques for recognizing jet substructure and $b$-tagging to the improve signal acceptance of high mass resonances. 



The discovery of the Standard Model (SM) Higgs boson (\h) \cite{Aad:2012tfa,Chatrchyan201230} at the Large Hadron Collider (LHC) motivates searches for new physics using the Higgs boson as a probe. In particular, many models predict cross sections for Higgs boson pair production that are significantly greater than the SM prediction. Resonant Higgs boson pair production is predicted by models such as the bulk Randall--Sundrum model~\cite{Agashe:2007zd,Fitzpatrick}, which features spin-2 Kaluza--Klein gravitons, \Grav, that subsequently decay to a pair of Higgs bosons. Extensions of the Higgs sector, such as two-Higgs-doublet models~\cite{PhysRevD.8.1226, Branco:2011iw}, propose the existence of a heavy spin-0 scalar that can decay into \h pairs. Enhanced non-resonant Higgs boson pair production is predicted by other models, for example those featuring light coloured scalars \cite{PhysRevD.86.095023} or direct $t\bar{t}\hh$ vertices \cite{Grober:2010yv,Contino:2012xk}.

Previous searches for Higgs boson pair production have all yielded null results. In the \bbbb channel, ATLAS searched for both non-resonant and resonant production in the mass range of 400--3000~\GeV\ using 3.2 $\mathrm{fb}^{-1}$ of 13~TeV data~\cite{EXOT-2015-11} collected during 2015. CMS searched for the production of resonances with masses of 750--3000~\GeV~\cite{CMS-B2G-16-026} using 13~TeV data and with masses 270--1100~\GeV\ with 8~TeV data~\cite{CMS-HIG-14-013}. Using 8 TeV data, ATLAS has examined the \bbbb~\cite{Aad:2015uka}, \bbgg~\cite{HIGG-2013-29}, \bbtautau and \WWgg channels, all of which were combined in Ref.~\cite{HIGG-2013-33}. CMS has performed searches using 13 TeV data for the  \bbtautau~\cite{CMS-HIG-17-002} and $bb\ell\nu\ell\nu$~\cite{CMSHIG17006} final states, and used 8 TeV data to search for \bbgg~\cite{Khachatryan:2016sey} in addition to a search in multilepton and multilepton+photons final states~\cite{Khachatryan:2014jya}.

The analyses presented in this paper exploit the dominant \hbb decay mode to search for Higgs boson pair production in both resonant and non-resonant production. Two analyses are presented, which are complementary in their acceptance, each employing a unique technique to reconstruct the Higgs boson. The ``resolved'' analysis is used for \hh systems in which the Higgs bosons have Lorentz boosts low enough that four $b$-jets can be reconstructed. The ``boosted'' analysis is used for those \hh systems in which the Higgs bosons have higher Lorentz boosts, which prevents the Higgs boson decay products from being resolved in the detector as separate \bjets. Instead, each Higgs boson candidate consists of a single large-radius jet, and $b$-decays are identified using smaller-radius jets built from charged-particle tracks.

Both analyses were re-optimized with respect to the former ATLAS publication~\cite{EXOT-2015-11}; an improved algorithm to pair $b$-jets to Higgs boson candidates is used in the resolved analysis, and in the boosted analysis an additional signal-enriched sample is utilized. The dataset comprises the 2015 and 2016 data, corresponding to 27.5 $\mathrm{fb}^{-1}$ for the resolved analysis and 36.1 $\mathrm{fb}^{-1}$ for the boosted analysis, with the difference due to the trigger selections used. The results are obtained using the resolved analysis for a resonance mass between 260 and 1400~\GeV, and the boosted analysis between 800~\GeV\ and 3000~\GeV. The main background is multijet production, which is estimated from data; the sub-leading background is \ttbar, which is estimated using both data and simulations. The two analyses employ orthogonal selections, and a statistical combination is performed in the mass range where they overlap. The final discriminants are the four-jet and dijet mass distributions in the resolved and boosted analyses, respectively. Searches are performed for the following benchmark signals: a spin-2 graviton decaying into Higgs bosons, a scalar resonance decaying into a Higgs boson pair, and SM non-resonant Higgs boson pair production.


\paragraph{}
This dissertation begins by discussing the status of di-Higgs. Chapter 1 gives an overview of double Higgs production in the Standard Model and beyond. Chapter 2 and 3 present details regarding the Large Hadron Collider and the ATLAS experiment. Chapter 4 provides an overview of object reconstruction in ATLAS, with a focus on Muon Segment Seeding. A brief interlude in Chapter 5 on the ATLAS Muon Data Quality, as this has been a focus of my graduate work. 

\paragraph{}
The rest of the dissertation presents a search for Higgs pair production in the $HH \to b\bar{b} b\bar{b}$ channel. Chapter 6 presents an overview of physics object selection, where the Higgs pairs are the result of the decay of a heavy resonance. Chapter 7 discusses the background estimation techinics in detail, followed by Chapter 8, Systematics. Chapter 9 presents the results, and Chapter 10 shows the limits between the boosted regime and the resolved regime, which is sensitive to lower mass resonances and non-resonant Higgs pair production. Finally, the work is summaried a conclusion and brief outlook of future Higgs physics with ATLAS.



% There are two types of analysis in particle physics. The first one is measurement, which yeilds a observable with an uncertainty. 
% This could either imrpove our current knowledge, or show some inconsistency. 
% The other type is search, which generally assumes some new physics model and try to justfy in data wether the new model is justified in some observables. 
% A successful search turns the subject into a measurement, yet a null search result will set a new limit for a given physics model.
