%!TEX root = ../dissertation.tex
\begin{savequote}[75mm]
A finely crafted blade will never meet as many blows on the battlefield as it did on the anvil.
%You gotta have a swine to show you where the truffles are.
%\qauthor{Edward Albee}
\end{savequote}

\chapter{Event Selection}
\label{sec:selection}

\begin{figure}[htb!]
  \centering
  \includegraphics[width=0.7\textwidth]{figures/theory/JiveXML_282712_595661746-YX-2018-04-16-20-50-34}
  \caption{2015 ATLAS boosted di-Higgs event candidate's event display. The data is taken in run 282712, lumiblock 503 and has event number 595661746. ID is in gray, ECAL is in green, HCAL is in red and MS is in blue. Only tracks with \pt $>25$ \GeV are shown. Jets are gray cones, and ID tracks are colored lines in the ID.}
  \label{fig:event_display}
\end{figure}

\paragraph{}
An \Xtohhb~ boosted event candidate observed in 2015 data is shown in Figure~\ref{fig:event_display}. 
There are $2b$ tagged tracks within in each of the large-\R jets. 
The one large-\R jet has $119$ \GeV mass and $543$ \GeV \pt, and the other one has $127$ \GeV mass and $413$ \GeV \pt. 
The $\Delta R$ between the two large-\R jet is $3.54$ and the invariant mass of the two large-\R jets is $1336$ \GeV.
This chapter explains why and how this event is selected as an \Xtohhb~ candidate.

\section{Data Cleaning}
\label{evt-sel:cleaning}
\paragraph{}
The following data cleaning requirements are made:
\begin{itemize}
\item Events with problems in SCT/TileCal/LAr are removed.
%\item Events that are affected by the recovery procedure for single event upsets in the SCT are removed.
\item Events that fail the jet cleaning procedure are removed. This is designed to exclude jets caused by detector noise, non-collision background, and cosmic rays. 
\item Incomplete events are removed.
\end{itemize}
% \paragraph{}
% Jets with $p_T<60$~\GeV\, $|\eta|<2.4$, and with a large fraction of their energy arising from pile-up interactions are suppressed using tracking information. 
% Events that pass a ``medium'' jet vertex tagger working point, corresponding to a 92\% efficiency for jets at the EM scale with $20<p_T<60$~\GeV, are retained in the analysis. 
% Quality criteria are applied to the jets, and events with jets consistent with noise in the calorimeter or non-collision backgrounds are vetoed~\cite{jetcleanATLAS}.
%\paragraph{}
The analysis also runs over the debug stream, which contains events recorded that could not be reconstructed on-line due to CPU time constraints. 
No event passing the full signal selection is found.

%%%%%%%%%%%%%%%%%%%%%%%%%%%%%%%%%%%%%%%%%%%%%%%%%%%%%%%%%%%%%%%%%%%%%%%%%%%%%%%%%%%%%%%%%%
\section{Trigger}
\label{evt-sel:trig}

\begin{figure}[htbp!]
\centering
\captionsetup{justification=centering}
    \begin{subfigure}[b]{0.45\textwidth}
        \includegraphics[width=\textwidth,angle=-90]{figures/boosted/Trigger/app_trig_b77_Efficiency_PreSel.pdf}
        \caption{Efficiency without extra requirement}
        \label{fig:boosted-trigger-HLT_Pre}
    \end{subfigure}
    \quad
    \begin{subfigure}[b]{0.45\textwidth}
        \includegraphics[width=\textwidth,angle=-90]{figures/boosted/Trigger/app_trig_b77_Efficiency_All.pdf}
        \caption{Efficiency with two large-\R jets requirement}
        \label{fig:boosted-trigger-HLT_All}
    \end{subfigure}
  %\includegraphics[width=0.45\textwidth,angle=-90]{figures/boosted/Trigger/trig_Moriond_Efficiency_PreSel.pdf}
  %\includegraphics[width=0.45\textwidth,angle=-90]{figures/boosted/Trigger/trig_Moriond_Efficiency_All.pdf}
\caption{Different trigger efficiencies as a function of the resonance signal mass with respect to all events without additional selections (left) and with respect to events passing the two large-\R jets and leading (subleading) jet \pt $> 400 (250)$ \GeV (right).}
\label{fig:boosted-trigger-HLT}
\end{figure}

\paragraph{}
%See \href{https://arxiv.org/pdf/1611.09661.pdf}{6.4.3} 
Events in data and MC are required to pass the lowest unprescaled large-\R jet trigger: \\
\verb|HLT_j360_a10_lcw| in 2015 and \verb|HLT_j420_a10_lcw| in 2016, where LCW stands for locally clustered weighted. 
The triggered jets are topo-cluster jets with local calibration weights and pile-up subtraction.
They are seeded by the lowest unprescaled L1 jet trigger, \texttt{L1\_J100}. 
LCW cluster triggers are chosen, because the alternative option, reclustered large-\R jet trigger, has a slower jet momentum turn-on in multi-jet events. 
Other options such as the lowest unprescaled transverse energy trigger, \verb|HLT_ht1000|, have a much slower turn-on compared to large-\R jet triggers.
Another trigger option is a four jet trigger \verb|HLT_4j100|, but because of boosted jet merging, the trigger efficiency decreases rapidly as the signal mass increases. 
The trigger efficiencies with respect to all events and events with two large-\R jets are shown in Figure~\ref{fig:boosted-trigger-HLT}.
For $1.4$ \TeV~ signal with two large-\R jets, the large-\R jet trigger efficiency is about $98\%$.
% with the requirement that the event has two large-\R jets that satisfy the \pt requirements: leading jet \pt $> 400$ \GeV, and subleading jet leading jet \pt $> 250$ \GeV. 
The trigger turn-on curve in 2015 and 2016 data, as a function of leading jet \pt, is shown in Figure~\ref{fig:boosted-trigger-HLT-turnon}.
Based on this figure, the large-\R jet \pt~ is required to be above $450$ \GeV, which is above the trigger turn-on threshold.

\begin{figure}[htbp!]
    \captionsetup{justification=centering}
    \begin{subfigure}[b]{0.45\textwidth}
        \includegraphics[width=\textwidth,angle=-90]{figures/boosted/Trigger/trig_15_b77_pT_Efficiency.pdf}
        \caption{2015}
        \label{fig:boosted-trigger-HLT-turnon-2015}
    \end{subfigure}
    \quad
    \begin{subfigure}[b]{0.45\textwidth}
        \includegraphics[width=\textwidth,angle=-90]{figures/boosted/Trigger/trig_16_b77_pT_Efficiency.pdf}
        \caption{2016}
        \label{fig:boosted-trigger-HLT-turnon-2016}
    \end{subfigure}
  \caption{Large-\R jet trigger (\texttt{HLT\_j360\_a10\_lcw}, left and \texttt{HLT\_j420\_a10\_lcw}, right) efficiencies with respect to the large-\R jet \pt, defined as the ratio between events that fired the trigger and the all the simulated events, measured in 2015 and 2016 data and \Grav~ with $c=1.0$ MC.}
  \label{fig:boosted-trigger-HLT-turnon}
\end{figure}


%%%%%%%%%%%%%%%%%%%%%%%%%%%%%%%%%%%%%%%%%%%%%%%%%%%%%%%%%%%%%%%%%%%%%%%%%%%%%%%%%%%%%%%%%%
\section{Object Selection}
\label{sec:obj-objectselection}

\begin{table}[bhp]
\caption{Physics objects and their technical names in the boosted analysis.} %%%
\begin{center}
\begin{tabular}{c|c}
  object & technical name \\
  \hline
  large-$R$ calorimeter jets & AntiKt10LCTopoTrimmedPtFrac5SmallR20Jets \\
  small-$R$ track jets       & AntiKt2PV0TrackJets \\
  b-tagging                  & on track jets, MV2c10, $70\%$ $b$-tagging wp \\
\end{tabular}
\label{tab:boosted-objects}
\end{center}
\end{table}

\begin{table}[bhp]
\begin{center}
\caption{List of kinematic selection used in the boosted analysis. These cuts are generally efficient for signal events.}
\begin{tabular}{l}
  kinematic selection \\
  \hline
  %$\bullet$ 
  leading large-\R jet must have $p_\text{T} > 450$ \GeV, $|\eta| < 2$, $m > 50$ \GeV \\
  sub-leading large-\R jet must have $p_\text{T} > 250$ \GeV, $|\eta| < 2$, $m > 50$ \GeV \\
  track jets must have $p_\text{T} > 10$ \GeV, $|\eta| < 2.5$ \\
  $|\Delta\eta| = |\eta_{\text{leadjet}} -\eta_{\text{subljet}} |< 1.7$
\end{tabular}
\label{tab:boosted-preselection}
\end{center}
\end{table}

\paragraph{}
The specific physics objects used in the boosted analysis are described in previous sections and reiterated in Table~\ref{tab:boosted-objects}.
The selection cuts are listed in Table~\ref{tab:boosted-preselection} and are explained in detail.

\begin{figure}[htbp!]
  \begin{center}
  \includegraphics[width=0.5\textwidth,angle=-90]{figures/boosted/Truth/truth_higgs-matching.pdf}
  \caption{Percentages of truth Higgs boson contained by a large-\R jet with $\Delta R_{Higgs, jet}<1.0$ as a function of \Grav~ masses. The three cases listed in the legend are orthogonal. Two Higgs bosons almost never match to the same large-\R jet.}
  \label{fig:truth-Higgs-largeRjet}
\end{center}
\end{figure}

\paragraph{}
Each event must have at least two high momentum large-\R jets, defined in section~\ref{obj:largeRjet}.
They are sorted by \pt.
The leading large-\R jet, also referred to as the leading Higgs candidate (leadJ), is defined to be the large-\R jet with the highest \pt.
The second highest \pt~ large-\R jet is called the subleading large-\R jet, also referred to as the subleading Higgs Candidate, or sublJ.
The large-\R jets are required to have \pt $> 250$ \GeV~, $|\eta| < 2$ to guarantee the tracking acceptance, and mass $> 50$ \GeV~ to reduce the size of data and simulations.
The leading large-\R jet is also required to have \pt $> 450$ \GeV~ to be above the trigger turn on threshold.
Only the leading and subleading large-\R jets are considered in the rest of this thesis.
Both truth-level Higgs bosons are matched to the two large-\R jets $\sim 95\%$ of the time for $1.2$\TeV~ signal MC, shown as the red curve in Figure~\ref{fig:truth-Higgs-largeRjet}. 
Requiring $R = 1.0$ ensures that the two $b$ quarks and their decay products are very likely to be contained within the large-\R jet, as shown in Figure~\ref{fig:truth-HbdR}. 

\begin{figure}[htbp!]
    \captionsetup{justification=centering}
    \begin{subfigure}[b]{0.45\textwidth}
        \includegraphics[width=\textwidth,angle=-90]{figures/boosted/Truth/truth_hbdR.pdf}
        \caption{leading Higgs boson}
        \label{fig:truth-HbdR-lead}
    \end{subfigure}
    \quad
    \begin{subfigure}[b]{0.45\textwidth}
        \includegraphics[width=\textwidth,angle=-90]{figures/boosted/Truth/truth_hbdR2.pdf}
        \caption{subleading Higgs boson}
        \label{fig:truth-HbdR-subl}
    \end{subfigure}
\caption{Normalized $\Delta R$ between the truth Higgs (leading on left, subleading on right) and the truth children $b$-quarks for \Grav~ MC. Lines are drawn at $\Delta R = 0.4$ ($R$ of small-\R jets) and $\Delta R = 1.0$ ($R$ of large-\R jets). }
\label{fig:truth-HbdR}
\end{figure}

\paragraph{}
The track jets, defined in section~\ref{obj:trackjet}, are required to have \pt $> 10$ \GeV, $|\eta| < 2.5$ and at least two associated tracks. 
A track jet is considered $b$-tagged if it has MV2c10 > $0.6455$ (see Section~\ref{sec:btag_70wp} for details).
Each large-\R jet is required to have at least one ghost-associated track jet.
Allowing only one ghost-associated track jet is motivated by the kinematic signatures of signals with masses above $2.5$ \TeV, where the resulting $R = 0.2$ track jets merge due to large Lorentz boosts.
If there is more than one track jet contained in the large-\R jet, they are also sorted by \pt.
The highest \pt~ track jet is defined to be the leading track jet, and the second highest \pt~ one is defined to be the subleading track jet.
Only the two leading \pt~ track jets are considered in this thesis.
One or two track jets are $\Delta R$ matched to the truth $b$ quarks $80\%$ of the time, as shown in Figure~\ref{fig:truth-bmatch}. 
In the Figure, red means the truth Higgs doesn't match the large-\R jet. 
Blue means both truth $b$s match the two track jets. 
Orange means two truth $b$s match the same track jet. 
Green indicates one $b$ quark has $\Delta R<0.2$ for both the leading and subleading track jets, and the other $b$ is matched to one of the two track jets. 
Pink means one $b$ matches to one of the two track jets, while the other $b$ doesn't match to the two leading track jets.

\begin{figure}[htbp!]
    \captionsetup{justification=centering}
    \begin{subfigure}[b]{0.45\textwidth}
        \includegraphics[width=\textwidth,angle=-90]{figures/boosted/Truth/truth_b-matching.pdf}
        \caption{leading Higgs boson}
        \label{fig:truth-bmatch-lead}
    \end{subfigure}
    \quad
    \begin{subfigure}[b]{0.45\textwidth}
        \includegraphics[width=\textwidth,angle=-90]{figures/boosted/Truth/truth_b-matching-sublead.pdf}
        \caption{subleading Higgs boson}
        \label{fig:truth-bmatch-subl}
    \end{subfigure}
\caption{Percentage of $\Delta R<0.2$ matching truth $b$'s to track jets (leading Higgs on the left, subleading Higgs on the right) for different \Grav~ masses. The cases listed in the legend are orthogonal to each other. The cases not listed on the legend (including when a truth $b$ is not contained in the large-\R jet) happen in total at most $1.6\%$ of the total for a given \Grav~ mass.}
\label{fig:truth-bmatch}
\end{figure}


\begin{figure*}[htb!]
\begin{center}
    \captionsetup{justification=centering}
    \begin{subfigure}[b]{0.45\textwidth}
        \includegraphics[width=\textwidth]{figures/boosted/muons/h1_mass_dbl.pdf}
        \caption{\mleadJ}
        \label{fig:boosted-muons-signal-leadj}
    \end{subfigure}
    \quad
    \begin{subfigure}[b]{0.45\textwidth}
        \includegraphics[width=\textwidth]{figures/boosted/muons/h2_mass_dbl.pdf}
        \caption{\msublJ}
        \label{fig:boosted-muons-signal-sublj}
    \end{subfigure}\\
    \captionsetup{justification=centering}
    \begin{subfigure}[b]{0.45\textwidth}
        \includegraphics[width=\textwidth]{figures/boosted/muons/hh_mass_dbl.pdf}
        \caption{\mtwoJ}
        \label{fig:boosted-muons-signal-mjj}
    \end{subfigure}
    \quad
    \begin{subfigure}[b]{0.45\textwidth}
        \includegraphics[width=\textwidth]{figures/boosted/muons/h12_corr_mass.pdf}
        \caption{\mleadJ-\msublJ}
        \label{fig:boosted-muons-signal-mj2d}
    \end{subfigure}
  \caption{Kinematics of $1$\TeV~ \Grav~ MC before (blue) and after (red) muon-in-jet corrections. The reconstructed Higgs masses (top row, left for leading large-\R jet, right for subleading large-\R jet) are closer to $125$\GeV after the correction, which improves the signal efficiency for the signal region selection by $\sim\!10\%$ (bottom row, left for $m_{JJ}$, right for event distribution differences on the leading-subleading large-\R jet mass plane.).}
  \label{fig:boosted-muons-signal}
\end{center}
\end{figure*}

\paragraph{}
Muon correction to the large-\R jet four-momenta is applied.
$b$-hadrons decay to a muon and a muon neutrino through a $W$ boson with $20\%$ probability.
The energy carried by the muon is not measured in the large-\R jet.
A further muon correction on large-\R jet four-momentum accounts for this energy loss.
The muon-in-jet corrections are applied only after the fiducial large-\R jet requirements on \pt~ and $\eta$.
The muons are required have $\Delta R < 0.2$ with the $b$-tagged track jets within each large-\R jet. 
In the case where more than one muon is found within a track jet, only the muon with the smallest $\Delta R$ is considered. 
If two $b$-tagged track jets are found to have muons, both corrections are considered. 
The four-momenta of the matched muon is added to the large-\R jet four-momentum, with the muon calorimeter energy deposits subtracted. 
This correction is only applied to the calorimeter mass portion of the combined mass, because the track mass calculation already contains the muon track. 
The muon-in-jet correction improves the overall large-\R jet mass resolution by approximately $5\%$, and Figure~\ref{fig:boosted-muons-signal} shows the impact of this correction on the $1$\TeV~ \Grav.

\begin{figure*}
\begin{center}
  \includegraphics[width=0.45\textwidth,angle=-90]{figures/boosted/Other/AllTag_Signal_hCandDeta_F_c10-cb-no-deta-cut_truth_0.pdf}
  \caption{ After large-\R jet requirements, normalized $\Delta \eta_{JJ}$ distribution in $1.5$ \TeV \Grav~ and data, where the data consists of mostly multijet events (> 90$\%$). Higgs candidate refers to the two large-\R jets. The background multijet events are more forward in $\Delta \eta_{JJ}$ distribution.}
\label{fig:app-check-deta}
\end{center}
\end{figure*}

\paragraph{}
Finally, the Higgs candidates (large-\R jets) are also required to have $|\Delta\eta| = |\eta_{\text{leadJ}} -\eta_{\text{sublJ}} |< 1.7$. 
This is because the spin 2 \Grav~ production is mostly $s$-channel with a more central distribution in $\eta$, while the multi-jet events have larger $t$-channel or $u$-channel productions with a more forward distribution. 
Figure ~\ref{fig:app-check-deta} shows the $\Delta \eta_{JJ}$ distribution for signal MC and data in inclusive $b$-tag channels.
This cut is not entirely optimal for scalar signals due to the different spin, but it is fixed for both \Grav~ and scalar selections.



%%%%%%%%%%%%%%%%%%%%%%%%%%%%%%%%%%%%%%%%%%%%%%%%%%%%%%%%%%%%%%%%%%%%%%%%%%%%%%%%%%%%%%%%%%
\section{Resolved Veto}
\label{sec:resollvedveto}

\paragraph{}
Sometimes one event can be reconstructed in both the resolved method as four small-\R jets and the boosted method as two large-\R jets with track jets. Figure~\ref{fig:obj_evt_display} shows an example event display of collision data recorded in 2015.

\begin{figure}[htbp!]
\centering
\captionsetup{justification=centering}
    \begin{subfigure}[b]{0.45\textwidth}
        \includegraphics[width=\textwidth]{figures/object/JiveXML_282631_530527390-YX-RZ-YZ-EventInfo-2016-02-29-15-48-39}
        \caption{resolved reconstruction}
        \label{fig:obj_evt_display_resolved}
    \end{subfigure}
    \quad
    \begin{subfigure}[b]{0.45\textwidth}
        \includegraphics[width=\textwidth]{figures/object/JiveXML_282631_530527390-YX-RZ-YZ-EventInfo-2016-02-29-15-48-08}
        \caption{roosted reconstruction}
        \label{fig:obj_evt_display_boosted}
    \end{subfigure}
\caption{Event display of the same event in 2015 data using resolved~\ref{fig:obj_evt_display_resolved} and  boosted~\ref{fig:obj_evt_display_boosted} topologies. The resolved reconstruction gives a $m_{4J}$ of $873$ \GeV~, and the boosted reconstruction gives $m_{2J}$ of $852$ \GeV. }
\label{fig:obj_evt_display}
\end{figure}

\paragraph{}
In order to avoid events being reconstructed by both the resolved and the boosted analysis, events that pass the resolved signal region selections are vetoed in the boosted analysis.
This is an ATLAS political decision, and the effect of vetoing boosted event selection in the resolved analysis was never tested.
The gain is a full statistical combination of the resolved and boosted result.
This veto reduces the boosted analysis sensitivity in the resonant mass region up to $1.5$ \TeV, which is recovered in the statistical combination with the resolved analysis.
Hence, it is necessary to introduce the resolved selection~\cite{Aaboud:2018knk}.

\paragraph{}
For the resolved analysis, the four small-\R jets with the highest $b$-tagging score are paired to construct two Higgs boson candidates.  
Each jet must have \pt $> 40$ \GeV~, $|\eta| < 2.5$, MV2c10 $> 0.8244$ (small-\R jet $70\%$ $b$-tagging working point). 
Pairings of jets are only accepted as Higgs candidates if they satisfy the following requirements, where \mfourj~ is expressed in \GeV:\\
if \mfourj  < 1250\,\GeV:
\begin{equation}
\frac{360\,\GeV}{\mfourj} - 0.5 < \DR_{jj}^{lead} < \frac{653\,\GeV}{\mfourj} + 0.475;\quad
\frac{235\,\GeV}{\mfourj} < \DR_{jj}^{subl}  < \frac{875\,\GeV}{\mfourj} + 0.35
\end{equation}
\quad if \mfourj  > 1250\,\GeV:
\begin{equation}
0< \DR_{jj}^{lead} < 1;\quad 0 < \DR_{jj}^{subl} < 1
\end{equation}
In these expressions, $\DR_{jj, \mathrm{lead}}$ is the angular distance between jets in the leading Higgs boson candidate and $\DR_{jj, \mathrm{subl}}$ for the sub-leading candidate. 
The leading Higgs boson candidate is defined to be the candidate with the highest scalar sum of jet \pt. 
This $\Delta R$ requirement efficiently rejects jet-pairings where one of the $b$-tagged jets did not originate from a Higgs boson decay.

\paragraph{}
Also, mass-dependent requirements are made on the leading Higgs boson candidate \pt~ and the sub-leading Higgs boson \pt:
\begin{equation}
p_T^{lead} > 0.5\mfourj - 105\,\GeV, \quad
p_T^{subl} > 0.33\mfourj - 75\,\GeV
\end{equation}
where \mfourj~ is again expressed in \GeV.

\paragraph{}
A further (\mfourj-independent) requirement is placed on the pseudorapidity difference between the two Higgs boson candidates, $\dEta < 1.5$, which rejects multi-jet events.
\begin{equation}
\dEta < 1.1 \quad \mathrm{if}\ \mfourj < 850\,\GeV , \quad
\dEta < 2\times10^{-3} \mfourj - 0.6 \quad \mathrm{if}\ \mfourj > 850\,\GeV
\end{equation}

\paragraph{}
Events that have multiple Higgs boson candidates satisfying these requirements (which happens often when $\mfourj < 500\,\GeV$) necessitate an algorithm to choose the correct pairs. 
The optimal choice would be the combination most consistent with the decays of two Higgs bosons of equal mass.
To account for energy loss in measurement and reconstruction resolution, the requirement of equal masses is modified. 
The distance, \Dhh, of the leading and subleading Higgs boson candidate masses, $\left(\leadm, \sublm\right)$ from the line connecting $\left(0\,\GeV, 0\,\GeV\right)$ and $\left(120\,\GeV, 110\,\GeV\right)$ is computed, and the pairing with the smallest value of \Dhh~ is chosen.
The values of 120\,\GeV\, and 110\,\GeV\, are chosen because they correspond to the median values of the narrowest large-\R jet mass intervals that contain $90\%$ of the signal in simulations. %the centre of the signal region in \leadm and \sublm respectively,
\Dhh~ can be expressed as follows:
\begin{equation}
D_{hh} = \frac{\left|\leadm - \frac{120}{110}\sublm\right|}{\sqrt{1+\left(\frac{110}{120}\right)^{2}}}.
\end{equation}

\paragraph{}
A requirement on the Higgs boson candidate mass is used to define the resolved signal region:
\begin{equation}
X_{hh-resolved} = \sqrt{\left(\frac{\mlead - 120\,\GeV}{0.1\mlead}\right)^2 + \left(\frac{\msubl - 110\,\GeV}{0.1\msubl}\right)^2} < 1.6,
\label{eqn:resolvedXhh}
\end{equation}
where the $0.1$\mtwoj~ terms represent the widths of the leading and sub-leading Higgs boson candidate mass distributions, derived from simulation.
In summary, any event is rejected in the boosted selection if the \Dhh~ minimized Higgs candidate passes the resolved signal region $X_{hh-resolved}$ cut.

%For more detail, please see the resolved signal region definition. 
%For the impact on the boosted analysis, see Appendix ~\ref{sec:app-optimization-resveto}.

%%%%%%%%%%%%%%%%%%%%%%%%%%%%%%%%%%%%%%%%%%%%%%%%%%%%%%%%%%%%%%%%%%%%%%%%%%%%%%%%%%%%%%%%%%

\section{2D Higgs Mass Cut}
\begin{figure*}[htbp!]
\centering
\captionsetup{justification=centering}
    \hspace{-2cm}
    \begin{subfigure}[b]{0.4\textwidth}
        \includegraphics[width=\textwidth,angle=-90]{figures/boosted/Truth/Sig_1200_AllTag_Incl_mH0H1.pdf}
        \caption{$1.2$\TeV~ \Grav}
        \label{fig:evt-signal-mhh_1200}
    \end{subfigure}
    \quad \quad \quad \quad
    \begin{subfigure}[b]{0.4\textwidth}
        \includegraphics[width=\textwidth,angle=-90]{figures/boosted/Truth/Sig_2000_AllTag_Incl_mH0H1.pdf}
        \caption{$2$\TeV~ \Grav}
        \label{fig:evt-signal-mhh_2000}
    \end{subfigure}
\caption{For RSG $c=1.0$ samples, the number of events as a function of leading Higgs candidate mass and subleading Higgs candidate mass, for $1.2$ \TeV~(left) signal and $2$ \TeV~(right) signal samples. The red dotted line in the center correspond to the signal region, passing $X_{hh} < 1.6$.}
\label{fig:evt-signal-mhh}
\end{figure*}

\paragraph{}
To separate di-Higgs decays from background productions like QCD multi-jets and top, requirements on the leading and subleading large-\R jet masses are imposed.
Figure ~\ref{fig:evt-signal-mhh} shows the \Grav~ MC \mleadJ-\msublJ 2D distribution.
The signal is centered around the Higgs boson mass $125$\GeV~ in the 2D plane.

\paragraph{}
The signal region is defined using the expression~\ref{eq:boosted_XhhDef}:
\begin{equation}
\label{eq:boosted_XhhDef}
X_{hh} = \sqrt{\left(\frac{m^{\rm lead}_{\rm J} - \text{124 GeV}}{0.1 \left(m^{\rm lead}_{\rm J}\right)}\right)^2 + \left(\frac{m^{\rm subl}_{\rm J}- \text{115 GeV}}{0.1 \left(m^{\rm subl}_{\rm J}\right)}\right)^2}
\end{equation}
The denominator of each term in \Xhh~ represents the resolution of the reconstructed leading and subleading jet masses.
Hence \Xhh~ can be interpreted as a $\chi^2$ compatibility with the di-Higgs hypothesis.
The subleading large-\R jet typically has a biased downward reconstructed mass.
Because it is the subleading jet and hence has lower boost.
The jet decay constituents are sometimes outside of the jet cone.
The energy losses from neutrinos in leptonic $b$ decays and cracks in the calorimeter also contribute.
The subleading jet mass center value of $115$\GeV~ accounts for the larger energy loss in subleading jets.
The signal region requires $X_{hh} < 1.6$. 
This cut keeps $\sim 40\%$ of signal events passing all selections. 
A more optimal signal region definition, using asymmetric signal jet mass resolution and a momentum dependent cut accounting for the higher mass resolution of lager \pt~ jets, can improve the overall search sensitivity by $2-8\%$.
Since the gain in sensitivity is small, the signal region is kept to be consistent with the $X_{hh-resolved}$.

%The denominator of each term in the definition can be interpreted as a resolution on the reconstructed mass of $8.5\%$ for the leading jet and $12\%$ for the subleading jet, hence $X_{hh}$ can be interpreted as a $\chi^2$ compatibility with the $hh$ hypothesis. Similarly to the resolved analysis, these $\sigma \left(m_{\rm J}\right)$ are only a rough approximation to the true resolution, but the $X_{hh}$ requirement gives nearly optimal performance. 
%The last substraction for high $p_\text{T}$ cases is because of the higher mass resolution of high mass signal samples. This effectively increases the size of $X_{hh}$ to 1.8 for a 3 TeV signal.

\section{Number of $b$-tagged track jet requirements}
\paragraph{}
In addition to requirements for basic object selection and the \Xhh~ cut, the signal regions are defined by further requiring multiple $b$-tags which are consistent with the di-Higgs decay.
The MC events passing all signal region selections are sorted into different $b$-tagging categories. 
This is shown in Figure~\ref{fig:boosted-nbjet-signal-efficiency}. 

\begin{figure*}[htbp!]
    \captionsetup{justification=centering}
    \begin{subfigure}[b]{0.45\textwidth}
        \includegraphics[width=\textwidth,angle=-90]{figures/boosted/SigEff/region_lst_Moriond_Efficiency_AllTag_Signal.pdf}
        \caption{all $n$-$b$tag regions}
        \label{fig:boosted-nbjet-signal-efficiency-region}
    \end{subfigure}
    \quad
    \begin{subfigure}[b]{0.45\textwidth}
        \includegraphics[width=\textwidth,angle=-90]{figures/boosted/SigEff/detail_lst_Moriond_Efficiency_AllTag_Signal.pdf}
        \caption{$n$-$b$tag regions with specific number of track jet}
        \label{fig:boosted-nbjet-signal-efficiency-detail}
    \end{subfigure}
  \caption{Signal fraction in different $b$-tag categories (left) and detailed fraction in different number of track jet and $b$-tag categories (right) as a function of signal resonance mass hypothesis for selection cuts. The efficiencies are relative to the total number of events passing the 2D mass cut.}
  \label{fig:boosted-nbjet-signal-efficiency}
\end{figure*}

\paragraph{}
\Xtohhb~ results in four $b$-quarks, which ideally leads to four track jets passing $b$-tagging requirements.
This is defined as the $4b$ channel.
The $4b$ requirement has an overall efficiency of roughly $\epsilon^4$, where $\epsilon$ is the $b$-tagging efficiency chosen to be $70\%$.
This means an overall $0.7^4 \sim 0.24$ probability, but having one actual $b$-jet failing $b$-tagging while the other three passing has probability $3 \times 0.7^3 \times (1-0.7) \sim 0.31$.
Therefore, a $3b$ selection is also introduced to recover the signal efficiency. 
An event with $3b$-tags must have at least $3b$-tagged track jets, but can have any number of additional un-tagged track jets.
In $4b$ and $3b$, each Higgs candidate can have at most two $b$-tagged track jets, hence $\geq 3b$-tagged track jets cannot be in the same large-\R jet.

\paragraph{}
At the highest resonance mass, the Lorentz boost of the Higgs boson can be large enough to collimate the $b$-quarks, resulting in $\Delta R_{bb} < 0.2$.
The two $b$-quarks decay products are therefore contained inside only one $R = 0.2$ track jet instead of two.
This decreases the selection efficiency in $4b$ and $3b$ channels.
A third signal channel is added to account for this, denoted by two-tag-split or simply $2bs$.
It requires that exactly one $b$-tagged track jet is found in each Higgs candidate, plus an arbitrary number of track jets that must fail the $b$-tag.
For masses above $2.5$ \TeV, the $2bs$ channel (where each large-\R jet has exactly one $b$-tagged track jet) significantly improves signal acceptance.

\paragraph{}
$4b$, $3b$, and $2bs$ channels are chosen as three signal channels, also referred to as $n$-$b$tag channels.
The other $b$-tagging channels, also referred to as lower-$b$tag channels, are also sorted and studied.
$2b$ channel is defined as one large-\R jet has two $b$-tagged track jets, and the other large-\R jet has no $b$-tagged track jet. 
$1b$ channel is defined as one large-\R jet has one and only one $b$-tagged track jets, and the other large-\R jet has no $b$-tagged track jet. 
$0b$ channel is defined as both large-\R jets have no $b$-tagged track jet. 
They all have relatively small signal acceptance. 



%%%%%%%%%%%%%%%%%%%%%%%%%%%%%%%%%%%%%%%%%%%%%%%%%%%%%%%%%%%%%%%%%%%%%%%%%%%%%%%%%%%%%%%%%%
\section{Signal efficiency and cutflow}
\paragraph{}
Acceptance means the geometric fiducial volume of the detector. 
Efficiency refers to the detector effectiveness in finding objects.
The signal efficiency as a function of \Grav~ resonance mass is shown in Figure~\ref{fig:boosted-selection-efficiency}, both for the absolute signal efficiency and for the efficiency relative to the previous cut in the selection.
Above a mass of $\sim\!1$ \TeV, the reconstruction of high momentum large-\R jets with small $\Delta\eta$ is efficient, as shown in the orange-triangle curve.
The biggest limitation of the boosted analysis is $b$-tagging inefficiency for track jets with momentum above $500$ \GeV, as shown in the pink-circle curve.
The second biggest limitation is the large-\R jet mass resolution, as shown in the purple-square curve.
Across the mass range considered, the signal jet masses requirement ($X_\text{hh}$) and $b$-tagging requirements are $\mathcal{O}(40\%)$ efficient relative to the previous cuts.

\begin{figure*}[htbp!]
    \captionsetup{justification=centering}
    \begin{subfigure}[b]{0.45\textwidth}
        \includegraphics[width=\textwidth,angle=-90]{figures/boosted/SigEff/evtsel_Moriond_Efficiency_PreSel.pdf}
        \caption{all $n$-$b$tag channels}
        \label{fig:boosted-selection-efficiency-abs}
    \end{subfigure}
    \quad
    \begin{subfigure}[b]{0.45\textwidth}
        \includegraphics[width=\textwidth,angle=-90]{figures/boosted/SigEff/evtsel_Moriond_Efficiency_PreSel_rel.pdf}
        \caption{$n$-$b$tag channels with specific number of track jet}
        \label{fig:boosted-selection-efficiency-rel}
    \end{subfigure}
  \caption{Selection efficiency (left) and relative efficiency with respect to the previous cut (right) as a function of RSG c=1.0 signal resonance mass hypothesis for selection cuts. The relative efficiency is defined from the previous cut, where the order of cuts is given by the legend. PassTrig means the event passes the trigger selection; PassDiJetPt means the event passes the leading and sub-leading jet \pt cuts; PassDiJetEta means the event passes the leading and sub-leading jet $\eta$ cuts; PassDetaH means the events passes the $|\Delta \eta| < 1.7$ cut; PassBJetSkim means the event contains at least two $b$-tagged track jets, inclusive of $2b$, $2bs$, $3b$ and $4b$ configurations; PassSignal means the event passes the signal region cut $X_{hh} < 1.6$.}
  \label{fig:boosted-selection-efficiency}
\end{figure*}


\begin{table}[htb!]
\scriptsize
\caption{The selection efficiency for $G_{KK}^{*}\rightarrow hh\rightarrow b\bar{b}b\bar{b}$ events ($c=1.0$) at each stage of the event selection. Uncertainties are the MC stat uncertainty only.}
\begin{center}
\resizebox{\textwidth}{!}{
\begin{footnotesize} 
\begin{tabular}{c|c|c|c|c|c|c|c} 
Resonance Mass [GeV] & Mini-ntuple Skimming & 2 large-R jets & $\Delta\eta$ & Xhh < 1.6 & 2bs SR & 3b SR & 4b SR \\ 
\hline\hline 
500 & 317.31 $\pm$ 6.0 & 295.75 $\pm$ 5.79 & 164.5 $\pm$ 4.32 & 8.45 $\pm$ 0.99 & 1.08 $\pm$ 0.37 & 2.14 $\pm$ 0.52 & 0 $\pm$ 0\\ 
600 & 269.07 $\pm$ 3.64 & 247.94 $\pm$ 3.5 & 136.31 $\pm$ 2.59 & 11.31 $\pm$ 0.76 & 2.57 $\pm$ 0.37 & 3.84 $\pm$ 0.45 & 0.66 $\pm$ 0.19\\ 
700 & 253.68 $\pm$ 3.35 & 226.93 $\pm$ 3.16 & 124.83 $\pm$ 2.35 & 16.79 $\pm$ 0.86 & 3.74 $\pm$ 0.42 & 6.99 $\pm$ 0.56 & 1.91 $\pm$ 0.29\\ 
800 & 286.26 $\pm$ 2.28 & 245.36 $\pm$ 2.11 & 129.2 $\pm$ 1.53 & 24.41 $\pm$ 0.67 & 5.11 $\pm$ 0.31 & 11.27 $\pm$ 0.46 & 4.13 $\pm$ 0.27\\ 
900 & 306.51 $\pm$ 1.61 & 275.57 $\pm$ 1.52 & 158.03 $\pm$ 1.15 & 40.72 $\pm$ 0.59 & 8.81 $\pm$ 0.28 & 19.76 $\pm$ 0.41 & 7.5 $\pm$ 0.25\\ 
1000 & 238.2 $\pm$ 0.98 & 226.98 $\pm$ 0.96 & 165.2 $\pm$ 0.82 & 52.86 $\pm$ 0.47 & 10.87 $\pm$ 0.22 & 26.0 $\pm$ 0.33 & 10.07 $\pm$ 0.2\\ 
1100 & 164.5 $\pm$ 0.63 & 160.94 $\pm$ 0.63 & 132.53 $\pm$ 0.57 & 45.26 $\pm$ 0.34 & 9.55 $\pm$ 0.16 & 21.88 $\pm$ 0.23 & 9.03 $\pm$ 0.14\\ 
1200 & 109.24 $\pm$ 0.41 & 107.92 $\pm$ 0.4 & 93.45 $\pm$ 0.38 & 33.53 $\pm$ 0.23 & 6.96 $\pm$ 0.11 & 15.8 $\pm$ 0.16 & 7.38 $\pm$ 0.1\\ 
1300 & 72.72 $\pm$ 0.59 & 72.2 $\pm$ 0.59 & 63.74 $\pm$ 0.56 & 24.19 $\pm$ 0.35 & 5.02 $\pm$ 0.17 & 11.33 $\pm$ 0.24 & 5.45 $\pm$ 0.16\\ 
1400 & 48.83 $\pm$ 0.17 & 48.61 $\pm$ 0.17 & 42.96 $\pm$ 0.16 & 16.62 $\pm$ 0.1 & 3.72 $\pm$ 0.052 & 7.61 $\pm$ 0.07 & 3.68 $\pm$ 0.046\\ 
1500 & 33.13 $\pm$ 0.12 & 33.02 $\pm$ 0.12 & 29.25 $\pm$ 0.11 & 11.31 $\pm$ 0.07 & 2.67 $\pm$ 0.036 & 5.08 $\pm$ 0.047 & 2.44 $\pm$ 0.031\\ 
1600 & 22.81 $\pm$ 0.08 & 22.75 $\pm$ 0.08 & 20.16 $\pm$ 0.075 & 7.74 $\pm$ 0.048 & 1.93 $\pm$ 0.025 & 3.48 $\pm$ 0.032 & 1.53 $\pm$ 0.02\\ 
1800 & 11.2 $\pm$ 0.1 & 11.18 $\pm$ 0.1 & 9.93 $\pm$ 0.094 & 3.71 $\pm$ 0.059 & 1.1 $\pm$ 0.034 & 1.6 $\pm$ 0.038 & 0.6 $\pm$ 0.022\\ 
2000 & 5.72 $\pm$ 0.021 & 5.71 $\pm$ 0.021 & 5.07 $\pm$ 0.019 & 1.83 $\pm$ 0.012 & 0.6 $\pm$ 0.0072 & 0.76 $\pm$ 0.0076 & 0.25 $\pm$ 0.0041\\ 
2250 & 2.61 $\pm$ 0.0088 & 2.61 $\pm$ 0.0088 & 2.32 $\pm$ 0.0083 & 0.78 $\pm$ 0.005 & 0.31 $\pm$ 0.0032 & 0.3 $\pm$ 0.003 & 0.078 $\pm$ 0.0014\\ 
2500 & 1.24 $\pm$ 0.0054 & 1.24 $\pm$ 0.0054 & 1.11 $\pm$ 0.0051 & 0.33 $\pm$ 0.0028 & 0.16 $\pm$ 0.002 & 0.11 $\pm$ 0.0016 & 0.021 $\pm$ 0.00066\\ 
2750 & 0.6 $\pm$ 0.0026 & 0.6 $\pm$ 0.0026 & 0.54 $\pm$ 0.0025 & 0.14 $\pm$ 0.0013 & 0.081 $\pm$ 0.00099 & 0.038 $\pm$ 0.00065 & 0.0055 $\pm$ 0.00024\\ 
3000 & 0.3 $\pm$ 0.0011 & 0.3 $\pm$ 0.0011 & 0.27 $\pm$ 0.0011 & 0.058 $\pm$ 0.00051 & 0.039 $\pm$ 0.00041 & 0.013 $\pm$ 0.00023 & 0.0016 $\pm$ 8e-05\\ 
\hline\hline 
\end{tabular} 
\end{footnotesize} 
\newline 

}
\end{center}
\label{boosted-eff-RSG_c10}
\end{table}

\begin{table}[htb!]
\scriptsize
\caption{The selection efficiency for $G_{KK}^{*}\rightarrow hh\rightarrow b\bar{b}b\bar{b}$ events ($c=2.0$) at each stage of the event selection. Uncertainties are the MC stat uncertainty only.}
\begin{center}
\resizebox{\textwidth}{!}{
\begin{footnotesize} 
\begin{tabular}{c|c|c|c|c|c|c|c} 
Resonance Mass [GeV] & Mini-ntuple Skimming & 2 large-R jets & $\Delta\eta$ & Xhh < 1.6 & 2bs SR & 3b SR & 4b SR \\ 
\hline\hline 
& & & & & & &\\ 
500 & 3705.15 $\pm$ 40.86 & 3479.44 $\pm$ 39.59 & 2325.18 $\pm$ 32.37 & 568.04 $\pm$ 16.17 & 122.56 $\pm$ 7.76 & 253.49 $\pm$ 10.78 & 100.7 $\pm$ 6.53\\ 
600 & 2549.14 $\pm$ 22.55 & 2374.01 $\pm$ 21.76 & 1591.92 $\pm$ 17.82 & 396.96 $\pm$ 9.03 & 89.01 $\pm$ 4.46 & 178.63 $\pm$ 6.05 & 74.31 $\pm$ 3.71\\ 
700 & 1928.4 $\pm$ 13.57 & 1782.85 $\pm$ 13.04 & 1183.86 $\pm$ 10.63 & 320.53 $\pm$ 5.62 & 71.38 $\pm$ 2.76 & 148.41 $\pm$ 3.81 & 59.31 $\pm$ 2.31\\ 
800 & 1595.14 $\pm$ 8.82 & 1457.71 $\pm$ 8.43 & 958.89 $\pm$ 6.84 & 268.75 $\pm$ 3.67 & 64.14 $\pm$ 1.86 & 123.48 $\pm$ 2.47 & 49.43 $\pm$ 1.51\\ 
900 & 1264.78 $\pm$ 5.77 & 1179.88 $\pm$ 5.58 & 819.75 $\pm$ 4.65 & 251.29 $\pm$ 2.61 & 55.63 $\pm$ 1.27 & 119.44 $\pm$ 1.79 & 48.72 $\pm$ 1.11\\ 
1000 & 891.0 $\pm$ 3.66 & 856.95 $\pm$ 3.59 & 662.54 $\pm$ 3.15 & 219.04 $\pm$ 1.84 & 49.45 $\pm$ 0.91 & 104.31 $\pm$ 1.26 & 42.97 $\pm$ 0.78\\ 
1100 & 595.58 $\pm$ 2.98 & 581.72 $\pm$ 2.95 & 481.67 $\pm$ 2.68 & 167.96 $\pm$ 1.61 & 37.64 $\pm$ 0.79 & 78.28 $\pm$ 1.09 & 34.59 $\pm$ 0.7\\ 
1200 & 390.84 $\pm$ 1.69 & 385.41 $\pm$ 1.68 & 330.23 $\pm$ 1.55 & 118.0 $\pm$ 0.94 & 26.23 $\pm$ 0.46 & 54.18 $\pm$ 0.64 & 25.34 $\pm$ 0.42\\ 
1300 & 257.66 $\pm$ 0.94 & 255.35 $\pm$ 0.94 & 222.37 $\pm$ 0.88 & 82.11 $\pm$ 0.54 & 19.04 $\pm$ 0.27 & 37.8 $\pm$ 0.37 & 16.99 $\pm$ 0.23\\ 
1400 & 172.09 $\pm$ 0.72 & 171.02 $\pm$ 0.71 & 150.22 $\pm$ 0.67 & 56.23 $\pm$ 0.42 & 13.6 $\pm$ 0.22 & 25.36 $\pm$ 0.28 & 11.78 $\pm$ 0.18\\ 
1500 & 116.25 $\pm$ 0.41 & 115.72 $\pm$ 0.41 & 101.92 $\pm$ 0.39 & 38.5 $\pm$ 0.24 & 9.94 $\pm$ 0.13 & 17.04 $\pm$ 0.16 & 7.64 $\pm$ 0.1\\ 
1600 & 80.09 $\pm$ 0.28 & 79.82 $\pm$ 0.28 & 70.48 $\pm$ 0.26 & 26.24 $\pm$ 0.16 & 7.01 $\pm$ 0.09 & 11.62 $\pm$ 0.11 & 4.92 $\pm$ 0.067\\ 
1800 & 38.99 $\pm$ 0.14 & 38.9 $\pm$ 0.14 & 34.39 $\pm$ 0.13 & 12.65 $\pm$ 0.081 & 3.82 $\pm$ 0.047 & 5.46 $\pm$ 0.053 & 2.02 $\pm$ 0.03\\ 
2000 & 19.94 $\pm$ 0.088 & 19.91 $\pm$ 0.088 & 17.68 $\pm$ 0.083 & 6.17 $\pm$ 0.05 & 2.15 $\pm$ 0.031 & 2.52 $\pm$ 0.032 & 0.85 $\pm$ 0.017\\ 
2250 & 9.02 $\pm$ 0.031 & 9.01 $\pm$ 0.031 & 7.99 $\pm$ 0.029 & 2.62 $\pm$ 0.017 & 1.03 $\pm$ 0.011 & 1.02 $\pm$ 0.01 & 0.28 $\pm$ 0.0051\\ 
2500 & 4.28 $\pm$ 0.016 & 4.28 $\pm$ 0.016 & 3.8 $\pm$ 0.015 & 1.13 $\pm$ 0.0083 & 0.52 $\pm$ 0.0058 & 0.4 $\pm$ 0.0048 & 0.098 $\pm$ 0.0022\\ 
3000 & 1.07 $\pm$ 0.004 & 1.07 $\pm$ 0.004 & 0.96 $\pm$ 0.0038 & 0.23 $\pm$ 0.0019 & 0.13 $\pm$ 0.0015 & 0.062 $\pm$ 0.00097 & 0.013 $\pm$ 0.00043\\ 
& & & & & & &\\ 
\hline\hline 
\end{tabular} 
\end{footnotesize} 
\newline 

}
\end{center}
\label{boosted-eff-RSG_c20}
\end{table}

\begin{table}[htb!]
\scriptsize
\caption{The selection efficiency for $H\rightarrow hh\rightarrow b\bar{b}b\bar{b}$ events at each stage of the event selection.}
\begin{center}
\resizebox{\textwidth}{!}{
\begin{footnotesize} 
\begin{tabular}{c|c|c|c|c|c|c|c} 
Resonance Mass [GeV] & Mini-ntuple Skimming & 2 large-R jets & $\Delta\eta$ & Xhh < 1.6 & 2bs SR & 3b SR & 4b SR \\ 
\hline\hline 
& & & & & & &\\ 
500 & 1557.94 $\pm$ 136.12 & 1022.77 $\pm$ 110.29 & 95.14 $\pm$ 33.64 & 11.69 $\pm$ 11.69 & 0 $\pm$ 0 & 11.69 $\pm$ 11.69 & 0 $\pm$ 0\\ 
600 & 3289.78 $\pm$ 123.99 & 2542.11 $\pm$ 108.99 & 485.99 $\pm$ 47.66 & 54.55 $\pm$ 15.77 & 18.73 $\pm$ 9.39 & 9.17 $\pm$ 6.49 & 0 $\pm$ 0\\ 
700 & 4655.21 $\pm$ 94.59 & 3855.64 $\pm$ 86.09 & 1237.8 $\pm$ 48.78 & 142.42 $\pm$ 17.03 & 28.55 $\pm$ 7.74 & 52.6 $\pm$ 10.75 & 7.69 $\pm$ 3.85\\ 
800 & 7506.31 $\pm$ 81.79 & 6020.56 $\pm$ 73.25 & 2150.64 $\pm$ 43.78 & 320.63 $\pm$ 17.02 & 67.63 $\pm$ 7.83 & 139.97 $\pm$ 11.23 & 47.57 $\pm$ 6.75\\ 
900 & 9732.89 $\pm$ 61.17 & 8400.91 $\pm$ 56.83 & 3574.63 $\pm$ 37.07 & 806.13 $\pm$ 17.76 & 188.71 $\pm$ 8.71 & 377.92 $\pm$ 12.12 & 127.7 $\pm$ 6.99\\ 
1000 & 7516.07 $\pm$ 37.72 & 7033.18 $\pm$ 36.49 & 4496.85 $\pm$ 29.18 & 1351.1 $\pm$ 16.2 & 303.71 $\pm$ 7.88 & 650.89 $\pm$ 11.19 & 234.2 $\pm$ 6.57\\ 
1100 & 4731.39 $\pm$ 21.54 & 4563.4 $\pm$ 21.15 & 3485.58 $\pm$ 18.49 & 1135.18 $\pm$ 10.7 & 251.77 $\pm$ 5.18 & 539.51 $\pm$ 7.36 & 215.39 $\pm$ 4.5\\ 
1200 & 2853.51 $\pm$ 12.23 & 2782.42 $\pm$ 12.07 & 2253.95 $\pm$ 10.87 & 786.92 $\pm$ 6.53 & 175.53 $\pm$ 3.21 & 366.01 $\pm$ 4.44 & 158.29 $\pm$ 2.8\\ 
1300 & 1700.83 $\pm$ 7.01 & 1668.05 $\pm$ 6.94 & 1362.91 $\pm$ 6.27 & 494.36 $\pm$ 3.84 & 107.0 $\pm$ 1.86 & 224.19 $\pm$ 2.58 & 107.55 $\pm$ 1.7\\ 
1400 & 1016.14 $\pm$ 4.03 & 999.98 $\pm$ 4.0 & 802.44 $\pm$ 3.59 & 296.46 $\pm$ 2.22 & 65.49 $\pm$ 1.1 & 133.86 $\pm$ 1.49 & 65.58 $\pm$ 0.99\\ 
1500 & 621.34 $\pm$ 2.41 & 613.46 $\pm$ 2.39 & 484.92 $\pm$ 2.13 & 179.29 $\pm$ 1.32 & 42.75 $\pm$ 0.68 & 79.32 $\pm$ 0.88 & 36.77 $\pm$ 0.56\\ 
1600 & 386.3 $\pm$ 1.46 & 382.44 $\pm$ 1.46 & 297.63 $\pm$ 1.28 & 109.99 $\pm$ 0.8 & 27.68 $\pm$ 0.42 & 49.3 $\pm$ 0.53 & 20.92 $\pm$ 0.32\\ 
1800 & 154.8 $\pm$ 0.58 & 153.5 $\pm$ 0.57 & 116.52 $\pm$ 0.5 & 42.24 $\pm$ 0.31 & 12.41 $\pm$ 0.18 & 18.2 $\pm$ 0.2 & 6.66 $\pm$ 0.11\\ 
2000 & 65.4 $\pm$ 0.24 & 65.02 $\pm$ 0.24 & 48.57 $\pm$ 0.2 & 16.89 $\pm$ 0.12 & 5.64 $\pm$ 0.076 & 7.01 $\pm$ 0.079 & 2.18 $\pm$ 0.041\\ 
2250 & 23.84 $\pm$ 0.085 & 23.73 $\pm$ 0.085 & 17.44 $\pm$ 0.073 & 5.57 $\pm$ 0.042 & 2.25 $\pm$ 0.028 & 2.11 $\pm$ 0.025 & 0.52 $\pm$ 0.012\\ 
2500 & 9.2 $\pm$ 0.032 & 9.17 $\pm$ 0.032 & 6.72 $\pm$ 0.028 & 1.9 $\pm$ 0.015 & 0.92 $\pm$ 0.011 & 0.64 $\pm$ 0.0086 & 0.11 $\pm$ 0.0034\\ 
2750 & 3.73 $\pm$ 0.013 & 3.73 $\pm$ 0.013 & 2.71 $\pm$ 0.011 & 0.63 $\pm$ 0.0054 & 0.37 $\pm$ 0.0042 & 0.17 $\pm$ 0.0027 & 0.021 $\pm$ 0.00093\\ 
3000 & 1.59 $\pm$ 0.0054 & 1.59 $\pm$ 0.0054 & 1.15 $\pm$ 0.0046 & 0.22 $\pm$ 0.0021 & 0.15 $\pm$ 0.0017 & 0.044 $\pm$ 0.0009 & 0.0038 $\pm$ 0.00025\\ 
& & & & & & &\\ 
\hline\hline 
\end{tabular} 
\end{footnotesize} 
\newline 

}
\end{center}
\label{boosted-eff-2HDM}
\end{table}


\paragraph{}
The selection efficiency at various stages for \Grav with $c=1.0$, \Grav with $c=2.0$, and Heavy Scalar signal samples of all mass points can be found in Tables~\ref{boosted-eff-RSG_c10}, ~\ref{boosted-eff-RSG_c20} and ~\ref{boosted-eff-2HDM}.
The signal production cross section is monotonically decreasing as a function of signal mass.


% \begin{figure*}
% \begin{center}
% \subfloat[]{\includegraphics[width=0.48\textwidth,angle=-90]{figures/boosted/SigEff/G_hh_c20_evtsel_Moriond_bkg_9_Efficiency_PreSel.pdf}
% \subfloat[]{\includegraphics[width=0.48\textwidth,angle=-90]{figures/boosted/SigEff/G_hh_c20_region_lst_Moriond_bkg_9_Efficiency_PreSel.pdf}
% \caption{(a) The selection acceptance times efficiency of the boosted analysis at each stage of the event selection as a function of the generated graviton mass for $\kMPl = 2$. The trigger efficiency is approximately 100\% after the requirement of two large-\R jets, so it is not shown. (b) The selection efficiency of the three signal region samples defined by $b$-tagging requirements, as a function of the generated graviton mass for $\kMPl = 2$.}
% \end{center}
% \end{figure*}


% \paragraph{}
% To improve the analysis, a quantity which defines the sensitivity of analysis is maximized. Technically, the optimal sensitivity is described as $\sqrt{2((S+B)\ln{1 + \frac{S}{B}} - S}$, see \href{https://www.pp.rhul.ac.uk/~cowan/stat/notes/SigCalcNote.pdf}{note}. This is usually considered at the $S << B$ limit and simplified as $\frac{S}{\sqrt{B}}$, when no knowledge of the model cross section is available, or $\frac{S}{\sqrt{S + B}}$, if the signal cross section is known. These parameterizations have limitations, particularly when the signal yield and the number of estimated background are both small. A better parametrization for low signal strength is $\frac{S}{\sqrt{1 + B}}$, where the extra $\sqrt{1 + B}$ accounts for poisson fluctuations. For a discussion of p-values, please see this \href{https://arxiv.org/pdf/hep-ex/0208005.pdf}{note}.

% \paragraph{}
% For this analysis, two methods are used: one is calculate the number of signal and backgrounds within $68\%$ of the signal \mhh mass window, the other is to implement the full signal and background predictions after smoothing and compare the asymptotic expected exclusion limits. Both methods yield comparable results.
