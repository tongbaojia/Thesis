%!TEX root = ../dissertation.tex
\chapter{Conclusion}
\label{conclusion}

\paragraph{}
Di-Higgs search has a short history, but will have a long future. This thesis presents a search for both resonant and non-resonant production of pairs of Standard Model Higgs bosons has been carried out in the dominant \fourb\ channel, using 27.5--36.1~\ifb\ of LHC $pp$ collision data at $\rts = 13$~\TeV\ collected by ATLAS in 2015 and 2016. The search sensitivity of this analysis exceeds that of the previous analysis of the $\rts = 13\,\TeV$ 2015 dataset~\cite{Aaboud:2016xco} for non-resonant signal and also across the entire mass range of 260-3000 GeV for the resonance search, with significantly improvement in the high mass resonance sensitivities. The resolved analysis has each $h \to b\bar{b}$ reconstructed as two separate $b$-tagged jets, and the boosted analysis has each $h \to b\bar{b}$ reconstructed as a single large-radius jet associated with at least one small-radius $b$-tagged track-jet. The estimated background consists mainly of multi-jet and \ttbar\ events.

\paragraph{}
No significant excess is observed in the data. The largest deviation from the background-only hypothesis is observed for narrow signal models at a mass of 280~\GeV\ in the resolved analysis, with a global significance of 2.3$\sigma$. This excess could be a trigger-turn on combined with kinematic selection effect and might not last with more data. Upper limits on the production cross section times branching ratio to the \fourb\ final state are set for a narrow-width scalar and for spin-2 resonances. The 95\% CL upper limit on the non-resonant production is 147~fb, which corresponds to 13.0 times the SM expectation. %The bulk RS model with $\kMPl = 1$ is excluded for masses between 313 and 1362~\GeV, and the bulk RS model with $\kMPl = 2$ is excluded for masses below 1744~\GeV.

\paragraph{}
Future improvement with the rest of $\rts = 13$~\TeV\ Run II could come from $b$-tagging, especially in the high \pt region. Advanced trigger technologies and selections will increase the data rate, and better jet energy and mass resolution will increase the purity in selection. With the larger dataset and improvements in physics performance, it is possible to reach twice as the current sensitivity of resonance searches. For non-resonance searches, an order of 10 times the SM expectation is more sensible at the end of Run II.


\paragraph{}
For longer term perspectives, di-Higgs measurements will continue to be one of the most important analysis that help constraining our understanding of physics beyond the Standard Model. Given the current status, it is possible that in fifteen to twenty years there will be no new discovery, and the experiments at the LHC will be able to constrain the Higgs self-coupling within unity. 

\paragraph{}
As humans, we have a limited life. However, physics, as well as the understanding of the universe, is an endless journey. I sincerely hope that my biased prediction of the future of Higgs physics will be wrong, but nevertheless I am deeply honored to be a small part of this odyssey towards Veritas.