%!TEX root = ../dissertation.tex
\begin{savequote}[75mm]
It is a far, far better thing that I do, than I have ever done; it is a far, far better rest that I go to than I have ever known.
\qauthor{Charles Dickens}
\end{savequote}

\chapter{Conclusion}
\label{conclusion}

%%% Why you are doing; what are you doing in the beginning
%%% conclusioins is the result, what this means for now and for the future
%%% Simple short sentences;

\paragraph{}
The searches for di-Higgs have a short history, but they will have a long future. 
This thesis presents a search for both resonant and non-resonant production of pairs of Standard Model Higgs bosons in the dominant \bbbb~ channel, using $36.1$ \ifb\ of LHC $pp$ collision data at $\rts = 13$~\TeV\ collected by ATLAS in 2015 and 2016. 
The search sensitivity of this analysis exceeds that of the previous analysis of the $\rts = 13\,\TeV$ 2015 dataset~\cite{EXOT-2015-11} for non-resonant signal and also across the entire mass range of $260$ to $3000$ \GeV~ for the resonance search, with significant improvement in the high mass resonance region. 
The resolved analysis reconstructs each $h \to b\bar{b}$ decay as two separate $b$-tagged jets, while the boosted analysis reconstructs each $h \to b\bar{b}$ decay as a single large-radius jet associated with at least one small-radius $b$-tagged track jet.
The estimated background consists mainly of multi-jet and \ttbar\ events.

\paragraph{}
No significant excess is observed in the data. 
%The largest deviation from the background-only hypothesis is observed for narrow signal models at a mass of 280~\GeV\ in the resolved analysis, with a global significance of 2.3$\sigma$. This excess could be a trigger-turn on combined with kinematic selection effect and might not last with more data. 
Upper limits on the production cross-section times branching ratio to the \bbbb~ final state are set for a narrow-width spin-0 scalar and for wider spin-2 resonances. 
The bulk RS model with $\kMPl = 1$ is excluded for masses between $313$ and $1362$ \GeV, and the bulk RS model with $\kMPl = 2$ is excluded for masses below $1744$ \GeV.
The $95\%$ CL upper limit on the non-resonant production is $147$~fb, which corresponds to $13.0$ times the SM expectation.
This result confirms the great success of the Standard Model.
In light of these results, the Higgs potential shape at the \TeV~ energy scale is consistent with the SM prediction.
Without any significant excess, the phase space for beyond the Standard Model physics is further constrained.

\paragraph{}
Increased sensitivity with future Run 2 datasets could come through efficiency improvements in $b$-tagging for the high \pt~ jets.
Other improvements involve advanced trigger technologies, which would increase the number of signal events recorded, as well as improvements in jet energy and mass resolution, which would increase the purity of the signal selection.
These improvements, combined with a larger dataset, may double the current resonance search sensitivity. 
For the non-resonance search, a combined sensitivity of $|\frac{\rm\lambda}{\rm\lambda_{SM}}| \sim 10$ is possible at the end of Run 2 in 2020.

\paragraph{}
Di-Higgs searches and measurements will continue to be one of the most important analyses. 
These measurements can show hints of the physics beyond the Standard Model. 
After the 2030 to 2040 run, the High Luminosity LHC will be able to constrain $|\frac{\rm\lambda}{\rm\lambda_{SM}}| \sim 1$ with all the different channels from both ATLAS and CMS combined.
Future electron-electron colliders can produce di-Higgs events from ZH channel and could further constrain $|\frac{\rm\lambda}{\rm\lambda_{SM}}| \sim 0.3$.
This requires a combined advance in accelerator, detector, computation and theory~\cite{Richter:2014pga, Hawking:2018tcn}.

\paragraph{}
%Looking even more to the future for high energy experiments, all aspects--accelerator, detector, computation and theory--must advance together to answer the questions the Standard Model cannot answer, or find questions the Standard Model has failed to ask~\cite{Richter:2014pga, Hawking:2018tcn}.
Life is short, and the understanding of the universe is an endless journey. 
I am deeply honored to be a small part of this odyssey by working on the boosted \pptofourb~ analysis.
