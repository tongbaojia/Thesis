\subsection{Unblinded Results}
%%\paragraph{}
The unblinded results are summarised in Table ~\ref{tab:sr-summary}. 

%%\paragraph{}
For reader's interest, we integrate the background prediction from a certain mass point on and compare that with our unblinded observations. These are listed in Table ~\ref{tab:sr-region-4b}, ~\ref{tab:sr-region-3b}, ~\ref{tab:sr-region-2b}. The unscaled $2bs$/$3b$/$4b$s dijet mass distributions are shown in Figures~\ref{fig:boosted-2b-signal-l}, \ref{fig:boosted-3b-signal-l}, \ref{fig:boosted-4b-signal-l}. No significant excess of number of events or in the dijet mass distribution is observed.

%%\paragraph{}
For the scaled dijet mass, the integral values are listed in Table ~\ref{tab:sr-region-4b-pole}, ~\ref{tab:sr-region-3b-pole}, ~\ref{tab:sr-region-2b-pole}. The scaled $2bs$/$3b$/$4b$s dijet mass distributions are shown in Figures~\ref{fig:boosted-2b-signal-pole}, \ref{fig:boosted-3b-signal-pole}, \ref{fig:boosted-4b-signal-pole}. No significant excess of number of events or in the dijet mass distribution is observed as well.


\begin{table}[htbp!]
\scriptsize
\begin{center}
\begin{footnotesize} 
\begin{tabular}{c|c|c|c} 
Sample & FourTag & ThreeTag & TwoTag split \\ 
\hline\hline 
qcd & 32.92 $\pm$ 7.07 & 702.16 $\pm$ 63.12 & 3393.81 $\pm$ 148.78\\ 
ttbar & 1.68 $\pm$ 1.43 & 79.41 $\pm$ 33.12 & 859.03 $\pm$ 107.86\\ 
totalbkg & 34.6 $\pm$ 6.28 & 781.56 $\pm$ 52.42 & 4252.83 $\pm$ 125.73\\ 
\hline 
Data & 31.0 $\pm$ 5.57 & 801.0 $\pm$ 28.3 & 4376.0 $\pm$ 66.15\\ 
\hline\hline 
\end{tabular} 
\end{footnotesize} 
\newline 

\caption{Unblinded Signal Region predictions and results. All systemtic uncertainties included for backgrounds. For Data, the statistical uncertainty is shown.}
\label{tab:sr-summary}
\end{center}
\end{table}

\begin{table}[htbp!]
\scriptsize
\begin{center}
\begin{footnotesize} 
\begin{tabular}{c|c|c|c|c|c} 
Mass Range & >1000 & >1500 & >2000 & >2500 & >3000 \\ 
\hline\hline 
totalbkg & 23.09 $\pm$ 1.59 & 1.94 $\pm$ 0.15 & 0.26 $\pm$ 0.072 & 0.061 $\pm$ 0.058 & 0.021 $\pm$ 0.047\\ 
data & 21.0 $\pm$ 4.58 & 3.0 $\pm$ 1.73 &  -  &  -  &  - \\ 
\hline\hline 
\end{tabular} 
\end{footnotesize} 
\newline 

\caption{$4b$ unblinded Signal Region predictions and results. All systemtic uncertainties included for backgrounds. For Data, the statistical uncertainty is shown. Mass range is broken into greater than 1 TeV, 1.5 TeV, 2 TeV, 2.5 TeV, and 3 TeV intevals.}
\label{tab:sr-region-4b}
\end{center}
\end{table}

\begin{table}[htbp!]
\scriptsize
\begin{center}
\begin{footnotesize} 
\begin{tabular}{c|c|c|c|c|c} 
Mass Range & >1000 & >1500 & >2000 & >2500 & >3000 \\ 
\hline\hline 
totalbkg & 495.92 $\pm$ 12.34 & 51.72 $\pm$ 2.46 & 10.42 $\pm$ 0.95 & 4.07 $\pm$ 0.85 & 2.21 $\pm$ 0.79\\ 
data & 499.0 $\pm$ 22.34 & 42.0 $\pm$ 6.48 & 3.0 $\pm$ 1.73 & 1.0 $\pm$ 1.0 &  - \\ 
\hline\hline 
\end{tabular} 
\end{footnotesize} 
\newline 

\caption{$3b$ unblinded Signal Region predictions and results. All systemtic uncertainties included for backgrounds. For Data, the statistical uncertainty is shown. Mass range is broken into greater than 1 TeV, 1.5 TeV, 2 TeV, 2.5 TeV, and 3 TeV intevals.}
\label{tab:sr-region-3b}
\end{center}
\end{table}

\begin{table}[htbp!]
\scriptsize
\begin{center}
\begin{footnotesize} 
\begin{tabular}{c|c|c|c|c|c} 
Mass Range & >1000 & >1500 & >2000 & >2500 & >3000 \\ 
\hline\hline 
totalbkg & 2688.71 $\pm$ 34.09 & 288.51 $\pm$ 4.96 & 42.19 $\pm$ 2.13 & 8.85 $\pm$ 1.55 & 2.72 $\pm$ 1.09\\ 
data & 2755.0 $\pm$ 52.49 & 287.0 $\pm$ 16.94 & 38.0 $\pm$ 6.16 & 4.0 $\pm$ 2.0 & 1.0 $\pm$ 1.0\\ 
\hline\hline 
\end{tabular} 
\end{footnotesize} 
\newline 

\caption{$2bs$ unblinded Signal Region predictions and results. All systemtic uncertainties included for backgrounds. For Data, the statistical uncertainty is shown. Mass range is broken into greater than 1 TeV, 1.5 TeV, 2 TeV, 2.5 TeV, and 3 TeV intevals.}
\label{tab:sr-region-2b}
\end{center}
\end{table}


\begin{table}[htbp!]
\scriptsize
\begin{center}
\begin{footnotesize} 
\begin{tabular}{c|c|c|c|c|c} 
Mass Range & >1000 & >1500 & >2000 & >2500 & >3000 \\ 
\hline\hline 
totalbkg & 24.64 $\pm$ 1.84 & 2.84 $\pm$ 0.22 & 0.25 $\pm$ 0.044 & 0.02 $\pm$ 0.011 & 0.0014 $\pm$ 0.0026\\ 
data & 22.0 $\pm$ 4.69 & 4.0 $\pm$ 2.0 & 1.0 $\pm$ 1.0 &  -  &  - \\ 
\hline\hline 
\end{tabular} 
\end{footnotesize} 
\newline 

\caption{$4b$ unblinded Scaled dijet mass Region predictions and results. All systemtic uncertainties included for backgrounds. For Data, the statistical uncertainty is shown. Mass range is broken into greater than 1 TeV, 1.5 TeV, 2 TeV, 2.5 TeV, and 3 TeV intevals.}
\label{tab:sr-region-4b-pole}
\end{center}
\end{table}

\begin{table}[htbp!]
\scriptsize
\begin{center}
\begin{footnotesize} 
\begin{tabular}{c|c|c|c|c|c} 
Mass Range & >1000 & >1500 & >2000 & >2500 & >3000 \\ 
\hline\hline 
totalbkg & 559.38 $\pm$ 14.06 & 69.27 $\pm$ 3.22 & 13.35 $\pm$ 1.14 & 4.37 $\pm$ 0.96 & 2.0 $\pm$ 0.87\\ 
data & 570.0 $\pm$ 23.87 & 59.0 $\pm$ 7.68 & 4.0 $\pm$ 2.0 & 1.0 $\pm$ 1.0 &  - \\ 
\hline\hline 
\end{tabular} 
\end{footnotesize} 
\newline 

\caption{$3b$ unblinded Scaled dijet mass Signal Region predictions and results. All systemtic uncertainties included for backgrounds. For Data, the statistical uncertainty is shown. Mass range is broken into greater than 1 TeV, 1.5 TeV, 2 TeV, 2.5 TeV, and 3 TeV intevals.}
\label{tab:sr-region-3b-pole}
\end{center}
\end{table}

\begin{table}[htbp!]
\scriptsize
\begin{center}
\begin{footnotesize} 
\begin{tabular}{c|c|c|c|c|c} 
Mass Range & >1000 & >1500 & >2000 & >2500 & >3000 \\ 
\hline\hline 
totalbkg & 2998.69 $\pm$ 40.31 & 377.8 $\pm$ 6.39 & 57.47 $\pm$ 2.88 & 11.78 $\pm$ 1.95 & 3.39 $\pm$ 1.27\\ 
data & 3078.0 $\pm$ 55.48 & 379.0 $\pm$ 19.47 & 47.0 $\pm$ 6.86 & 6.0 $\pm$ 2.45 & 2.0 $\pm$ 1.41\\ 
\hline\hline 
\end{tabular} 
\end{footnotesize} 
\newline 

\caption{$2bs$ unblinded Scaled dijet mass Signal Region predictions and results. All systemtic uncertainties included for backgrounds. For Data, the statistical uncertainty is shown. Mass range is broken into greater than 1 TeV, 1.5 TeV, 2 TeV, 2.5 TeV, and 3 TeV intevals.}
\label{tab:sr-region-2b-pole}
\end{center}
\end{table}
\clearpage

%%%%%%%%%%%%%%%%%%%%%%%%%%%plots%%%%%%%%%%%%%
\begin{figure*}[htbp!]
\begin{center}
\includegraphics[width=0.31\textwidth,angle=-90]{figures/boosted/Signal_Syst/Moriond_bkg_9_FourTag_Signal_mHH_l.pdf}
\includegraphics[width=0.31\textwidth,angle=-90]{figures/boosted/Signal_Syst/Moriond_bkg_9_FourTag_Signal_mHH_l_1.pdf}
  \caption{Unscaled dijet mass distribution in the $4b$ Signal Region after unblinding. The left plot is on linear scale and the right plot is on log scale. Stat uncertainty and systematic ucnertainties are shown on the plot.}
  \label{fig:boosted-4b-signal-l}
\end{center}
\end{figure*}

\begin{figure*}[htbp!]
\begin{center}
\includegraphics[width=0.31\textwidth,angle=-90]{figures/boosted/Signal_Syst/Moriond_bkg_9_ThreeTag_Signal_mHH_l.pdf}
\includegraphics[width=0.31\textwidth,angle=-90]{figures/boosted/Signal_Syst/Moriond_bkg_9_ThreeTag_Signal_mHH_l_1.pdf}  
  \caption{Unscaled dijet mass distribution in the $3b$ Signal Region after unblinding. The left plot is on linear scale and the right plot is on log scale. Stat uncertainty and systematic ucnertainties are shown on the plot.}
  \label{fig:boosted-3b-signal-l}
\end{center}
\end{figure*}

\begin{figure*}[htbp!]
\begin{center}
\includegraphics[width=0.31\textwidth,angle=-90]{figures/boosted/Signal_Syst/Moriond_bkg_9_TwoTag_split_Signal_mHH_l.pdf}
\includegraphics[width=0.31\textwidth,angle=-90]{figures/boosted/Signal_Syst/Moriond_bkg_9_TwoTag_split_Signal_mHH_l_1.pdf}  
  \caption{Unscaled dijet mass distribution in the $2bs$ Signal Region after unblinding. The left plot is on linear scale and the right plot is on log scale. Stat uncertainty and systematic ucnertainties are shown on the plot.}
  \label{fig:boosted-2b-signal-l}
\end{center}
\end{figure*}

\begin{figure*}[htbp!]
\begin{center}
\includegraphics[width=0.31\textwidth,angle=-90]{figures/boosted/Signal_Syst/Moriond_bkg_9_FourTag_Signal_mHH_pole.pdf}
\includegraphics[width=0.31\textwidth,angle=-90]{figures/boosted/Signal_Syst/Moriond_bkg_9_FourTag_Signal_mHH_pole_1.pdf}
  \caption{Scaled dijet mass distribution in the $4b$ Signal Region after unblinding. The left plot is on linear scale and the right plot is on log scale. Stat uncertainty and systematic ucnertainties are shown on the plot.}
  \label{fig:boosted-4b-signal-pole}
\end{center}
\end{figure*}

\begin{figure*}[htbp!]
\begin{center}
\includegraphics[width=0.31\textwidth,angle=-90]{figures/boosted/Signal_Syst/Moriond_bkg_9_ThreeTag_Signal_mHH_pole.pdf}
\includegraphics[width=0.31\textwidth,angle=-90]{figures/boosted/Signal_Syst/Moriond_bkg_9_ThreeTag_Signal_mHH_pole_1.pdf}  
  \caption{Scaled dijet mass distribution in the $3b$ Signal Region after unblinding. The left plot is on linear scale and the right plot is on log scale. Stat uncertainty and systematic ucnertainties are shown on the plot.}
  \label{fig:boosted-3b-signal-pole}
\end{center}
\end{figure*}

\begin{figure*}[htbp!]
\begin{center}
\includegraphics[width=0.31\textwidth,angle=-90]{figures/boosted/Signal_Syst/Moriond_bkg_9_TwoTag_split_Signal_mHH_pole.pdf}
\includegraphics[width=0.31\textwidth,angle=-90]{figures/boosted/Signal_Syst/Moriond_bkg_9_TwoTag_split_Signal_mHH_pole_1.pdf}  
  \caption{Scaled dijet mass distribution in the $2bs$ Signal Region after unblinding. The left plot is on linear scale and the right plot is on log scale. Stat uncertainty and systematic ucnertainties are shown on the plot.}
  \label{fig:boosted-2b-signal-pole}
\end{center}
\end{figure*}

\clearpage
\subsection{Kinematic Distributions}

%%\paragraph{}
This section shows unblinded comparisons of data with the prediction of QCD multi-jets and \ttbar in the signal region (SR). Plots shown are with stat uncertainty only.

%%\paragraph{}
Figures~\ref{fig:boosted-4b-signal-ak10-lead}, \ref{fig:boosted-4b-signal-ak10-subl}, \ref{fig:boosted-4b-signal-ak2}, and \ref{fig:boosted-4b-signal-ak10-system} show predictions of various kinematics of the large-$R$ jets and their associated track jets in the $4b$ selection.

%%\paragraph{}
Figures~\ref{fig:boosted-3b-signal-ak10-lead}, \ref{fig:boosted-3b-signal-ak10-subl}, \ref{fig:boosted-3b-signal-ak2}, and \ref{fig:boosted-3b-signal-ak10-system} show predictions of various kinematics of the large-$R$ jets and their associated track jets in the $3b$ selection.

%%\paragraph{}
Figures~\ref{fig:boosted-2bs-signal-ak10-lead}, \ref{fig:boosted-2bs-signal-ak10-subl}, \ref{fig:boosted-2bs-signal-ak2}, and \ref{fig:boosted-2bs-signal-ak10-system} show predictions of various kinematics of the large-$R$ jets and their associated track jets in the $2b$ selection.

\begin{figure*}[htbp!]
\begin{center}
\includegraphics[width=0.31\textwidth,angle=-90]{figures/boosted/Signal/b77_FourTag_Signal_leadHCand_Pt_m.pdf}
\includegraphics[width=0.31\textwidth,angle=-90]{figures/boosted/Signal/b77_FourTag_Signal_leadHCand_Mass_s.pdf}\\
\includegraphics[width=0.31\textwidth,angle=-90]{figures/boosted/Signal/b77_FourTag_Signal_leadHCand_Eta.pdf}
\includegraphics[width=0.31\textwidth,angle=-90]{figures/boosted/Signal/b77_FourTag_Signal_leadHCand_Phi.pdf}
  \caption{Kinematics of the lead large-$R$ jet in data and prediction in the signal region after requiring 4 $b$-tags. }
  \label{fig:boosted-4b-signal-ak10-lead}
\end{center}
\end{figure*}

\begin{figure*}[htbp!]
\begin{center}
\includegraphics[width=0.31\textwidth,angle=-90]{figures/boosted/Signal/b77_FourTag_Signal_sublHCand_Pt_m.pdf}
\includegraphics[width=0.31\textwidth,angle=-90]{figures/boosted/Signal/b77_FourTag_Signal_sublHCand_Mass_s.pdf}\\
\includegraphics[width=0.31\textwidth,angle=-90]{figures/boosted/Signal/b77_FourTag_Signal_sublHCand_Eta.pdf}
\includegraphics[width=0.31\textwidth,angle=-90]{figures/boosted/Signal/b77_FourTag_Signal_sublHCand_Phi.pdf}
  \caption{Kinematics of the sub-lead large-$R$ jet in data and prediction in the signal region after requiring 4 $b$-tags. }
  \label{fig:boosted-4b-signal-ak10-subl}
\end{center}
\end{figure*}

\begin{figure*}[htbp!]
\begin{center}
\includegraphics[width=0.31\textwidth,angle=-90]{figures/boosted/Signal/b77_FourTag_Signal_leadHCand_trk0_Pt.pdf}
\includegraphics[width=0.31\textwidth,angle=-90]{figures/boosted/Signal/b77_FourTag_Signal_leadHCand_trk1_Pt.pdf}\\
\includegraphics[width=0.31\textwidth,angle=-90]{figures/boosted/Signal/b77_FourTag_Signal_sublHCand_trk0_Pt.pdf}
\includegraphics[width=0.31\textwidth,angle=-90]{figures/boosted/Signal/b77_FourTag_Signal_sublHCand_trk1_Pt.pdf}\\
\includegraphics[width=0.31\textwidth,angle=-90]{figures/boosted/Signal/b77_FourTag_Signal_leadHCand_trk_dr.pdf}
\includegraphics[width=0.31\textwidth,angle=-90]{figures/boosted/Signal/b77_FourTag_Signal_sublHCand_trk_dr.pdf}
  \caption{First two rows show the kinematics of the lead (left) and sub-lead (right) small-$R$ track jets associated to the lead (first-row) and sub-lead (second-row) large-$R$ jet in data and prediction in the signal region after requiring 4 $b$-tags. Third row shows the $\Delta R$ between two leading small-$R$ track-jets associated to the leading (left) and sub-leading (right) large-$R$ jet.  }
  \label{fig:boosted-4b-signal-ak2}
\end{center}
\end{figure*}


\begin{figure*}[htbp!]
\begin{center}
\includegraphics[width=0.31\textwidth,angle=-90]{figures/boosted/Signal/b77_FourTag_Signal_mHH_l_1.pdf}
\includegraphics[width=0.31\textwidth,angle=-90]{figures/boosted/Signal/b77_FourTag_Signal_hCandDr.pdf}\\
\includegraphics[width=0.31\textwidth,angle=-90]{figures/boosted/Signal/b77_FourTag_Signal_hCandDeta.pdf}
\includegraphics[width=0.31\textwidth,angle=-90]{figures/boosted/Signal/b77_FourTag_Signal_hCandDphi.pdf}
  \caption{Kinematics of the large-$R$ jet system in data and prediction in the signal region after requiring 4 $b$-tags.  }
  \label{fig:boosted-4b-signal-ak10-system}
\end{center}
\end{figure*}

\clearpage

\begin{figure*}[htbp!]
\begin{center}
\includegraphics[width=0.31\textwidth,angle=-90]{figures/boosted/Signal/b77_ThreeTag_Signal_leadHCand_Pt_m.pdf}
\includegraphics[width=0.31\textwidth,angle=-90]{figures/boosted/Signal/b77_ThreeTag_Signal_leadHCand_Mass_s.pdf}\\
\includegraphics[width=0.31\textwidth,angle=-90]{figures/boosted/Signal/b77_ThreeTag_Signal_leadHCand_Eta.pdf}
\includegraphics[width=0.31\textwidth,angle=-90]{figures/boosted/Signal/b77_ThreeTag_Signal_leadHCand_Phi.pdf}
  \caption{Kinematics of the lead large-$R$ jet in data and prediction in the signal region after requiring 3 $b$-tags. }
  \label{fig:boosted-3b-signal-ak10-lead}
\end{center}
\end{figure*}

\begin{figure*}[htbp!]
\begin{center}
\includegraphics[width=0.31\textwidth,angle=-90]{figures/boosted/Signal/b77_ThreeTag_Signal_sublHCand_Pt_m.pdf}
\includegraphics[width=0.31\textwidth,angle=-90]{figures/boosted/Signal/b77_ThreeTag_Signal_sublHCand_Mass_s.pdf}\\
\includegraphics[width=0.31\textwidth,angle=-90]{figures/boosted/Signal/b77_ThreeTag_Signal_sublHCand_Eta.pdf}
\includegraphics[width=0.31\textwidth,angle=-90]{figures/boosted/Signal/b77_ThreeTag_Signal_sublHCand_Phi.pdf}
  \caption{Kinematics of the sub-lead large-$R$ jet in data and prediction in the signal region after requiring 3 $b$-tags. }
  \label{fig:boosted-3b-signal-ak10-subl}
\end{center}
\end{figure*}

\begin{figure*}[htbp!]
\begin{center}
\includegraphics[width=0.31\textwidth,angle=-90]{figures/boosted/Signal/b77_ThreeTag_Signal_leadHCand_trk0_Pt.pdf}
\includegraphics[width=0.31\textwidth,angle=-90]{figures/boosted/Signal/b77_ThreeTag_Signal_leadHCand_trk1_Pt.pdf}\\
\includegraphics[width=0.31\textwidth,angle=-90]{figures/boosted/Signal/b77_ThreeTag_Signal_sublHCand_trk0_Pt.pdf}
\includegraphics[width=0.31\textwidth,angle=-90]{figures/boosted/Signal/b77_ThreeTag_Signal_sublHCand_trk1_Pt.pdf}\\
\includegraphics[width=0.31\textwidth,angle=-90]{figures/boosted/Signal/b77_ThreeTag_Signal_leadHCand_trk_dr.pdf}
\includegraphics[width=0.31\textwidth,angle=-90]{figures/boosted/Signal/b77_ThreeTag_Signal_sublHCand_trk_dr.pdf}
  \caption{First two rows show the kinematics of the lead (left) and sub-lead (right) small-$R$ track jets associated to the lead (first-row) and sub-lead (second-row) large-$R$ jet in data and prediction in the signal region after requiring 3 $b$-tags. Third row shows the $\Delta R$ between two leading small-$R$ track-jets associated to the leading (left) and sub-leading (right) large-$R$ jet.  }
  \label{fig:boosted-3b-signal-ak2}
\end{center}
\end{figure*}


\begin{figure*}[htbp!]
\begin{center}
\includegraphics[width=0.31\textwidth,angle=-90]{figures/boosted/Signal/b77_ThreeTag_Signal_mHH_l_1.pdf}
\includegraphics[width=0.31\textwidth,angle=-90]{figures/boosted/Signal/b77_ThreeTag_Signal_hCandDr.pdf}\\
\includegraphics[width=0.31\textwidth,angle=-90]{figures/boosted/Signal/b77_ThreeTag_Signal_hCandDeta.pdf}
\includegraphics[width=0.31\textwidth,angle=-90]{figures/boosted/Signal/b77_ThreeTag_Signal_hCandDphi.pdf}
  \caption{Kinematics of the large-$R$ jet system in data and prediction in the signal region after requiring 3 $b$-tags.  }
  \label{fig:boosted-3b-signal-ak10-system}
\end{center}
\end{figure*}

\clearpage

\begin{figure*}[htbp!]
\begin{center}
\includegraphics[width=0.31\textwidth,angle=-90]{figures/boosted/Signal/b77_TwoTag_split_Signal_leadHCand_Pt_m.pdf}
\includegraphics[width=0.31\textwidth,angle=-90]{figures/boosted/Signal/b77_TwoTag_split_Signal_leadHCand_Mass_s.pdf}\\
\includegraphics[width=0.31\textwidth,angle=-90]{figures/boosted/Signal/b77_TwoTag_split_Signal_leadHCand_Eta.pdf}
\includegraphics[width=0.31\textwidth,angle=-90]{figures/boosted/Signal/b77_TwoTag_split_Signal_leadHCand_Phi.pdf}
  \caption{Kinematics of the lead large-$R$ jet in data and prediction in the signal region after requiring 2 $b$-tags split. }
  \label{fig:boosted-2bs-signal-ak10-lead}
\end{center}
\end{figure*}

\begin{figure*}[htbp!]
\begin{center}
\includegraphics[width=0.31\textwidth,angle=-90]{figures/boosted/Signal/b77_TwoTag_split_Signal_sublHCand_Pt_m.pdf}
\includegraphics[width=0.31\textwidth,angle=-90]{figures/boosted/Signal/b77_TwoTag_split_Signal_sublHCand_Mass_s.pdf}\\
\includegraphics[width=0.31\textwidth,angle=-90]{figures/boosted/Signal/b77_TwoTag_split_Signal_sublHCand_Eta.pdf}
\includegraphics[width=0.31\textwidth,angle=-90]{figures/boosted/Signal/b77_TwoTag_split_Signal_sublHCand_Phi.pdf}
  \caption{Kinematics of the sub-lead large-$R$ jet in data and prediction in the signal region after requiring 2 $b$-tags split. }
  \label{fig:boosted-2bs-signal-ak10-subl}
\end{center}
\end{figure*}

\begin{figure*}[htbp!]
\begin{center}
\includegraphics[width=0.31\textwidth,angle=-90]{figures/boosted/Signal/b77_TwoTag_split_Signal_leadHCand_trk0_Pt.pdf}
\includegraphics[width=0.31\textwidth,angle=-90]{figures/boosted/Signal/b77_TwoTag_split_Signal_leadHCand_trk1_Pt.pdf}\\
\includegraphics[width=0.31\textwidth,angle=-90]{figures/boosted/Signal/b77_TwoTag_split_Signal_sublHCand_trk0_Pt.pdf}
\includegraphics[width=0.31\textwidth,angle=-90]{figures/boosted/Signal/b77_TwoTag_split_Signal_sublHCand_trk1_Pt.pdf}\\
\includegraphics[width=0.31\textwidth,angle=-90]{figures/boosted/Signal/b77_TwoTag_split_Signal_leadHCand_trk_dr.pdf}
\includegraphics[width=0.31\textwidth,angle=-90]{figures/boosted/Signal/b77_TwoTag_split_Signal_sublHCand_trk_dr.pdf}
  \caption{First two rows show the kinematics of the lead (left) and sub-lead (right) small-$R$ track jets associated to the lead (first-row) and sub-lead (second-row) large-$R$ jet in data and prediction in the signal region after requiring 2 $b$-tags split. Third row shows the $\Delta R$ between two leading small-$R$ track-jets associated to the leading (left) and sub-leading (right) large-$R$ jet.  }
  \label{fig:boosted-2bs-signal-ak2}
\end{center}
\end{figure*}


\begin{figure*}[htbp!]
\begin{center}
\includegraphics[width=0.31\textwidth,angle=-90]{figures/boosted/Signal/b77_TwoTag_split_Signal_mHH_l_1.pdf}
\includegraphics[width=0.31\textwidth,angle=-90]{figures/boosted/Signal/b77_TwoTag_split_Signal_hCandDr.pdf}\\
\includegraphics[width=0.31\textwidth,angle=-90]{figures/boosted/Signal/b77_TwoTag_split_Signal_hCandDeta.pdf}
\includegraphics[width=0.31\textwidth,angle=-90]{figures/boosted/Signal/b77_TwoTag_split_Signal_hCandDphi.pdf}
  \caption{Kinematics of the large-$R$ jet system in data and prediction in the signal region after requiring 2 $b$-tags split.  }
  \label{fig:boosted-2bs-signal-ak10-system}
\end{center}
\end{figure*}


\clearpage

\subsection{Test of the background model hypothesis}

Here the results are displayed for the mass range that includes the boosted categories (ie. 800 GeV and above).  In the range 800-1400 GeV the boosted categories are combined with the resolved categories. The full mass range results are collected in the resolved note.

A search for statistically significant deviation from the background model hypothesis is performed following the procedure described in Sec.\,\ref{sec-search-procedure}, computing the local $p_0$ value using the asymptotic approximation.

The background model is found to describe the data and no significant excess is observed. The smallest local $p_0=0.175$ (1$\sigma$) is found at 1100 GeV when fitting with the narrow scalar model. The local $p_0$ values for the three signal models as a function of the resonance mass are shown in Fig.~\ref{fig:localp0}.

\begin{figure*}[htbp!]
\begin{center}
\includegraphics[width=0.32\textwidth,angle=-90]{figures/boosted/results/p0_s_allmasses_boosted.pdf}
\includegraphics[width=0.32\textwidth,angle=-90]{figures/boosted/results/p0_g10_allmasses_boosted.pdf}
\includegraphics[width=0.32\textwidth,angle=-90]{figures/boosted/results/p0_g20_allmasses_boosted.pdf} 
\caption{Local $p_0$ of the (a) scalar, (b) c=1 Graviton and (c) c=2 Graviton.}
\label{fig:localp0}
\end{center}
\end{figure*}

\subsection{Observed Limits}
\label{sec:observedlimits}

The observed limit for the narrow scalar is shown in Fig.~\ref{fig:limit_scalar}. The stat-only limit is also shown. The impact of systematic uncertainties is small. The observed limits for the Graviton models is shown in Fig.~\ref{fig:limit_g10} for c=1 and in Fig.~\ref{fig:limit_g20} for c=2. These limits do not contain any of the resolved categories.

\begin{figure*}
\begin{center}
\includegraphics[width=0.75\textwidth,angle=-90]{figures/boosted/results/limit_boosted_boosted_okt18_s.pdf}
\caption{The expected and observed 95\% C.L. upper exclusion limits for the boosted $4b$ analysis calculated including all systematic uncertainties for the narrow scalar model. The dot-dashed line shows the expected limit when only statistical uncertainties are included. The limits are derived within the asymptotic approximation.}
\label{fig:limit_scalar}
\end{center}
\end{figure*}

\begin{figure*}
\begin{center}
\includegraphics[width=0.75\textwidth,angle=-90]{figures/boosted/results/limit_boosted_boosted_okt18_g10.pdf}
\caption{The expected and observed 95\% C.L. upper exclusion limits for the boosted $4b$ analysis calculated including all systematic uncertainties for the c=1.0 Graviton. The dot-dashed line shows the expected limit when only statistical uncertainties are included. The limits are derived within the asymptotic approximation.}
\label{fig:limit_scalar}
\end{center}
\end{figure*}
\begin{figure*}

\begin{center}
\includegraphics[width=0.75\textwidth,angle=-90]{figures/boosted/results/limit_boosted_boosted_okt18_g20.pdf}
\caption{The expected and observed 95\% C.L. upper exclusion limits for the boosted $4b$ analysis calculated including all systematic uncertainties for the c=2.0 Graviton. The dot-dashed line shows the expected limit when only statistical uncertainties are included. The limits are derived within the asymptotic approximation.}
\label{fig:limit_scalar}
\end{center}
\end{figure*}

Figure~\ref{fig:nuisanceParams} shows the pulls of the systematic uncertainty nuisance parameters and their correlations for the 2000 GeV mass point. One nuisance parameter (QCD\_ShapeCRHigh) in both the $2b$ and $3b$ samples shows a significant constraint coming from the signal region data. This nuisance parameter corresponds to the shape uncertainty on the QCD background derived from the $2b$ and $3b$ control regions, as explained in Section \ref{unc-shape-qcd-in-sr}. The prior probability distributions for this nuisance parameter is very broad, with the relative uncertainty on the background prediction reaching 15000\% at high $m_{hh}$. This is because there is very little data in the control region at high mass to constrain the uncertainty. In the signal region however, the two events found suffice to constrain it: very tightly in comparison to the extremely loose prior constraint.

\begin{figure*}[htbp!]
\begin{center}
\includegraphics[width=0.39\textwidth,angle=-90]{figures/boosted/results/pulls_2000_data.pdf}
\includegraphics[width=0.59\textwidth,angle=-90]{figures/boosted/results/corr_2000_data.pdf}
\caption{Nuisance parameters associated with the background modelling, after the conditional likelihood fit for a bulk RS graviton signal with $\mGrav=2\,\TeV$ and $\kMPl=1.0$. The tight constraints of 2b\_QCD\_CRShape and 3b\_QCD\_CRShape are a result of the nuisance parameter prior being unconstrained due to a lack of control region data at high mass.}
\label{fig:nuisanceParams}
\end{center}
\end{figure*}

Examples of the fit used to set these limits are shown in Figures~\ref{fig:postfit2000} and~\ref{fig:postfit2500}, where a narrow scalar is used for the signal model. In the case the best-fit is negative, the fit is repeated with mu bounded to zero. This happens at several mass points, for example at 1.5, 2.5 and 3 TeV. At 2~TeV, the fitted signal is positive ($\mu=0.1\pm0.25$) though well consistent with the background-only hypothesis. In all fits, good agreement is seen between data and the background model.

\begin{figure*}[htbp!]
\begin{center}
\includegraphics[width=0.32\textwidth,angle=-90]{figures/boosted/results/postfitplot_s_2000_b2b.pdf} 
\includegraphics[width=0.32\textwidth,angle=-90]{figures/boosted/results/postfitplot_s_2000_b3b.pdf} 
\includegraphics[width=0.32\textwidth,angle=-90]{figures/boosted/results/postfitplot_s_2000_b4b.pdf} 
\caption{Postfit distributions after fitting the data with the 2000 GeV signal hypothesis. The signal strenght is slightly positive.}
\label{fig:postfit2000}
\end{center}
\end{figure*}

\begin{figure*}[htbp!]
\begin{center}
\includegraphics[width=0.32\textwidth,angle=-90]{figures/boosted/results/postfitplot_s_2500_b2b.pdf} 
\includegraphics[width=0.32\textwidth,angle=-90]{figures/boosted/results/postfitplot_s_2500_b3b.pdf} 
\includegraphics[width=0.32\textwidth,angle=-90]{figures/boosted/results/postfitplot_s_2500_b4b.pdf} 
\caption{Postfit distributions after fitting the data with the 2500 GeV signal hypothesis. The signal strength is zero.}
\label{fig:postfit2500}
\end{center}
\end{figure*}

The impact of the uncertainties on the fitted signal cross section is displayed in Fig.~\ref{fig:ranking2000} for the three signal models at 2000 GeV. The parameters are ranked by their postfit impact. Only the leading 30 nuisance parameters are displayed.

\begin{figure*}[htbp!]
\begin{center}
\includegraphics[width=0.32\textwidth,angle=-90]{figures/boosted/results/ranking_okt18_s_2000.pdf} 
\includegraphics[width=0.32\textwidth,angle=-90]{figures/boosted/results/ranking_okt18_g10_2000.pdf} 
\includegraphics[width=0.32\textwidth,angle=-90]{figures/boosted/results/ranking_okt18_g20_2000.pdf} 
\caption{The impact of nuisance parameters on the fitted cross section, ranked by their postfit impact. The signal mass used in this fits is 2000~GeV, and the signal model is (a) narrow scalar, (b) c=1 Gravition and (c) c=2 Graviton.}
\label{fig:ranking2000}
\end{center}
\end{figure*}
