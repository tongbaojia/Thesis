\textbf{THIS SECTION IS STILL TO BE UPDATED}
\paragraph{}
The effect of the following theoretical uncertainties on the signal acceptance and efficiency have been evaluated: uncertainties in the parton density functions; uncertainties due to missing higher order terms in the matrix element; uncertainties in the modelling of the underlying event (including multi-parton interactions), of hadronic showers and of initial and final state radiation. All of the uncertainties are evaluated for the Bulk RS $c=1.0$ benchmark model. These have been evaluated in exactly the same way as for the resolved analysis.%, as described in Section \ref{sec:theorysyst-resolved}.

\paragraph{}
To evaluate the potential effect of missing higher order terms in the matrix element, the renormalisation and factorisation scales used in the signal generation were varied coherently by factors of $0.5\times$ and $2\times$. 
%The effect is shown as function of graviton mass in Figure \ref{fig:BoostedScaleVar}. 
The shift induced by doubling the scales is consistent with zero across the full mass range, however halving the scales results in a constant -1\% shift. % ($p(\chi^2, n_{d.o.f.}) = 0.21$ and 0.30). 

% \begin{figure*}
% \begin{center}
% %%\includegraphics[angle=270, width=0.68\textwidth]{figures/boosted/Syst_Theory/ScaleVar_4Tag.pdf}
% \caption{The fractional change of the signal acceptance due to coherent variation of the renormalisation and factorisation scale as a function of resonance mass is shown as the triangles. Zeroth-order fits to these points are also shown as the solid lines.}\label{fig:BoostedScaleVar}
% \end{center}
% \end{figure*}

\paragraph{}
Uncertainties due to modelling of the parton shower and the underlying event (including multi-parton interactions) are evaluated by varying the parameters of the A14 Pythia 8 tune used in the simulation of the graviton signal samples. Ten ``eigentune'' variations are produced for each mass point. These variations, along the principal directions of the covariance matrix at the A14 tune minimum as described in Ref. \cite{ATL-PHYS-PUB-2014-021}, provide good coverage of the experimental and modelling uncertainties implicit in the tuning. Variation 1 is influenced mainly by underlying event effects; variation 2 by jet structure; and variations 3a-3c are mainly influenced by extra jet production. 
%Figures \ref{fig:A14Var1and2} and Figures \ref{fig:A14Var3} show the impact of these variations on the signal acceptance.

% \begin{figure*}
% \begin{center}
% %\subfloat[]{%\includegraphics[angle=270, width=0.48\textwidth]{figures/boosted/Syst_Theory/A14_Var1.pdf}}
% %\subfloat[]{%\includegraphics[angle=270, width=0.48\textwidth]{figures/boosted/Syst_Theory/A14_Var2.pdf}}
% \caption{Fractional change of the signal acceptance due to the variation of A14 tune parameters as a function of resonance mass. (a) shows the impact of eigentune variation 1, while (b) shows the impact of eigentune variation 2. For each variation, the positive and negative shifts are depicted as upward or downward facing triangles respectively. Fits to these points are shown as solid lines. \textbf{TO BE UPDATED} }
% \label{fig:BoostedA14Var1and2}
% \end{center}
% \end{figure*}

% \begin{figure*}
% \begin{center}
% %\subfloat[]{%\includegraphics[angle=270, width=0.48\textwidth]{figures/boosted/Syst_Theory/A14_Var3a.pdf}}
% %\subfloat[]{%\includegraphics[angle=270, width=0.48\textwidth]{figures/boosted/Syst_Theory/A14_Var3b.pdf}}\\
% %\subfloat[]{%\includegraphics[angle=270, width=0.48\textwidth]{figures/boosted/Syst_Theory/A14_Var3c.pdf}}
% \caption{Fractional change of the signal acceptance due to the variation of A14 tune parameters as a function of resonance mass. (a) shows the impact of eigentune variation 3a, while (b) shows the impact of eigentune variation 3b and (c) variation 3c. For each variation, the positive and negative shifts are depicted as upward or downward facing triangles respectively. Fits to these points are shown as solid lines. \textbf{TO BE UPDATED} }
% \label{fig:BoostedA14Var3}
% \end{center}
% \end{figure*}

\paragraph{}
A14 tune variations 1 does not have a significant effect on the signal acceptance. Variation 2 has an asymmetric impact on the acceptance that is quadratically dependent on the resonance mass, reaching +4\% and -8\% at $\mGrav=3$\,TeV. Variation 3a is consistent with zero shift for all masses when the downward variation is applied, while the upward variation leads to a signal acceptance uncertainty which is quadratic in \mGrav reaching $-8\%$ for $\mGrav=3$\,TeV. Variation 3b is again asymmetric with quadratic behaviour, with the upward variation reaching $+2\%$ and the downward variation $-6\%$ at $\mGrav=3$\,TeV. Variation 3c is symmetric, with the corresponding acceptance uncertainty $\pm 2\%$ for all masses. These uncertainties will be considered in the statistical analysis.

\paragraph{}
The PDF uncertainty is evaluated using the 100 NNPDF 2.3 LO replicas, as in the resolved analysis. 
%The calculated PDF uncertainty is shown in Figure \ref{fig:BoostedPDFUnc} as upward and downward shifts from unity. 
In this case, the uncertainty in acceptance due to PDF uncertainties is grows to $\pm3\%$ at the very highest masses considered for the boosted analysis. %For this reason, it is neglected in the statistical analysis described in Section \ref{sec:statistical-analysis}.
% \begin{figure*}
% \begin{center}
% %%\includegraphics[angle=270, width=0.68\textwidth]{figures/boosted/Syst_Theory/Boosted_PDFUncertaintyGraph.pdf}
% \caption{The fractional uncertainty on the acceptance due to PDF uncertainties as a function of resonance mass is shown as the triangles. A quadratic fit to these points is also shown as the solid line. \textbf{TO BE UPDATED} }\label{fig:BoostedPDFUnc}
% \end{center}
% \end{figure*}
