%sec-introduction.tex
\paragraph{}
The discovery of a Higgs boson ($h$) \cite{Aad:2012tfa,Chatrchyan201230} at the Large Hadron Collider (LHC) consistent with the predictions of the Standard Model (SM) \cite{ATLAS:HiggsCouplings,PhysRevD.89.092007} motivates an enhanced effort to search for new physics via the Higgs sector.  
\TeV-scale resonances such as the first Kaluza-Klein (KK) excitation of the graviton,
\Grav, predicted in the bulk Randall-Sundrum (RS) model~\cite{Agashe:2007zd,Fitzpatrick} or the heavy neutral scalar, $H$, of two-Higgs-doublet models (2HDM)~\cite{Branco:2011iw} are among many new physics models predict rates of Higgs boson pair production significantly higher than the SM rate~\cite{PhysRevD.58.115012,Grigo20131,PhysRevLett.111.201801}. Enhanced non-resonant \pptohh production can arise in models such as those with new, light, coloured scalars \cite{PhysRevD.86.095023} or direct $t\bar{t}hh$ vertexes \cite{Grober:2010yv,Contino:2012xk}.

\paragraph{}
This analysis is based on the previous ATLAS searches in the \bbbb final state~\cite{ATLASHHbbbb, ATLAS-CONF-2016-017}, which set limits on both resonant and non-resonant Higgs boson pair production.  For resonant KK Gravitons, the Run 1 analysis excluded resonance masses between 500 \GeV\ up to 720 - 990~\GeV, depending on the model parameters, and set cross section limits of $\sim 5-10$ fb depending on resonance mass.  For non-resonant production, the analysis set a cross section upper limit of 0.62 pb~\cite{ATLASHHComb}.  ATLAS has carried out searches using other final states -- \bbgg, \bbtautau and \WWgg --, which were combined with the \bbbb results in \cite{ATLASHHComb}. CMS has searched in the multi-lepton and multi-lepton+photons final-states in the context of 2HDM extensions of the Higgs sector~\cite{Khachatryan:2014jya}. CMS has also searched for narrow resonances in the $b\bar{b}b\bar{b}$ channel~\cite{Khachatryan:2015yea}, as well as the \bbgg~\cite{CMSbbgammagamma} and \bbtautau~\cite{CMSbbtt}.

\paragraph{}
The ATLAS and CMS results along with phenomenological studies demonstrate that despite the fully hadronic final state being subject to a large multijet background, searches for new physics in the \pptofourb process have good sensitivity for both resonant~\cite{PhenoBBBB,GouzevitchBBBBPheno} and non-resonant signals~\cite{DeLimaBBBB, Wardrope:2014kya}. One contributing factor to this sensitivity is the high expected branching ratio for \hhbbbb of 33\%.. 

\paragraph{}
The analysis presented in this note is designed to search for two high-momentum \bbbar systems with masses consistent with $m_{h}$. Compared to a more inclusive $b\bar{b}b\bar{b}$ final-state analysis, this topology has many benefits due to the large required momentum and angular separation between the two \bbbar systems: (i) excellent rejection of all backgrounds; (ii) highly efficient triggering using high-$p_{T}$ jet triggers; and (iii) negligible combinatorial ambiguity in forming Higgs boson candidates.

\paragraph{}
Higgs boson candidates are reconstructed using large \akt jets~\cite{Cacciari:2008gp} with radius parameter $R = 1.0$, matched to small radius jets constructed from tracks which have been $b$-tagged with a multivariate $b$-tagging algorithm \cite{ATL-PHYS-PUB-2015-022}. This reconstruction technique offers good efficiency for Higgs boson momenta above $\sim 250$ GeV and is therefore complementary to the resolved approach pursued in \cite{ResolvedNote}, which offers good acceptance for Higgs bosons with low \pt.

%The analysis in this note dramatically improves the acceptance to events where the mass of the di-Higgs-boson system is above 2 TeV through the use of additional signal regions and through loosening the requirements on track jets within the \largeR jets.  

\paragraph{}
The final signal regions in this analyses are determined using the leading and sub-leading Higgs candidate masses. In addition, the background and systematic uncertainty estimations methods rely on (a) reducing the number of required $b$-tagged jets and (b) using alternative selection of the leading and sub-leading Higgs candidate masses which are orthogonal to the signal region. 

%The decay of heavy resonances $X$ gives rise to boosted final states.  At moderate masses $m_{X} \in [500, 100]$\, GeV, this boost produces events where the two Higgs bosons decay back to back, thus creating events with two hemispheres each containing two $b$-jets.  At higher masses, $m_{X}>1$\, TeV,  the system becomes sufficiently boosted and the two b-hadrons from a Higgs boson decay can become collimated to the point
%that the separation between them will be smaller than the distance parameter $R$ of the individual $b$-jets.
%In this case, the $b$-jets will fail to be individually resolved by standard jet algorithms. 
%
%This analysis studies both the resolved ($m_{X} \in [500, 100]$\, GeV) and boosted ($m_{X}>1$\, TeV) regime.  In the resolved case, all four $b$-jets from the Higgs decay are reconstructed and tagged using the ATLAS $b$-tagging algorithms, and the moderate boost of the Higgs bosons allows for the disambiguation of the possible $h\to b\bar{b}$ pairings.  In the boosted case, the Higgs bosons are reconstructed with large radius jets ($R\geq1.0$) combined with jet grooming algorithms\cite{JetMassAndSubstructure,atlasConf:BoostedTop:2012} to mitigate the effects of pileup and improve the jet mass and energy resolution.  In addition, due to the large boost, the $b$-jets in the boosted regime can become highly collimated, thus requiring the use of small radius track-bases jet $b$-tagging~\cite{trackjetBtag}.  Track-based jets (called ``track jets'') can be chosen to originate from the 
%primary vertex, and may be calibrated down to a smaller distance parameter and lower \pt\ than calorimeter jets without suffering 
%from a large fake rate due to pileup interactions in the event. The b-tagging algorithm is applied to tracks with loose impact parameter (IP) constraints in a region of interest around each track jet axis,
%enabling the secondary vertex from the b-hadron decay to be reconstructed.  B-tagged track jets may then be
%associated to large-$R$ calorimeter jets to fully exploit calorimeter-based jet substructure and to access the full energy measurement.

