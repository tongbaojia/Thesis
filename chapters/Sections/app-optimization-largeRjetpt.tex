\section{Optimization of large-\R jet $\pt$ thresholds}
\label{sec:app-optimization-largeRjetpt}

Note: this section was written before the HLT trigger prescale was changed for 2016.

The large-\R jet $\pt$ thresholds are chosen to be leading jet $\pt >  350 $ GeV, sub-leading jet $ \pt >  250 $ GeV in 2015. In 2016, due to the higher luminosity, the HLT trigger prescale could be changed. This hints that a possible raise of jet $\pt$ threshold is necessary. Different jet $\pt$ cuts have been tested, with the method described in Appendix ~\ref{sec:app-selection-optimization}. The results are shown in Figure ~\ref{fig:app-optimization-largeRjetpt400}, ~\ref{fig:app-optimization-largeRjetpt425}, and ~\ref{fig:app-optimization-largeRjetpt450}. As shown, the sub-leading jet $\pt$ cut has little effects on the final significance, while the leading jet $\pt$ cut mostly affects the significance for signal around 1 TeV. For leading jet  $\pt$ cut at 400 GeV, 1 TeV signal has comparable significance as before, while 900 GeV signal sample has a lower significance. For leading jet $\pt$ cut at 450 GeV, 1 TeV signal significance is brought down, while 1.1 TeV signal isn't affected as much. Hence, it should be optimal to raise the leading jet $\pt$ cut up to 450 GeV if only above 1.1 TeV signal samples are considered.

\begin{figure*}[htbp!]
\begin{center}
%\includegraphics[width=0.45\textwidth,angle=-90]{figures/boosted/SigEff/jl400_relsig_0_3100_1.pdf}\\
%\includegraphics[width=0.45\textwidth,angle=-90]{figures/boosted/SigEff/jl400js275_relsig_0_3100_1}\\
%\includegraphics[width=0.45\textwidth,angle=-90]{figures/boosted/SigEff/jl400js300_relsig_0_3100_1}
  \caption{Estimated significance and comparison to leading jet $\pt >  350 $ GeV, sub-leading jet $ \pt >  250 $ GeV. Top: leading jet $\pt >  400 $ GeV, sub-leading jet $ \pt >  250 $ GeV; middle: leading jet $\pt >  400 $ GeV, sub-leading jet $ \pt >  275 $ GeV; bottom: leading jet $\pt >  400 $ GeV, sub-leading jet $ \pt > 300 $ GeV. }
  \label{fig:app-optimization-largeRjetpt400}
\end{center}
\end{figure*}


\begin{figure*}[htbp!]
\begin{center}
%\includegraphics[width=0.45\textwidth,angle=-90]{figures/boosted/SigEff/jl425_relsig_0_3100_1.pdf}\\
%\includegraphics[width=0.45\textwidth,angle=-90]{figures/boosted/SigEff/jl425js275_relsig_0_3100_1}\\
%\includegraphics[width=0.45\textwidth,angle=-90]{figures/boosted/SigEff/jl425js300_relsig_0_3100_1}
  \caption{Estimated significance and comparison to leading jet $\pt >  350 $ GeV, sub-leading jet $ \pt >  250 $ GeV. Top: leading jet $\pt >  425 $ GeV, sub-leading jet $ \pt >  250 $ GeV; middle: leading jet $\pt >  425 $ GeV, sub-leading jet $ \pt >  275 $ GeV; bottom: leading jet $\pt >  425 $ GeV, sub-leading jet $ \pt > 300 $ GeV. }
  \label{fig:app-optimization-largeRjetpt425}
\end{center}
\end{figure*}


\begin{figure*}[htbp!]
\begin{center}
%\includegraphics[width=0.45\textwidth,angle=-90]{figures/boosted/SigEff/jl450_relsig_0_3100_1.pdf}\\
%\includegraphics[width=0.45\textwidth,angle=-90]{figures/boosted/SigEff/jl450js275_relsig_0_3100_1}\\
%\includegraphics[width=0.45\textwidth,angle=-90]{figures/boosted/SigEff/jl450js300_relsig_0_3100_1}
  \caption{Estimated significance and comparison to leading jet $\pt >  350 $ GeV, subleading jet $ \pt >  250 $ GeV. Top: leading jet $\pt >  450 $ GeV, subleading jet $ \pt >  250 $ GeV; middle: leading jet $\pt >  450 $ GeV, subleading jet $ \pt >  275 $ GeV; bottom: leading jet $\pt >  450 $ GeV, subleading jet $ \pt > 300 $ GeV. }
  \label{fig:app-optimization-largeRjetpt450}
\end{center}
\end{figure*}
