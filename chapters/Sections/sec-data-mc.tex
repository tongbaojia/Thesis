\subsection{Dataset}

%%\paragraph{}
The studies presented in this note are based on the 2015 and 2016 dataset at $\sqrt{s}=13$~TeV recorded by the ATLAS experiment~\cite{detPaper}. Jet triggers were used to select the data (details below). The analysis also ran over the 2015 and the 2016 debug stream, in which no events that pass the full event selection were found. The 2015 dataset corresponds to a total integrated luminosity of 3.2\, fb$^{-1}$, and the 2016 dataset corresponds to a total integrated luminosity of 32.9\, fb$^{-1}$. The total luminosity is 36.1\, fb$^{-1}$.

%%\paragraph{}
The analysis is based on EXOT8 derivations of the xAOD, the contents of which can be found at \href{https://svnweb.cern.ch/trac/atlasoff/browser/PhysicsAnalysis/DerivationFramework/DerivationFrameworkExotics/trunk/share/EXOT8.py}{DerivationFrameworkExotics/EXOT8}. The 20.7.8.7 derivation cache was used in the analysis, which corresponds to \texttt{DerivationFrameworkExotics-00-03-86} and $p$-tags \texttt{p2950}. The boosted slimming keeps events with at least two large-$R$ jets with \pt~$>$200 GeV. 
%In the derivationt here is another cut requiring there are at least two trackjets with MV2c10 score greater than -10. Because there is no pT cut, this requirement is almost always passed--events doesn't pass this hints the large-R jet has one and only one tracking activities, which is also probably a good idea to keep it out. However, this should be modified in the future.

%%\paragraph{}
The studies presented currently in the main context of this note are extended to incorporate 2016 data. In this note, the reconstruction of both the 2015 and 2016 datasets are performed using software release 20.7. Simulated data samples from the mc15c campaign are be used, corresponding to $p$-tags \texttt{p2952-p2949}.
% to look up derivation tag info:
% > grep Exotics /cvmfs/atlas.cern.ch/repo/sw/software/x86_64-slc6-gcc48-opt/20.1.8/AtlasDerivation/20.1.8.5/AtlasDerivationRelease/cmt/requirements
% https://twiki.cern.ch/twiki/bin/viewauth/AtlasProtected/DerivationProductionTeam#Info_on_AtlasDerivation_caches_a

%%\paragraph{}
The 2015 data sets from release 20.7 that are currently used in this analysis are:
\noindent
\\
{\scriptsize
\verb|data15_13TeV.periodD.physics_Main.PhysCont.DAOD_EXOT8.grp15_v01_p2950|\\
\verb|data15_13TeV.periodE.physics_Main.PhysCont.DAOD_EXOT8.grp15_v01_p2950|\\
\verb|data15_13TeV.periodF.physics_Main.PhysCont.DAOD_EXOT8.grp15_v01_p2950|\\
\verb|data15_13TeV.periodG.physics_Main.PhysCont.DAOD_EXOT8.grp15_v01_p2950|\\
\verb|data15_13TeV.periodH.physics_Main.PhysCont.DAOD_EXOT8.grp15_v01_p2950|\\
\verb|data15_13TeV.periodJ.physics_Main.PhysCont.DAOD_EXOT8.grp15_v01_p2950|
}

%%\paragraph{}
The current 2016 data sets are:
\noindent
\\
{\scriptsize
\verb|data16_13TeV.periodA.physics_Main.PhysCont.DAOD_EXOT8.grp16_v01_p2950|\\
\verb|data16_13TeV.periodB.physics_Main.PhysCont.DAOD_EXOT8.grp16_v01_p2950|\\
\verb|data16_13TeV.periodC.physics_Main.PhysCont.DAOD_EXOT8.grp16_v01_p2950|\\
\verb|data16_13TeV.periodD.physics_Main.PhysCont.DAOD_EXOT8.grp16_v01_p2950|\\
\verb|data16_13TeV.periodE.physics_Main.PhysCont.DAOD_EXOT8.grp16_v01_p2950|\\
\verb|data16_13TeV.periodF.physics_Main.PhysCont.DAOD_EXOT8.grp16_v01_p2950|\\
\verb|data16_13TeV.periodG.physics_Main.PhysCont.DAOD_EXOT8.grp16_v01_p2950|\\
\verb|data16_13TeV.periodI.physics_Main.PhysCont.DAOD_EXOT8.grp16_v01_p2950|\\
\verb|data16_13TeV.periodK.physics_Main.PhysCont.DAOD_EXOT8.grp16_v01_p2950|\\
\verb|data16_13TeV.periodL.physics_Main.PhysCont.DAOD_EXOT8.grp16_v01_p2950|
}


\clearpage
\subsection{Simulated signal and background samples}
%%\paragraph{}
All MC samples used in this analysis are produced with full simulation, with additional \pileup interactions in each simulated event 
modeled by adding multiple soft $pp$ collisions
generated by \Pythia~8.165~\cite{pythia8} with the MSTW2008 LO PDF and AU2 tune~\cite{MC12AU2}.  After event generation and the addition of \pileup, 
the response of the ATLAS detector to particles
passing through the detector elements is simulated with the GEANT4 toolkit~\cite{Geant4,simulation}
and events are reconstructed using the same software used to reconstruct events in data.
%Simulation events are further corrected to reproduce the amount of pileup produced in the data, using the standard ATLAS pileup reweighting tool~\cite{pileuptwiki}.

\subsubsection{Signal MC production}
%A non-resonant SM signal sample has been generated using aMC@NLO. These \ggtofourb events are generated at NLO, using the exact form factors for the top loop taken from HPAIR~\cite{PhysRevD.58.115012,Plehn199646}. The sample is:
%\begin{Verbatim}[fontsize=\scriptsize]
%mc15_13TeV.342619.aMcAtNloHerwigppEvtGen_UEEE5_CTEQ6L1_CT10ME_hh_4b.merge.DAOD_EXOT8.e4419_s2608_r6869_r6282_p2454.
%\end{Verbatim}

%The gluon-fusion production cross-section used is evaluated at NNLO+NNLL in QCD \cite{LHCHXSWGHH}: $\sigma(pp\rightarrow hh\rightarrow b\bar{b}b\bar{b})~=~12.7\pm1.6$\,fb, where the uncertainty term includes the effects of uncertainties in the renormalization and factorization scale, PDFs, $\alpha_S$, effects of finite $m_t$ in loops and $Br\left(H\rightarrow b\bar{b}\right)$.
%, summed with the NLO predictions for vector-boson-fusion, top-pair-associated and vector-boson-associated production from Ref.~\cite{1401.7340}. The resulting cross-section is $\sigma(pp\rightarrow hh\rightarrow b\bar{b}b\bar{b}) = 3.6\pm0.5$\,fb, where the uncertainty term includes the effects of uncertainties in the renormalization and factorization scale, PDFs, $\alpha_S$ and $Br\left(H\rightarrow b\bar{b}\right)$.

%%\paragraph{}
Two benchmark resonant signal models are considered: a spin-2 graviton within a Bulk Randall-Sundrum Kaluza-Klein model and a spin-0 heavy neutral Higgs boson within a 2HDM model. 

%%\paragraph{}
The Bulk RS KK graviton signal samples have been generated for 20 mass points from 300 to 3000\,GeV using the \Madgraph generator\cite{MG5aMCatNLO} with the NNPDF2.3 LO PDF~\cite{Ball:2012cx} and the A14 tune \cite{ATL-PHYS-PUB-2014-021}, and hadronic showers are produced in \Pythia8.  For all signal samples, the Higgs mass has been set to 125.0 GeV. The cross-section times branching ratio values are reported in Tables \ref{tab:signal_c10_xsec} and \ref{tab:signal_c20_xsec}, setting $c \equiv k/\bar{M}_P$ to 1.0 and 2.0, respectively.  The names of signal datasets are listed in Appendix~\ref{app:signal-samples}. Concerning the level of freedom in his model, Kaustubh Agashe~\cite{Agashe} suggested that $c$ cannot be increased much beyond two.

%%\paragraph{}
For signal kinematic distributions, see Appendix~\ref{app:signal-dist}.
%\Figref{signal-samples} shows the RSG mass for the simulated signal points.  The larger intrinisc width of the RS graviton resonance for $c=2$ can be seen, as the width increases as the square of the coupling (see \Eref{RSwidth}). The cross-section times branching ratio values for the 2HDM samples are reported in Table \ref{tab:signal_2hdm_xsec}.

\begin{table}[htbp]
\begin{center}
\begin{tabular}{c | c | c | c | c | c}
\hline
   DSID  &  $m_{G_{KK}}$ (GeV)   &  $\Gamma_{G_{KK}}$  (GeV) &   $\sigma \times$ BR($G_{KK}\to hh$) (fb) &  BR($G_{KK}\to hh$) & $N_{events}$ \\
\hline
 301488  & 300 & 8.365 & 1319.9 $\pm$ 1.0 & 0.90 & 79000 \\
301490 & 500 &18.43 & 892.4 $\pm$ 0.6 & 6.43 & 93400\\ 
301491 & 600 & 26.08 & 410.4 $\pm$ 0.3 & 6.95 & 99000\\ 
301492 & 700 & 33.65 & 201.48 $\pm$ 0.15 & 7.19 & 54000\\ 
301493 & 800 & 41.06 & 105.49 $\pm$ 0.07 & 7.33 & 70000\\ 
301494 & 900 & 48.30 & 58.35 $\pm$ 0.04 & 7.41 & 85000\\ 
301495 & 1000 & 55.40 & 33.68 $\pm$ 0.02 & 7.47& 100000\\ 
301496 & 1100 & 62.38 & 20.23 $\pm$ 0.01 & 7.51 & 99000\\
301497 & 1200 & 69.27 & 12.54 $\pm$ 0.01 & 7.54 & 99000\\
301498 & 1300 & 76.09 & 7.979 $\pm$ 0.005 & 7.56 & 19000\\
301499 & 1400 & 82.84 & 5.201 $\pm$ 0.004 & 7.58 & 98600 \\
301500 & 1500 & 89.54 & 3.450 $\pm$ 0.002 & 7.59 & 99000\\
301501 & 1600 & 96.20 & 2.336 $\pm$ 0.002 & 7.60 & 99000\\
301502 & 1800 & 109.4 & 1.116 $\pm$ 0.001 & 7.62 & 15000\\
301503 & 2000 & 122.5 & $0.5559 \pm 3\times10^{-4}$ & 7.63 & 88800\\
301504 & 2250 & 138.8 & $0.2486 \pm 2\times10^{-4}$ & 7.64 & 99000\\
301505 & 2500 & 155.0 & $0.1158 \pm 1\times10^{-4}$ & 7.65 & 60000\\
301506 & 2750 & 171.1 & $0.05585 \pm 4\times10^{-5}$ & 7.66 & 58600\\
301507 & 3000 & 187.2 & $0.02772 \pm 2\times10^{-5}$ & 7.66 & 78000\\
\hline
\end{tabular}
\caption{Cross-section times branching ratio for RS graviton samples with $c \equiv k/\bar{M}_P = 1.0$ 
as a function of the graviton mass.}
\label{tab:signal_c10_xsec}
\end{center}
\end{table}

\begin{table}[htbp]
\begin{center}
\begin{tabular}{c | c | c | c | c | c}
\hline
   DSID  &  $m_{G_{KK}}$ (GeV)   &  $\Gamma_{G_{KK}}$  (GeV) &   $\sigma \times$ BR($G_{KK}\to hh$) (fb) &  BR($G_{KK}\to hh$) & $N_{events}$ \\
\hline
301508 & 300 & 33.46 & 9997 $\pm$ 11 & 0.90 & 90000\\
301509 & 400 & 45.22 & 8560 $\pm$ 7 & 4.99 & 60000\\
301510 & 500 & 73.74 & 3755 $\pm$ 3 & 6.43 & 100000\\
301511 & 600 & 104.3 & 1657 $\pm$ 1 & 6.95 & 98800\\
301512 & 700 & 134.6 & 789.9 $\pm$ 0.6 & 7.19 & 99000\\
301513 & 800 & 164.2 & 404.3 $\pm$ 0.3 & 7.33 & 99000\\
301514 & 900 & 193.2 & 219.3 $\pm$ 0.2 & 7.41 & 100000\\
301515 & 1000 & 221.6 & 125.1 $\pm$ 0.1 & 7.47 & 100000\\
301516 & 1100 & 249.5 & 74.19 $\pm$ 0.05 & 7.51 & 58600\\
301517 & 1200 & 277.1 & 45.48 $\pm$ 0.003 & 7.54 & 74000\\
301518 & 1300 & 304.4 & 28.72 $\pm$ 0.02 & 7.56 & 100000\\
301519 & 1400 & 331.4 & 18.55 $\pm$ 0.001 & 7.58 & 73800\\
301520 & 1500 & 358.2 & 12.27 $\pm$ 0.001 & 7.59 & 99000\\
301521 & 1600 & 384.8 & 8.254 $\pm$ 0.005 & 7.60 & 100000\\
301522 & 1800 & 437.7 & 3.913 $\pm$ 0.003 & 7.62 & 93400\\
301523 & 2000 & 490.1 & 1.951 $\pm$ 0.001 & 7.63 & 60000\\
301524 & 2250 & 555.2 & 0.8703 $\pm$ 0.0006 & 7.64 & 100000\\
301525 & 2500 & 620.0 & 0.4070 $\pm$ 0.0003 & 7.65 & 84000\\
\hline
\end{tabular}
\caption{Cross-section times branching ratio for RS graviton samples with $c \equiv k/\bar{M}_P = 2.0$ 
as a function of the graviton mass.}
\label{tab:signal_c20_xsec}
\end{center}
\end{table}

%%\paragraph{}
The heavy Higgs boson samples have been generated for the same 20 mass points from 300 to 3000\,GeV using the \Madgraph generator\cite{MG5aMCatNLO} with the 
CT10 PDF set. Hadronic showers are produced in \herwigpp using CTEQ6L1 and the UEEE5 event tune. 
For all signal samples, the Higgs boson mass has been set to 125.0 GeV. The width of the heavy Higgs boson, $\Gamma_H$, has been set to 1 GeV. This is because the width is dependent on 2HDM parameters. In Run-1, limits were set based on parameterised signal mass distributions with a representative range of widths.
%%\paragraph{}
To illustrate the properties of the $H$, a phase-space point \cba = 0.2, \tanb = 1 has been chosen within a Type-2 2HDM. The heavy Higgs boson partners' ($H, A, H^{\pm}$) masses are set such that $m_H = m_A = m_{H^{\pm}}$. The potential parameter that mixes the two Higgs doublets, $m_{12}$, is fixed such that $m_{12}^2 = m_A^2\tanb/(1+\tan^2\beta)$. This phase-space point has not been excluded by coupling measurements of the observed SM-like Higgs boson and is on the cusp of the observed 95\% C.L. exclusion of the Run 1 \Htohhb analysis. The relevant 2HDM properties are obtained from the HBSM group's ntuple for $\sqrt{s} = 13$\,TeV, version 1.6.3 \cite{HBSMNtuple}. They are reported in Table \ref{tab:signal_2hdm_xsec}.

\begin{table}[h]
\begin{center}
\begin{tabular}{c | c | c | c | c | c}\hline
DSID & $m_H$ (GeV) & $\Gamma_H$ (GeV) & $\sigma\times$Br($H\to hh \to b\bar{b}$) (pb) &  BR($H\to hh$) & $N_{\rm{events}}$\\\hline
343394 & 260 & 0.378 & 0.852 				& 0.489 &  -  \\
343395 & 300 & 0.961 & 0.925 				& 0.647 &  - \\
343396 & 400 & 5.84 & 0.522 				& 0.398 &  - \\
343397 & 500 & 15.2 & 0.211 				& 0.348 &  - \\
343398 & 600 & 26.0 & $9.96\times10^{-2}$ 	& 0.377 &  - \\
343399 & 700 & 38.9 &$ 5.02597\times10^{-2}$ & 0.418 &  - \\
343400 & 800 & 54.5 & $2.6561\times10^{-2}$ & 0.458 &  - \\
343401 & 900 & 73.2 & $1.45677\times10^{-2}$ & 0.494 &  - \\
343402 & 1000 & 95.6 & $8.24653\times10^{-3}$ & 0.525 &  - \\
343403 & 1100 & 122 & $4.79918\times10^{-3}$ & 0.552 &  - \\
343404 & 1200 & 154 & $2.86249\times10^{-3}$ & 0.575 &  - \\
343405 & 1300 & 190 & $1.74553\times10^{-3}$ & 0.594 &  - \\
343406 & 1400 & 232 & $1.08577\times10^{-3}$ & 0.610 &  - \\
343407 & 1500 & 280 & $6.87569\times10^{-4}$ & 0.624 &  - \\
\hline
\end{tabular}
\caption{Heavy Higgs boson properties for Type-2 2HDM with \cba = 0.2 and \tanb = 1. \BrHhh, \Brhbb and $\Gamma_H$ are parameter-dependent. The MC samples were generated with $\Gamma_H = 1$\,GeV.}
\label{tab:signal_2hdm_xsec}
\end{center}
\end{table}
 
%\begin{figure}[ht!]
%\begin{center}
%   %\includegraphics[width=0.45\textwidth,angle=-90]{figures/truth_plots/masses_RSG_c10.pdf}
%   %\includegraphics[width=0.45\textwidth,angle=-90]{figures/truth_plots/masses_RSG_c20.pdf}
%\caption{RSG signal mass points for $c=1.0$ (left) and $c=2.0$ (right).}
%\label{fig:signal-samples}
%\end{center}
%\end{figure}

\clearpage
\subsubsection{Background simulation}
%%\paragraph{}
While the dominant QCD multijet background (about 90\% of the total) was estimated in data, a \Pythia~\cite{pythia8} dijet sample was used to understand the physical processes contributing to this background and characteristics of the event selection. The usefulness of this background sample is limited by the generated number of events, given the high background rejection factors of the analysis selection. The dijet samples are listed below.
\\ \\
\noindent
{\scriptsize
\verb|mc15_13TeV.361020.Pythia8EvtGen_A14NNPDF23LO_jetjet_JZ0W.merge.DAOD_EXOT8.e3569_s2576_s2132_r7725_r7676_p2949|\\
\verb|mc15_13TeV.361021.Pythia8EvtGen_A14NNPDF23LO_jetjet_JZ1W.merge.DAOD_EXOT8.e3569_s2576_s2132_r7725_r7676_p2949|\\
\verb|mc15_13TeV.361022.Pythia8EvtGen_A14NNPDF23LO_jetjet_JZ2W.merge.DAOD_EXOT8.e3668_s2576_s2132_r7725_r7676_p2949|\\
\verb|mc15_13TeV.361023.Pythia8EvtGen_A14NNPDF23LO_jetjet_JZ3W.merge.DAOD_EXOT8.e3668_s2576_s2132_r7725_r7676_p2949|\\
\verb|mc15_13TeV.361024.Pythia8EvtGen_A14NNPDF23LO_jetjet_JZ4W.merge.DAOD_EXOT8.e3668_s2576_s2132_r7725_r7676_p2949|\\
\verb|mc15_13TeV.361025.Pythia8EvtGen_A14NNPDF23LO_jetjet_JZ5W.merge.DAOD_EXOT8.e3668_s2576_s2132_r7725_r7676_p2949|\\
\verb|mc15_13TeV.361026.Pythia8EvtGen_A14NNPDF23LO_jetjet_JZ6W.merge.DAOD_EXOT8.e3569_s2608_s2183_r7725_r7676_p2949|\\
\verb|mc15_13TeV.361027.Pythia8EvtGen_A14NNPDF23LO_jetjet_JZ7W.merge.DAOD_EXOT8.e3668_s2608_s2183_r7725_r7676_p2949|\\
\verb|mc15_13TeV.361028.Pythia8EvtGen_A14NNPDF23LO_jetjet_JZ8W.merge.DAOD_EXOT8.e3569_s2576_s2132_r7772_r7676_p2949|\\
\verb|mc15_13TeV.361029.Pythia8EvtGen_A14NNPDF23LO_jetjet_JZ9W.merge.DAOD_EXOT8.e3569_s2576_s2132_r7772_r7676_p2949|\\
\verb|mc15_13TeV.361030.Pythia8EvtGen_A14NNPDF23LO_jetjet_JZ10W.merge.DAOD_EXOT8.e3569_s2576_s2132_r7772_r7676_p2949|\\
\verb|mc15_13TeV.361031.Pythia8EvtGen_A14NNPDF23LO_jetjet_JZ11W.merge.DAOD_EXOT8.e3569_s2608_s2183_r7772_r7676_p2949|\\
\verb|mc15_13TeV.361032.Pythia8EvtGen_A14NNPDF23LO_jetjet_JZ12W.merge.DAOD_EXOT8.e3668_s2608_s2183_r7772_r7676_p2949|
}

%\Figref{pdgid-bg} demonstrates that the dominant multijet background is principally composed events arising from gluons (which then split to $b\bar{b}$), with less than 1 part per mille arising from light quarks or direct b-quark production.

%\begin{figure}[ht!]
%\begin{center}
%   %\includegraphics[width=0.5\textwidth,angle=-90]{figures/truth_plots/dijet_pdgId_4b_boosted.png}
%\caption{PDG values of a multijet background as well as various RS graviton signal samples.  The QCD background is primarily composed
%of jets arising from gluons (PDG ID = 21).}
%\label{fig:pdgid-bg}
%\end{center}
%\end{figure}

%%\paragraph{}
The \ttbar\ background is modeled using large all-hadronic and non-all-hadronic decay mode samples that have both been generated with \Powheg~\cite{powheg} and showered with \Pythia~\cite{pythia8}. The top mass in both samples is set to 172.5 GeV. Samples inclusive in $m_{tt}$ are listed here:
\\ \\
\noindent
{\scriptsize
\verb|mc15_13TeV.410000.PowhegPythiaEvtGen_P2012_ttbar_hdamp172p5_nonallhad.merge.DAOD_EXOT8.e3698_s2608_s2183_r7725_r7676_p2949|\\
\verb|mc15_13TeV.410007.PowhegPythiaEvtGen_P2012_ttbar_hdamp172p5_allhad.merge.DAOD_EXOT8.e4135_s2608_s2183_r7725_r7676_p2949|
}

The prediction of the \ttbar\ MC samples are normalized to the NNLO+NLL predicted inclusive \ttbar\ cross-section of
1821.87 pb multiplied by the all-hadronic branching ratio of 0.457 and non-all-hadronic of 0.543 as appropriate~\cite{TTbarXSec}. 

In order to keep statistical fluctuations small across the dijet mass spectrum, especially for large values of $m_{tt}$, additional \ttbar samples are generated in slices of \ttbar invariant mass. Those samples are listed in Table \ref{tab:tt}. The cross-section of the \ttbar process is normalized to NNLO+NNLL in QCD, as calculated by \textsc{Top++} 2.0 \cite{Czakon:2011xx}. The \POWHEG \textsc{hdamp} parameter \cite{ATL-PHYS-PUB-2014-005} is set to the top mass, taken as $m_{t} = 172.5$~GeV. Overlap with the inclusive ttbar samples is removed by a cut on the truth value of $m_{tt}$ at the analysis level.
 
\begin{table}[!htb]
\begin{small}
\begin{center}
\begin{tabular}{|c|l|c|c|c|c|r|}
        \hline
        DS ID & Process & Generator & $\sigma\times\text{BR}$ [nb] & $k$-factor & $\epsilon_{\text{filter}}$ & Events \\ \hline		
303722	& all-had \ttbar, $1.1 < m_{t\bar{t}} < 1.3$~TeV & \POWHEG + \PYTHIA6	&	$0.69625$ & 1.0 & 0.003958 &  513000 \\
303723	& all-had \ttbar, $1.3 < m_{t\bar{t}} < 1.5$~TeV & \POWHEG + \PYTHIA6	&	$0.69624$ & 1.0 & 0.001634 &  226000 \\
303724	& all-had \ttbar, $1.5 < m_{t\bar{t}} < 1.7$~TeV & \POWHEG + \PYTHIA6	&	$0.69622$ & 1.0 & 0.000723 &  100000 \\
303725	& all-had \ttbar, $1.7 < m_{t\bar{t}} < 2.0$~TeV & \POWHEG + \PYTHIA6	&	$0.69624$ & 1.0 & 0.000438 &  71000 \\
303726	& all-had \ttbar, $2.0 < m_{t\bar{t}} < 14$~TeV & \POWHEG + \PYTHIA6	&	$0.69624$ & 1.0 & 0.000259 &  44000 \\

301528	& nonall-had \ttbar, $1.1 < m_{t\bar{t}} < 1.3$~TeV & \POWHEG + \PYTHIA6	&	$0.69625$ & 1.0 & 0.00471933 &  544000 \\
301529	& nonall-had \ttbar, $1.3 < m_{t\bar{t}} < 1.5$~TeV & \POWHEG + \PYTHIA6	&	$0.69625$ & 1.0 & 0.00194400 &  229000 \\
301530	& nonall-had \ttbar, $1.5 < m_{t\bar{t}} < 1.7$~TeV & \POWHEG + \PYTHIA6	&	$0.69623$ & 1.0 & 0.00086308 &  99000 \\
301531	& nonall-had \ttbar, $1.7 < m_{t\bar{t}} < 2.0$~TeV & \POWHEG + \PYTHIA6	&	$0.69624$ & 1.0 & 0.00051910 &  74000 \\
301532	& nonall-had \ttbar, $2.0 < m_{t\bar{t}} < 14$~TeV & \POWHEG + \PYTHIA6	&	$0.69625$ & 1.0 & 0.00030919 &  44000 \\
\hline

\end{tabular}
\caption{\ttbar samples used in the analysis. The dataset ID, MC generator, production cross-sections,
$k$-factor, filter efficiency and total number of generated events are shown.}
\label{tab:tt}
\end{center}
\end{small}
\end{table}


%A reweighting is applied to both \ttbar\ samples to correct the top quark \pt\ spectra to be in agreement with the unfolded $\sqrt{s}=7$ TeV measurement as prescribed by the HSG5 group in the $h\rightarrow bb$ analysis~\cite{TopPt}.
%
%The overall \ttbar\ normalization is derived in a data-driven approach and the MC shape is validated in a \ttbar-enriched region, 
%as will be described in the background estimation sections of the resolved (Section~\ref{sec:datadriventtbar}) and boosted (Section~\ref{sec:boosted-ttbar}) analysis descriptions.
%as will be described in  Section~\ref{ttbar}, and the MC shape is validated in a \ttbar-enriched region as will be described in Section~\ref{ttbar}.

%A very small fraction of the background arises from $Z$ + jets events. A simulation sample generated with \Pythia for the resolved SM $Z\rightarrow bb$ cross section
%measurement~\cite{ZtobbMeasurement} is used to model this contribution for this analysis:
%\\ \\
%\noindent
%\texttt{\scriptsize
%mc12\_8TeV.147179.Pythia8\_AU2CTEQ6L1\_Zjets\_Ztobb\_BSubstruct\_pT\_160\_260.merge.NTUP\_COMMON.e1592\_s1499\_s1504\_r4168\_r3549\_p1654\\
%mc12\_8TeV.147180.Pythia8\_AU2CTEQ6L1\_Zjets\_Ztobb\_BSubstruct\_pT\_260.merge.NTUP\_COMMON.e1592\_s1499\_s1504\_r4168\_r3549\_p1654 
%}
%\\
%
%\noindent
%Here, a generator-level filter was applied using Cambridge-Aachen $R=1.2$ jets in two mutually exclusive 
%regions of pT phase space, with each sample containing 3 million events.  Events in the lower \pt\ sub-sample are filtered 
%with the requirement that the leading CA jet satisfies $160 <\pt< 260$~GeV 
%and events in the higher \pt\ sub-sample are filtered requiring that the leading CA jet
%has $\pt > 260$~GeV. These samples expected
%to cover our analysis phase space, as any other contributions would be much
%smaller, ie: have less b-tags in the final state and/or have a
%smaller cross section, or would not fall under our selection acceptance with a high \pt\ threshold applied.
%The event yield in the $Z$+jets MC sample is scaled to
%the \Pythia predicted cross-section, but after scaling the cross-section
%with a k-factor of 2.02/1.25 = 1.62, determined as the ratio of the \PowPythia
%NLO+PS cross-section for boosted $Z\rightarrow b\bar{b}$~production ($Z~\pt>$~200~GeV) to that
%predicted by this \Pythia sample (see~\cite{ZtobbMeasurement}). 
