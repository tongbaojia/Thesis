The statistical analysis is performed on the $m_{4j}$ ($m_{2J}$) distributions for the
resolved (boosted) analyses. They were shown in the previous section and are reported
below for a direct comparison between the different analyses and signal region
definitions.
%
%\begin{figure}[btph]
%\begin{center}
%%\includegraphics[scale=0.4]{../Paper2015/trunk/figuresOLD/resolved_4b_m4j_SR.pdf}
%\caption{Distribution of $m_{4j}$ in the signal region of the resolved analysis for data compared to the predicted background. 
%         The hashed bands represent the combined statistical and systematic uncertainty in the total background estimate. 
%         The expected signal distribution for a \Grav\ mass of 800 \GeV{} is also shown.}
%\label{fig:m4j_SR_resolved}
%\end{center}
%\end{figure}
%

\subsubsection{Test of the background model hypothesis}

A search for statistically significant deviation from the background model hypothesis is
performed following the procedure described in Sec.\,\ref{sec-search-procedure}, computing the
local $p_0$ value (using the toy MC technique).

The background model is found to describe the data and no significative excess is observed.

The results for the resolved and boosted 2b3b+4b analyses are shown in Figure~\ref{fig:p0_RSG10}.
The minimum local and global $p_0$ values and their significances are reported in Table~\ref{tab:p0_RSG10}.
I
%% In the resolved analysis the background model hypothesis is found to well describe the data,
%% within a local significance of 1.2 $\sigma$ for m=900 \GeV{}, when considering the largest deviation.

The background-only hypothesis is found to describe the data well, with the largest deviation---found at $\mGrav=1.5\,\GeV$ having a maximum local significance of only 1.1 $\sigma$.
%% is observed at 1300 \GeV{}.
%% In the 3b signal region a more significative local deviation is observed in the high mass region due
%% to the observation of one event with $m_{2J} = 2376$ \GeV{} (not corrected for the Higgs-mass constraint).
%% The background expectation for $m_{2J}>2$\,TeV (no events are observed within 2 and 2.5\,TeV)
%% is about 0.5 events.
%% The minimum local $p_0$ value corresponds to a significance of 3.4 $\sigma$.
%% Taking into account the look-elsewhere-effect the global significance is found to be 1.7 $\sigma$.
%% Similar results are found combining the 3b and 4b signal regions. 

\begin{figure}[ht!]
\begin{center}
  %\includegraphics[width=0.48\textwidth,angle=-90]{figures/StatAnalysis/p0/resolved_4b_p0_RSGC10.pdf}
\caption{Local $p_0$ value for the resolved (left) and boosted 3b+4b (right) analysed as a function of the Graviton mass for the KK graviton model
with $c \equiv k/\bar{M}_P = 1.0$.}
\label{fig:p0_RSG10}
\end{center}
\end{figure}

\begin{table}[ht!]
\begin{center}
\begin{tabular}{l|c|c|c|c|c}
                  & Local $p_0$ (mass [GeV]) & Local sign. & Trial factor & Global $p_0$ & Global sign. \\ 
  \hline
  \hline
Resolved          & 0.1 (900)        & 1.2         & -         & -        & - \\  
  \hline
\end{tabular}
\caption{Minimum local and global $p_0$ values along with the corresponding significances
  and the trial factors for the different analyses and for the KK graviton model
  with $c \equiv k/\bar{M}_P = 1.0$. The trial factor is computed only when a local
significance greater then 2 $sigma$ is observed.}
\label{tab:p0_RSG10}
\end{center}
\end{table}

\clearpage

\begin{figure}[ht!]
\begin{center}
  %\includegraphics[width=0.48\textwidth,angle=-90]{figures/StatAnalysis/p0/resolved_4b_p0_RSGC20.pdf}
\caption{Local $p_0$ value for the resolved (left) and boosted 3b+4b (right) analysed as a function of the Graviton mass for the KK graviton model
with $c \equiv k/\bar{M}_P = 2.0$.}
\label{fig:p0_RSG20}
\end{center}
\end{figure}

\begin{table}[ht!]
\begin{center}
\begin{tabular}{l|c|c|c|c|c}
                  & Local $p_0$ (mass [GeV]) & Local sign. & Trial factor & Global $p_0$ & Global sign. \\ 
  \hline
  \hline
Resolved          & 0.2 (900)        & 0.8         & -         & -        & - \\  
   \hline
\end{tabular}
\caption{Minimum local and global $p_0$ values along with the corresponding significances
  and the trial factors for the different analyses and for the KK graviton model
  with $c \equiv k/\bar{M}_P = 2.0$. The trial factor is computed only when a local
significance greater then 2 $sigma$ is observed.}
\label{tab:p0_RSG20}
\end{center}
\end{table}

\clearpage

\begin{figure}[ht!]
\begin{center}
  %\includegraphics[width=0.48\textwidth,angle=-90]{figures/StatAnalysis/p0/resolved_4b_p0_2HDM.pdf}
\caption{Local $p_0$ value for the resolved (left) and boosted 3b+4b (right) analysed as a function of the Graviton mass for the 2HDM model.}
\label{fig:p0_2HDM}
\end{center}
\end{figure}



\begin{table}[ht!]
\begin{center}
\begin{tabular}{l|c|c|c|c|c}
                  & Local $p_0$ (mass [GeV]) & Local sign. & Trial factor & Global $p_0$ & Global sign. \\ 
  \hline
  \hline
Resolved          & 0.059 (900)     & 1.6         & -         & -        & - \\  
  \hline
\end{tabular}
\caption{Minimum local and global $p_0$ values along with the corresponding significances
  and the trial factors for the different analyses and for the 2HDM model.
  The trial factor is computed only when a local
significance greater then 2 $sigma$ is observed.}
\label{tab:p0_2HDM}
\end{center}
\end{table}

\clearpage



\subsubsection{Exclusion limits}

As no significant excess was found, limits are placed on the RS graviton cross section.
The upper limit of $\sigma(\mathrm{G} \to{\mathrm{hh}}\to b\bar{b}b\bar{b})$ is shown 
The results for the resolved analysis are shown
in Figures~\ref{fig:brazil_hh_c10_obs},
Results for the RSG model with c=2.0 and 2HDM are reported in
 \ref{fig:brazil_hh_c20_obs}.
Finally the limits on the cross section for the 2HDM model are shown in
 \ref{fig:brazil_hh_2hdm_obs}.
The limits on the signal cross sections for all channels and models
are also tabulated at the end of the section.

The resolved analysis expects to exclude an RS graviton (c=1.0) in the range $463\leq\mGrav\leq 746$\,GeV
and observes an exclusion in the range $477\leq\mGrav\leq 785$\,GeV.
For this model no exclusion limit was expected for the boosted results and no exclusion limit
is observed.

In the case of the RSG model with c=2.0 the expected exclusion limit for the resolved analysis
is $m<1000$ \GeV{} and the observed limit is  $m<979$ \GeV{}.

\clearpage

\begin{figure*}
\begin{center}
%\includegraphics[width=0.48\textwidth]{{figures/StatAnalysis/limits/asymptotic/unblinded/BrazilPlot_asymptotics_RSGC10_AllSysts_Interpolated_unblinded}.pdf}
\caption{The expected and observed exclusion limits for the resolved (left) and boosted 3b+4b (right) analysis combining 4b-tag and 3b-tag results and
calculated including all systematic uncertainties for the KK graviton model with $c \equiv k/\bar{M}_P = 1.0$.
Limits are derived within the asymptotic approximation.}
\label{fig:brazil_hh_c10_obs}
\end{center}
\end{figure*}


\clearpage

\begin{figure*}
\begin{center}
%\includegraphics[width=0.48\textwidth]{{figures/StatAnalysis/limits/asymptotic/unblinded/BrazilPlot_asymptotics_hh_resolved_4b_allsyst_unblinded_c2.0}.pdf}
\caption{The expected and observed exclusion limits for the resolved (left) and boosted 3b+4b (right) analysis combining 4b-tag and 3b-tag results and
calculated including all systematic uncertainties for the KK graviton model with $c \equiv k/\bar{M}_P = 2.0$.
Limits are derived within the asymptotic approximation.}
\label{fig:brazil_hh_c20_obs}
\end{center}
\end{figure*}


\clearpage

\begin{figure*}
\begin{center}
%\includegraphics[width=0.48\textwidth]{{figures/StatAnalysis/limits/asymptotic/unblinded/BrazilPlot_asymptotics_2HDM_AllSysts_Interpolated_unblinded}.pdf}
\caption{The expected and observed exclusion limits for the resolved (left) and boosted 3b+4b (right) analysis combining 4b-tag and 3b-tag results and
calculated including all systematic uncertainties for the 2HDM model.
Limits are derived within the asymptotic approximation.}
\label{fig:brazil_hh_2hdm_obs}
\end{center}
\end{figure*}

\clearpage

\begin{table}
\begin{center}
%\input{figures/StatAnalysis/limits/asymptotic/unblinded/BrazilPlot_asymptotics_hh_resolved_4b_allsyst_unblinded_c1.0.tex}
\end{center}
\label{tab:brazil_hh_resolved_4b_obs_c10}
\caption{The expected and observed exclusion limits for the resolved 4b  analysis calculated including all systematic uncertainties
  for the KK graviton model with $c \equiv k/\bar{M}_P = 1.0$. Limits are derived within the asymptotic approximation.
  The values of the one and two standard deviations on the expected limit are reported.}
\end{table}
\clearpage



\begin{table}
\begin{center}
%\input{figures/StatAnalysis/limits/asymptotic/unblinded/BrazilPlot_asymptotics_hh_resolved_4b_allsyst_unblinded_c2.0.tex}
\end{center}
\label{tab:brazil_hh_resolved_4b_obs_c20}
\caption{The expected and observed exclusion limits for the resolved 4b  analysis calculated including all systematic uncertainties
  for the KK graviton model with $c \equiv k/\bar{M}_P = 2.0$. Limits are derived within the asymptotic approximation.
  The values of the one and two standard deviations on the expected limit are reported.}
\end{table}


\clearpage



\begin{table}
\begin{center}
%\input{figures/StatAnalysis/limits/asymptotic/unblinded/BrazilPlot_asymptotics_hh_resolved_4b_allsyst_unblinded.tex}
\end{center}
\label{tab:brazil_hh_resolved_4b_obs_2hdm}
\caption{The expected and observed exclusion limits for the resolved 4b  analysis calculated including all systematic uncertainties
  for the 2HDM model. Limits are derived within the asymptotic approximation.
  The values of the one and two standard deviations on the expected limit are reported.}
\end{table}

\clearpage

\subsubsection{Exclusion limits (Toy MC)}
\label{sec:exclusionlimits_syst_ToyMC}

The final exclusion limits, set with toy MC, for the KK graviton model with  $c \equiv k/\bar{M}_P = 1.0$
and including all systematic uncertainties are shown in XXX for both resolved analysis (4b only) and boosted analysis (3b and 4b). The same plots for KK graviton model with $c \equiv k/\bar{M}_P = 2.0$ can also be found in XXX. The same plot for 2HDM model can be found in XXX. The numbers of limits on signal cross sections for all channels and models are also tabulated at the end of the section.

The comparison of the exclusion limits for KK graviton model with $c \equiv k/\bar{M}_P = 1.0$
and including all systematic uncertainties derived with toy MC to the ones derived with asymptotic approximation can be found in XXX for both resolved and boosted analysis. The similar plots for KK graviton model with $c \equiv k/\bar{M}_P = 2.0$ can be found in XXX.

\clearpage


\subsubsection{Statistical analysis details}

Detailed results on the statistical analysis are reported in this section.
Pulls of the nuisance parameters (NP) and their correlation matrix are reported for each mass point
in Appendix~\ref{app:stat_details_obs} together with the signal and background distributions before and
after the fit to data compared to the observed data.

Three type of likelihood fits are considered in each of these sections:
\begin{itemize}
\item Unconstrained-likelihood fit: both the NP and the signal strength ($\mu$)
are fitted to data
\item Background-only constrained-likelihood fit: $\mu$ is fixed to 0 and only the NP are fitted to data
\item Background$+$signal constrained-likelihood fit: $\mu$ is fixed to the observed exclusion value ($\mu_{limit}$) and only the NP are fitted to data
\end{itemize}

\paragraph{Resolved analysis}

In the resolved analysis the observed data is found to be in very good agreement with the predicted background,
both in terms of normalization and shape, as can be seen from Figure~\ref{fig:m4j_SR_resolved}.
This level of agreement leads to a fitted value of $\mu$ close to 0 for the unconditional likelihood fit
and to the observed limit to be in very good agreement with the expected limit, as shown in the previous
section.

The pulls of the fitted NP have mean value $\sim$0 and width $\sim$1 for all fits, as shown
in the appendix (see for different signal mass hypotheses starting from Figure~\ref{fig:Obs_fitdetails_resolved_400}).

As a consequence of the above considerations and NP pulls observations either the background model
and the model with the signal scaled by the excluded value of $\mu$ are fitting well the data
without the need of changes in normalization and shapes of the background and signal components.
This can be appreciated from the unconditional, $\mu=0$ and $\mu=\mu_{limit}$ conditional fits
comparing the background and signal components before and after the fit.



\subsubsection{Non-resonant Limits}
Exclusion limits are set on non-resonant hh production using the same statistical procedures as for the resonant analysis, however event yields are used as the discriminating variable. Non-resonant Standard Model hh production via gluon-fusion is used as the benchmark signal. Given the mass of the di-Higgs-boson system peaks at approximately 400\,GeV, only the resolved analysis results shown in Table \ref{tab:unblinded} are considered. Those previously described systematic uncertainties that affect the signal or background normalisation are input to the limit-setting as nuisance parameters, however those affecting the \mfourj shape are not needed.

The expected and observed 95\% C.L. exclusion limits on SM non-resonant Higgs boson pair production are given in Table \ref{tab:smhhlims}.

\begin{table}[htp]
\caption{95\% C.L exclusion limits for SM non-resonant hh production, $\sigma(\mathrm{pp\rightarrow HH\rightarrow b\bar{b}b\bar{b}})$.}
\begin{center}\begin{tabular}{cccccc}\toprule
Observed $-2\sigma$\,[pb] & $-1\sigma$\,[pb] & Expected\,[pb] & $+1\sigma$\,[pb] & $+2\sigma$\,[pb]\\
\midrule
1.22 & 0.694 & 0.932 & 1.29 & 1.99 & 3.12 \\
\bottomrule
\label{tab:smhhlims}\end{tabular}\end{center}\end{table}
