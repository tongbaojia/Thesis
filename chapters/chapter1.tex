%!TEX root = ../dissertation.tex
\begin{savequote}[75mm]
Knowledge knows no bounds.
\qauthor{Creator}
\end{savequote}

\chapter{Motivation and Theory}
%\newthought{There's something out there that we don't know.} 

\section{The Standard Model and the Higgs Boson}
\paragraph{}
The Standard Model(SM) is the best description of the microscopic world ~\cite{Griffiths,Tully,Pdg,Schwartz}. 
The SM consists of three generations of leptons ($e$, $\mu$, $\tau$, $\nu_e$, $\nu_{\mu}$, $\nu_{tau}$) and quarks ($u$, $d$, $c$, $s$, $t$, $b$).
They all interact via the weak force. In addition, the charged leptons and quarks interact through the EM force and the quarks also interact through the strong force.
The SM also has the force mediators. EM force mediates through the photon $\gamma$, the strong force meidates through the gluon $g$, and the weak force mediates through $W^{\pm}$ and $Z$ bosons.
The SM predicts everything except the particle's mass, shown in Figure ~\ref{fig:SM}, which are measured experimentally.

\begin{figure}[h!]
  \centering
  \captionsetup{justification=centering}
  \includegraphics[width=0.6\textwidth]{figures/theory/SM}
  \caption{Fermions and bosons of the Standard Model and their properties~\cite{Pdg}.}
  \label{fig:SM}
\end{figure}

\paragraph{}
However, in SM, due to the gauge invriance under $SU(2)_{L}$, fermions have to be massless in order to have pure left handed states. 
The bosons must also be massless as required by the gauge principle. 
The Higgs mechanism introduces a scalr Higgs field with nonzero vaccum expectation values, which impledes and interacts with the propagation of gague bosons and fermions, hence gives them valid mass terms~\cite{Tully}. 
This broken symmetry of the Standard Model predicts the extra particle degree of freedom as the Higgs boson. The terms inside the Higgs potential are shown in equation~\ref{eqn:higgspotential}.

Include a shape.
\begin{equation}
\label{eqn:higgspotential}
V(\phi_{h}) = \lambda \nu^2 \phi_{h} ^2  + \lambda \nu \phi_{h} ^3  + \frac{1}{4}\lambda \phi_{h} ^4 
\end{equation}
where $\nu$ corresponds to the vaccum expectation value of the field, determined to be around $246$ \GeV.
The first term gives the Higgs mass, $m_h$, as $ \sqrt{2\lambda}\nu$, measured to be $125.09 \pm 0.24$ \GeV. 
The second term provives an $hhh$ vertex, which corresponds to the trilinear coupling of the Higgs boson. This coupling is also called the Higgs self-coupling.
This means that a two Higgs production, known as di-Higgs or Higgs pair production, can happen through a single Higgs even within the Standard Model. 
Therefore, Higgs pair production is extremely interesting to study because it is the only direct measurement of the $\lambda$ parameter of the Higgs potential.

\section{Standard Model di-Higgs at the LHC}
%https://cds.cern.ch/record/1746004/files/CERN-THESIS-2014-084.pdf
\paragraph{}
Through the Higgs self-coupling, di-Higgs events could be created at the LHC. One single Higgs produced with enough energy could split into two on-shell Higgs. Another way to produce di-Higgs is through a rectangular top/bottom quark loop. These are the two main processes that contribute most to the production at the LHC. Other Feynman diagrams exist but will have higher order loops and thus smaller cross section.
\paragraph{}
Interestingly, according to the SM, the two diagrams have opposite signs and destructively interfere, and hence the production cross section for di-Higgs as a function of the strengh of Higgs-self coupling is a non trivial distribution.
\paragraph{}
Gluon fusion remains by far the dominant production mode. The total cross section at the NNLO with current calculations is 34 \ifb for p-p collisions at 13 TeV.
\paragraph{}
There is much literature about modifications of Higgs self coupling. Using the SM measurements and their precisions, the self coupling parameter could be constrained to an order of magnitude, see \href{https://arxiv.org/abs/1702.07678}{note}.


\section{Beyond the Standard Model Physics di-Higgs at the LHC}
\paragraph{}
BSM physics could significantly enhance the production of di-Higgs at the LHC. This is separated into two categories: non-resonant and resonant productions. The non-resonant production generally refers to modifications of the Higgs couplings, either the Higgs self-coupling or the Higgs-top couplings. Resonant production refers to a particle with invariant mass greater than twice the Higgs mass decays directly into two Higgs bosons. The difference also comes from the distribution of the di-Higgs invariant mass at the truth level: in the non-resonant case typically has no clear peak, whereas in the resonant case the invariant mass distribution forms a peak with model-depdent width.
\paragraph{}

\paragraph{}
With the increased center of mass collison energy, the production cross section grows, particularly for heavy particles above TeV.


\section{Di-Higgs Decay and LHC search perspectives}
\paragraph{}
Di-Higgs decay is the combination of single Higgs decays. The partile width for Higgs to fermions and bosons (one of them is off-shell) at tree level are shown in equation~\ref{eqn:higgswidth}~\cite{Djouadi}:

\begin{equation}
\label{eqn:higgswidth}
\begin{array}{cc}
\Tau(h\to f\bar{f} ) = \frac{N_c \sqrt{2} G_{F} m_{f}^2 m_h}{8 \pi} \\
\Tau(h\to VV^{*} ) = \frac{1}{\pi^2} \int_{0}^{M_H^2} \frac{dq_1^2 M_V \Gamma_V}{(q_1^2-M_V^2)^2 + M_V^2\Gamma_V^2} \int_0^{(M_H - q_1)^2} \frac{dq_2^2 M_V \Gamma_V}{(q_2^2-M_V^2)^2 + M_V^2\Gamma_V^2} \frac{\sqrt{2} \delta_v G_{F} m_h^3}{32\pi} \sqrt{\lambda(q_1^2, q_2^2; m_h^2)} [\lambda(q_1^2, q_2^2; m_h^2) + 12\frac{q_1^2q_2^2}{m_h^2}]
%\Tau(h\to WW ) = \frac{2 \sqrt{2} G_{F} m_{W}^2 m_h}{32 \pi} \frac{\sqrt{1 - x_W}}{x_W^2} (3x_W^2 - 4x_w + 4) \\ %only for on shell
%\Tau(h\to ZZ ) = \frac{\sqrt{2} G_{F} m_{z}^2 m_h}{32 \pi} \frac{\sqrt{1 - x_z}}{x_z^2} (3x_z^2 - 4x_z + 4)
\end{array}
\end{equation}

where $N_c$ is the number os colors, $m_f$ is the fermion mass, $G_{F}$ is the Fermi constant. Hence, given the measured Higgs mass, tthe larges brancing ratio is to the \bbbar. Although there is no direct coupling between $h$ and \gg~ at tree level, this decay can happen through W or top loops.

%\paragraph{}
%CMS latest search result on di-Higgs is also included \href{https://arxiv.org/pdf/1708.08249.pdf}{here}.