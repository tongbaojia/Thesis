%!TEX root = ../dissertation.tex
\begin{savequote}[75mm]
A man who procrastinates in his choosing will inevitably have his choice made for him by circumstance.
\qauthor{Hunter S. Thompson}
\end{savequote}

\chapter{Machine of discovery--The Large Hadron Collider}

\section{Design}
\paragraph{}
The deflection angle $d\theta$, with radius $\rho(s)$ of a particle with charge $Ze$ and momentum $p$ in a magnetic field $B(s)$ is (see this \href{https://indico.cern.ch/event/626458/contributions/2529616/attachments/1434511/2205263/LHC-Ecal.ATLAS-plenary.Mar17.pdf}{talk}):
\begin{equation}
d\theta = \frac{ds}{\rho(s)} = \frac{ZeB(s)ds}{p}
\end{equation}
Ingerate this over the circumference:
\begin{equation}
\oint_C d\theta = 2\pi = \frac{Ze}{p} \oint_C B(s)ds
\end{equation}
Thus the momemtum is:
\begin{equation}
p = \frac{Ze}{2\pi}\oint_C B(s)ds = Z \times 44.7[\frac{MeV}{cTm}] \oint_C B(s)ds
\end{equation}
Given LHC has 1232 14.2m long dipoles with B field 8.33 T, we have p is $7.0$ TeV/c.

\section{Performance}
\paragraph{} 
Above injection energy the relative beam energy uncertainty is $0.1\%$, fully correlated between the 2 beams. No correction must be applied to the online energy values. Energy and uncertainty are determined by the magnetic model, see \href{https://indico.cern.ch/event/626458/contributions/2529616/attachments/1434511/2205263/LHC-Ecal.ATLAS-plenary.Mar17.pdf}{talk}.