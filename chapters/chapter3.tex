%!TEX root = ../dissertation.tex
\begin{savequote}[75mm]
Ugliness is in a way superior to beauty because it lasts.
\qauthor{Serge Gainsbourg}
\end{savequote}
%%%%%%%%%%%%%%
\chapter{Detector}
%We love ATLAS. It even has a writting style \href{guide}{https://cds.cern.ch/record/1110290}.
\paragraph{}
The ATLAS experiment~\cite{PERF-2007-01} at the LHC is a multipurpose particle detector with a forward-backward symmetric cylindrical geometry and a near $4\pi$ coverage insolid angle. 
ATLAS uses a right-handed coordinate system with its origin at the nominal interaction point (IP) in the centre of the detector and the $z$-axis along the beam pipe.
The $x$-axis points from the IP to the centre of the LHC ring, and the $y$-axis points towards the sky.
Cylindrical coordinates $(r,\phi)$ are used in the transverse plane, $\phi$ being the azimuthal angle around the $z$-axis.
The pseudorapidity is defined in terms of the polar angle $\theta$ as $\eta = -\ln \tan(\theta/2)$.
Angular distance is measured in units of $\Delta R \equiv \sqrt{(\Delta\eta)^{2} + (\Delta\phi)^{2}}$.
The ATLAS detector consists of an inner tracking detector (ID) surrounded by a thin superconducting solenoid providing a 2T axial magnetic field, electromagnetic (EM) and hadronic calorimeters, and a muon spectrometer (MS).
%%%%%%%%%%%%%%
\section{Inner Detector}
\paragraph{}
The ID covers the pseudorapidity range $|\eta| < 2.5$.
It has three parts: silicon pixel, silicon microstrip, and straw-tube transition-radiation tracking detectors.
An additional pixel detector layer~\cite{Capeans:1291633}, inserted at a mean radius of 3.3 cm,
is used in the Run-2 data-taking and improves the identification of $b$-jets~\cite{ATL-PHYS-PUB-2015-022}.
\paragraph{}
Material of ATLAS Inner Detector for Run 2 of the LHC. \href{https://cds.cern.ch/record/2260595/files/PERF-2015-07-002.pdf}{note}.
%%%%%%%%%%%%%%
\section{Calorimeter}
\paragraph{}
Lead/liquid-argon (LAr) sampling calorimeters provide EM energy measurements. 
A steel/scintillator-tile hadronic calorimeter covers the central pseudorapidity range ($|\eta| < 1.7$).
The endcap and forward regions are instrumented with LAr calorimeters
for both the EM and hadronic energy measurements up to $|\eta| = 4.9$.
%%%%%%%%%%%%%%
\section{MuonSpectrometer}
\paragraph{}
The muon spectrometer surrounds the calorimeters and includes three large superconducting air-core toroids. 
The field integral of the toroids ranges between 2 and 6 T/m for most of the detector.
The MS includes a system of precision tracking chambers and triggering chambers.
%%%%%%%%%%%%%%
\section{Trigger and Data Aquasition}
\paragraph{}
A dedicated trigger system is used to select events~\cite{ATLAS-TRIGGER}.
The first-level trigger is implemented in hardware and uses the calorimeter and muon detectors to reduce the accepted event rate to 100 kHZ.
This is followed by a software-based high-level trigger that reduces the accepted event rate to 1 kHZ on average.

\paragraph{}
To avoid too high accept rates for certain triggers, the triggers are often prescaled, which means the accepted events get rejected at the prescale. For example, a prescale of two means only every second event passing all trigger conditions gets accepted. 

\paragraph{}
For further trigger information, see \href{http://atlas.web.cern.ch/Atlas/GROUPS/PHYSICS/PAPERS/TRIG-2016-01/}{2015 note} and \href{https://cds.cern.ch/record/2242069/files/ATL-DAQ-PUB-2017-001.pdf}{2016 updates}.
